\documentclass[%
    twoside, 
    a4paper
    ]{article}
\let\MakeUppercase\relax
\linespread{1.2}
\usepackage{geometry}
\geometry{a4paper, twoside, top=2.5cm, bottom=2.5cm, left=3cm, right=3cm, bindingoffset=0.5cm}

\usepackage[T1]{fontenc}
\usepackage[utf8]{inputenc}
\usepackage[main=italian, english]{babel}

\usepackage{csquotes}
% Image
\usepackage{tikz, pgfplots, xargs,fp}
\usepackage{subcaption}
% Specific language
\usepackage{amsmath}
\usepackage{amssymb}
\usepackage{mathtools}
\usepackage{bm}
\usepackage{standalone}
\usepackage{stackrel}
\usepackage{amsthm}
\usepackage{pdfpages}
\usepackage{acronym}
\usepackage[inline]{enumitem}
\usepackage{mathrsfs}  
\usepackage{hyperref}

\newcommand{\imagesPath}{./Images}
\newcommand{\sectionsPath}{./Sections}
\newcommand{\subsectionsPath}{}

\title{Manuale del 470}
\author{Arthur Gurevitch}
\date{}

\acrodef{CE}[CE]{center of effort}
\acrodef{CLR}[CLR]{center of lateral resistance}

% Definizione dei termini specifici per i quali potrà essere rivista la
% traduzione
\def\cunningham{cunningham }
\def\vang{vang }
\def\leech{balumina }
\def\leeches{balumine }
\def\upperleech{parte alta della \leech }
\def\twist{svergolamento }
\def\windward{sopravento }
\def\leeward{sottovento }
\def\draft{grasso }
\def\halyard{drizza }
\def\luff{inferitura }
\def\sheet{scotta }
\def\sheets{scotte }
\def\prebend{prebend }
\def\telltales{telltales }
\def\telltale{telltale }
\def\outhaul{tesabase }
\def\mastpull{spingialbero }
\def\chainplate{chain plate }
\def\twingline{barber }
\def\twinglines{barber }
\def\guy{braccio }
\def\pole{tangone }

\begin{document}
\maketitle
\tableofcontents
\newpage

% !TEX root = ../main.tex
\section{Prefazione}
\label{sec:Prefazione}
Questa guida è la traduzione in italiano della guida di Arthur Gurevitch per il
470. Il testo originale è disponibile all'indirizzo
\url{https://www.waterwind.it/new/images/pdf/Manuale_470.pdf}.

L'ultima revisione dell'autore risale al 23 dicembre 1999. Ogni commento o
suggerimento all'autore può essere inviato all'indirizzo email
\url{hofacker@pointecom.net}.

La traduzione è a cura di Federico Forzano. Ogni commento o suggerimento può
essere inviato all'indirizzo email \url{f.forzano99@gmail.com}.
% !TEX root = ../main.tex
\section{Informazioni sul manuale al 470}
\label{sec:InformazioniSulManuale}
Questo manuale al 470 costituisce un corso intensivo di livello
intermedio-avanzato. Nell'introdurre le molteplici sfaccettature del 470, tra
cui il trapezio, lo spinnaker, la migliorata capacità di bolinare e
le prestazioni in presenza di venti forti, tratteremo di diversi argomenti quali
la messa a punto delle vele, la navigazione con barca piatta, lo sfruttamento
del rollio (virata con rollio) e, di
fondamentale importanza, il lavoro di squadra tra timoniere e prodiere. Il
lettore di tale manuale dovrebbe avere già una conoscenza preliminare di nozioni
quali: messa a punto delle vele in navigazione a bolina, virata con rollio,
regole di precedenza in mare e teoria della vela in generale. Nonostante ciò, l'esperienza
pratica costruita su molte ore di navigazione è l'unico modo per imparare
veramente ad andare a vela.
% Manca la traduzione di questi due punti: sailing the lifts, coming-about on
% the knocks

\subsection{Regole base di navigazione in 470}
\label{sub:RegoleBaseDiNavigazioneIn470}
Poche regole di base sono essenziali per la navigazione in 470:
\begin{itemize}
    \item L'equipaggio è composto esattamente da \textbf{due persone}, un timoniere e un
          prodiere (Numero massimo e minimo di persone a bordo);
    \item Nessun giubbotto di supporto al galleggiamento deve rimanere riposto
          nelle sacche dello spinnaker;
    \item Il tangone dello spinnaker deve essere fissato con delle straps quando
          non è in uso e non deve essere lasciato libero per la barca;
    \item Quando si alza o abbassa la deriva, il vang deve sempre essere lascato.
\end{itemize}

\subsection{Regole per la conservazione a terra della barca}
\label{sub:RegolePerLaConservazioneATerraDellaBarca}
\begin{itemize}
    \item Tutte le cime devono essere raccolte e riposte in modo ordinato.
          Nessuna cima deve essere lasciata sulla superficie calpestabile dello scafo;
    \item Le vele devono essere arrotolate e riposte nelle loro borse;
    \item Le sacche dello spi devono essere rivoltate e lasciate aperte per
          permettere il passaggio dell'aria;
    \item Mai tagliare le cime in eccesso;
    \item Le scotte del fiocco sono sempre lasciate in barca;
    \item tutte le catene e componenti allentate devono essere attaccate
          all'albero o riposte in una sacca trasparente ed aperta;
    \item Ogni danno o rottura deve essere immediatamente riparato;
    \item Ogni problema nell'armare la barca deve essere risolto il più in
          fretta possibile.
\end{itemize}
% !TEX root = ../main.tex

\section{Introduzione}
\label{sec:Introduzione}
Non imparai mai a navigare leggendo un libro. Pertanto, se guardando fuori c'è
bel tempo, lascia perdere quello che stai facendo e vai a navigare. Il tempo che
trascorri su qualsiasi barca probabilmente ti aiuterà molto di più a navigare in
470 rispetto a quanto possa fare un qualsiasi libro di
navigazione. D'altra parte, se hai fatto o stai per fare il tuo primo giro in
470, e hai una serata libera o se fuori dalla finestra il clima è piovoso, potresti trovare utili
ed illuminanti alcuni dei consigli contenuti in questo manuale.
Per diversi anni, nonostante un forte interesse e una comunità molto attiva, non
è stato pubblicato alcun manuale per il 470. Diverse ragioni hanno portato a
questa mancanza di lungimiranza. In primo luogo, scrivere un manuale richiede
del tempo e non offre molti ricavi. Inoltre, ci sono solo pochi aspetti tecnici
del 470 che sono distintivi e unici della barca. La maggior parte delle
informazioni contenute in questo manuale possono essere reperite in qualsiasi
libro di navigazione di livello intermedio. Tuttavia, la barca ha le sue
peculiarità e alcune idee che altrove vengono accennate devono essere
enfatizzate per navigare in 470. Alcune sezioni diventano piuttosto tecniche e
se non sai cosa sia la balumina della randa o una turbolenza, potresti rimanere molto
confuso. Questo manuale contiene pochi diagrammi riguardanti circuiti e messa a
punto della bara. Spesso ci sono diversi modi per raggiungere lo stesso
obiettivo nel montare una barca. L'evoluzione delle tecniche di navigazione e
dei relativi circuiti è talmente rapida che
qualsiasi schema di messa a punto potrebbe essere obsoleto prima ancora che
venga pubblicato. Verrà invece presentato uno schema generale delle cime che
controllano la forma e la regolazione della randa, del fiocco e dello spinnaker.
Se sei confuso su quali cime fanno cosa sulla barca che ti stai preparando a
navigare, passa qualche minuto a terra tirando le cime e osservando cosa
succede. In questo manuale inoltre, verranno presentati pochi trucchi su
come navigare in 470 (ad esempio, lascare il tesabase di circa 5cm con venti da 5
a 8 nodi). Più nel dettaglio invece, verrà presentata la teoria su come le tue vele dovrebbero
lavorare e cosa fanno tutti i comandi sulla forma delle vele. Sta a te capire
come tutto questo si applichi durante la navigazione in acqua.

Infine, il 470 è una barca da regata, molte delle informazioni contenute in
questo manuale riguardano l'ottenimento della massima velocità e prestazioni.
Come verrà evidenziato, il 470 è fondamentalmente una barca
piuttosto facile. Il trucco è farla navigare bene, che è ciò che la
regata richiede. Tuttavia, le tecniche descritte non finalizzate alla sola regata, poiché una
barca 470 ben condotta è notevolmente più divertente.
%
Giunto alla fine, dovresti essere un navigatore abbastanza competente per capire che la perfezione
di forma e funzionalità è il principale obiettivo di navigazione. È importante
ricordare che il 470 è una barca delicata e costosa. La manutenzione di essa
dipende da te. Ci sono troppe cose di cui il personale addetto, se presente, deve
tenere in ordine e che vengono prima delle riparazioni minori sul tuo 470. Eppure,
sono proprio i piccoli incidenti che peggiorano notevolmente la navigazione e
segnano la fine di una barca. Se prevedi di navigare in 470 per anni a venire,
metti in conto di dedicare del tempo per la riparazione e la manutenzione
preventiva della tua deriva. Se desideri vedere cambiamenti e miglioramenti
nell'attrezzatura,
nelle regate o nelle strutture, spetta a te fare in modo che qualcosa venga fatto.
% !TEX root = ../main.tex

\section{La Barca}
\label{sec:LaBarca}
% !TEX root = ../main.tex

\section{Armare e condurre il 470}
\label{sec:RegolazioneDellaBarca}
Dopo pochi secondi di ispezione del 470, ti renderai subito di quanto complicato
e al contempo delicato esso sia. Per garantire una navigazione sicura e
piacevole è opportuno prendere le adeguate precauzioni sia a terra che una volta
in mare. Dedica un po' di tempo a esaminare la barca a terra. Controlla ogni cima per vedere cosa fa e che funzioni correttamente.
Assicurati che non ci siano frizioni in nessuno dei dei circuiti della
barca. Se vedi un problema o un potenziale problema, risolvilo \textbf{prima} di andare a
navigare. Per una barca completamente attrezzata avrai bisogno di:
\begin{itemize}
    \item Randa e fiocco
    \item Tre stecche (1 lunga, 2 corte)
    \item Timone e barra
    \item Imbracatura per il trapezio
    \item Spinnaker e tangone
    \item Due giubbotti di salvataggio
\end{itemize}
Procedi d'apprima ad armare la randa inferendone la base nel boma e attaccando
dunque quest'ultimo all'albero tramite il \emph{corno di trozza} (perno presente
dull'albero). Inserisci a seguire, le stecche nella vela. Le due inferiori (corte)
sono abbastanza "normali" nel loro funzionamento. La stecca superiore lunga va
dall'inferitura della randa alla balumina. La sua tensione può essere regolata
per modificare la forma nelle sezioni superiori
della vela. In generale, con vento leggero e forte, la stecca dovrebbe essere
allentata (ma non così allentata da cadere). Con vento moderato, la stecca
dovrebbe essere abbastanza tesa. La tensione corretta tuttavia, dipende
dalla sua flessibilità, dallo stile di navigazione, dalle onde, dalle
condizioni della vela, dal peso dell'equipaggio e da una serie di altri fattori.
Per regolarla correttamente, devi guardare la tua vela e sperimentare per vedere
cosa ti sembra meglio. Come con tutti gli aggiustamenti della vela, se
hai domande, chiedi a un esperto locale.

Il fiocco 470 contiene al suo interno un cavo d'acciaio che costituisce lo
strallo dell'imbarcazione. Tale cavo non è inserito di base nella vela ma deve
essere inserito e rimosso prima e dopo ogni utilizzo. Quando il fiocco è issato,
l'albero è sostenuto da tale strallo e non cavo d'acciaio più sottile e
permanente (amichevolmente \emph{stralletto}) che si trova su altre
imbarcazioni. Quest'ultimo ha la sola funzione di evitare che l'albero cada
quando la barca non è armata. Il fiocco è dunque realizzato senza i comuni ganci
per lo strallo. Dopo aver attaccato il fiocco ad un grillo posizionato
a prua (solitamente più a poppa dello stralletto), aver attaccato la drizza
all'angolo di penna e le scotte alle balumine, hai la
scelta di tre opzioni per le estremità libere delle scotte del fiocco. La prima
opzione consiste nell'effettuare un nodo Savoia all'estremo. In tal caso, lascia
unaun margine di 6-8 pollici in modo da poterle afferrare se si
tirano sino allo strozzatore. In alternativa, le due estremità delle scotte del fiocco
possono essere legate insieme. Questo sistema continuo riduce l'incertezza del
prodiere poiché esiste solo una scotta da afferrare durante la virata. Una terza
possibilità è quella di far passare la scotta attraverso il pozzetto e legarla
alla maniglia del trapezio. Anche se questo sistema aumenta le probabilità
di inciampare nella barca, dopo poche sessioni di pratica diventa
sorprendentemente veloce ed efficiente. Sorprenderai i tuoi amici quando
vedranno quanto velocemente riesci a trovare la scotta giusta e a cazzare la
vela.

Sia la randa che il fiocco sono sostenute da drizze metalliche con terminazioni
tessili. L'inestensibilità di esse, fa sì che la posizione verticale delle vele
non vengano influenzate dalla
regolazione delle scotte o da venti forti. Le code tessili sono gli unici
elementi che ti permetteranno di non rovinarti le mani nell'issare le vele.
Questo sistema permette di non legare le vele, ma il cavo d'acciaio viene
agganciato ad un blocco di drizza (un elemento che permette di bloccare la
drizza) o alla ghinda. Dopo aver issato le vele e prima di uscire in acqua, scoprirai che è
una buona idea riporre le code delle drizze in una tasca di una delle tasche
dello spi. Le procedure di armatura e conservazione dello spi sono più facili a
vedersi che a spiegarsi (anche se essa verrà descritta dettagliatamente in una sezione successiva).

Alcuni accorgimenti da ricordare sono di controllore la drizza per assicurarti che sia
\emph{in chiaro} (non ingarbugliata) sin in cima all'albero. Controlla che sia completamente sciolta dalle sartie, non sia
bloccata tra uno spigolo e la randa e che lo spi sia riposto
all'esterno del fiocco e delle sartie. Probabilmente scoprirai che lo spi si
arma e ripone più facilmente se è montato prima del fiocco.

Prima di mettere la barca in acqua, attacca il tangone per verificare che il
\emph{carica alto} funzioni e sia a un'altezza approssimativamente corretta.
Dopo aver regolato l'altezza, riponi il tangone in modo sicuro nella barca in
modo che non sganciarsi e perdersi in caso di scuffia.

Il trapezio è un accessorio obbligatorio in tutte le condizioni di vento tranne
che in quelle più leggere. Fatta eccezione che per fini didattici, generalmente è
permesso solo un trapezio per barca. Esistono diverse scuole di pensiero sul suo
utilizzo. La maggior parte dei velisti concorda sul fatto che la cinghia in vita
debba essere abbastanza stretta. Il giubbotto di salvataggio può essere
indossato sia sotto le cinghie delle spalle, dove funge da cuscino e migliora la
vestibilità per equipaggi più piccoli, o sopra il trapezio per evitare che le
cinghie si incastrino sotto il boma. Alcuni velisti regoleranno le cinghie delle
spalle in base alla distribuzione del peso per ricercare il corretto supporto.

Se sei un principiante, mantieni la cinghia stretta per una sensazione di
sicurezza e tranquillità. Nel momento in cui diventerai più a tuo agio, prova ad
allentare la cinghia delle spalle con venti forti in modo da poterti allungare
di più.

Per il resto, l'abbigliamento del prodiere dovrebbe includere pantaloni lunghi e
una maglietta con colletto per evitare tagli e sfregamenti dalle cinghie del
trapezio. Le scarpe sono obbligatorie se intendi mantenere lo stesso paio di piedi
per il resto della tua vita. E, naturalmente, vestiti per il clima, che, in un
470 quando il vento è forte, significa vestiti per un continuo tuffo in acqua
che potrebbe essere piuttosto fredda.

La natura delicata del 470 lo rende particolarmente soggetto a danni quando
ormeggiato al molo.
Per minimizzare l'abuso che queste povere barche devono sopportare, lasca
completamente il
boomvavangng e alza la deriva ogni volta che la barca è ormeggiata. Ammaina le vele
e rimuovi il timone e la barra se la barca rimarrà al molo per più di qualche
minuto. Il dondolio e il rollio della barca potrebbero sbattere il timone contro
la barca e romperlo. Infine, non lasciare mai la tua barca
incustodita al molo, anche con le vele ammainate.

\end{document}