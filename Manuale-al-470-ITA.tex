\documentclass[%
    twoside, 
    a4paper
    ]{article}
\let\MakeUppercase\relax
\linespread{1.2}
\usepackage{geometry}
\geometry{a4paper, twoside, top=2.5cm, bottom=2.5cm, left=3cm, right=3cm, bindingoffset=0.5cm}

\usepackage[T1]{fontenc}
\usepackage[utf8]{inputenc}
\usepackage[main=italian, english]{babel}

\usepackage{csquotes}
% Image
\usepackage{tikz, pgfplots, xargs,fp}
\usepackage{subcaption}
% Specific language
\usepackage{amsmath}
\usepackage{amssymb}
\usepackage{mathtools}
\usepackage{bm}
\usepackage{standalone}
\usepackage{stackrel}
\usepackage{amsthm}
\usepackage{pdfpages}
\usepackage{acronym}
\usepackage[inline]{enumitem}
\usepackage{mathrsfs}  
\usepackage{hyperref}
\usepackage{siunitx}

\newcommand{\imagesPath}{./Images}
\newcommand{\sectionsPath}{./Sections}
\newcommand{\subsectionsPath}{}

\title{Manuale del 470}
\author{Arthur Gurevitch}
\date{}

\acrodef{CE}[CE]{center of effort}
\acrodef{CLR}[CLR]{center of lateral resistance}

\newcommand{\worddef}[4]{
    \expandafter\newcommand\csname #1\endcsname{#2 }
    \expandafter\newcommand\csname the#1\endcsname{#3 #2 }
    \expandafter\newcommand\csname a#1\endcsname{#4 #2 }
}

% Definizione dei termini specifici per i quali potrà essere rivista la
% traduzione
\def\cunningham{cunningham }
\def\vang{vang }
\def\leech{balumina }
\def\leeches{balumine }
\def\upperleech{parte alta della \leech }
\def\twist{svergolamento }
\def\windward{sopravento }
\def\leeward{sottovento }
\def\draft{grasso }
\def\halyard{drizza }
\def\sheet{scotta }
\def\sheets{scotte }
\def\prebend{prebend }
\def\telltales{telltales }
\def\telltale{telltale }
\def\outhaul{tesabase }
\def\mastpull{spingialbero }
\def\chainplate{chain plate }
\def\twingline{barber }
\def\twinglines{barber }
\worddef{guy}{braccio}{il}{un}
\worddef{pole}{tangone}{il}{un}
\worddef{luff}{ralinga}{la}{una}
\worddef{transom}{specchio di poppa}{lo}{uno}
\worddef{spreader}{spreader}{lo}{uno}

\begin{document}
\maketitle
\tableofcontents
\newpage

% !TEX root = ../main.tex
\section{Prefazione}
\label{sec:Prefazione}
Questa guida è la traduzione in italiano della guida di Arthur Gurevitch per il
470. Il testo originale è disponibile all'indirizzo
\url{https://www.waterwind.it/new/images/pdf/Manuale_470.pdf}.

L'ultima revisione dell'autore risale al 23 dicembre 1999. Ogni commento o
suggerimento all'autore può essere inviato all'indirizzo email
\url{hofacker@pointecom.net}.

La traduzione è a cura di Federico Forzano. Ogni commento o suggerimento può
essere inviato all'indirizzo email \url{f.forzano99@gmail.com}.
% !TEX root = ../main.tex
\section{Informazioni sul manuale al 470}
\label{sec:InformazioniSulManuale}
Questo manuale al 470 costituisce un corso intensivo di livello
intermedio-avanzato. Nell'introdurre le molteplici sfaccettature del 470, tra
cui il trapezio, lo spinnaker, la migliorata capacità di bolinare e
le prestazioni in presenza di venti forti, tratteremo di diversi argomenti quali
la messa a punto delle vele, la navigazione con barca piatta, lo sfruttamento
del rollio (virata con rollio) e, di
fondamentale importanza, il lavoro di squadra tra timoniere e prodiere. Il
lettore di tale manuale dovrebbe avere già una conoscenza preliminare di nozioni
quali: messa a punto delle vele in navigazione a bolina, virata con rollio,
regole di precedenza in mare e teoria della vela in generale. Nonostante ciò, l'esperienza
pratica costruita su molte ore di navigazione è l'unico modo per imparare
veramente ad andare a vela.
% Manca la traduzione di questi due punti: sailing the lifts, coming-about on
% the knocks

\subsection{Regole base di navigazione in 470}
\label{sub:RegoleBaseDiNavigazioneIn470}
Poche regole di base sono essenziali per la navigazione in 470:
\begin{itemize}
    \item L'equipaggio è composto esattamente da \textbf{due persone}, un timoniere e un
          prodiere (Numero massimo e minimo di persone a bordo);
    \item Nessun giubbotto di supporto al galleggiamento deve rimanere riposto
          nelle sacche dello spinnaker;
    \item Il tangone dello spinnaker deve essere fissato con delle straps quando
          non è in uso e non deve essere lasciato libero per la barca;
    \item Quando si alza o abbassa la deriva, il vang deve sempre essere lascato.
\end{itemize}

\subsection{Regole per la conservazione a terra della barca}
\label{sub:RegolePerLaConservazioneATerraDellaBarca}
\begin{itemize}
    \item Tutte le cime devono essere raccolte e riposte in modo ordinato.
          Nessuna cima deve essere lasciata sulla superficie calpestabile dello scafo;
    \item Le vele devono essere arrotolate e riposte nelle loro borse;
    \item Le sacche dello spi devono essere rivoltate e lasciate aperte per
          permettere il passaggio dell'aria;
    \item Mai tagliare le cime in eccesso;
    \item Le scotte del fiocco sono sempre lasciate in barca;
    \item tutte le catene e componenti allentate devono essere attaccate
          all'albero o riposte in una sacca trasparente ed aperta;
    \item Ogni danno o rottura deve essere immediatamente riparato;
    \item Ogni problema nell'armare la barca deve essere risolto il più in
          fretta possibile.
\end{itemize}
% !TEX root = ../main.tex

\section{Introduzione}
\label{sec:Introduzione}
Non imparai mai a navigare leggendo un libro. Pertanto, se guardando fuori c'è
bel tempo, lascia perdere quello che stai facendo e vai a navigare. Il tempo che
trascorri su qualsiasi barca probabilmente ti aiuterà molto di più a navigare in
470 rispetto a quanto possa fare un qualsiasi libro di
navigazione. D'altra parte, se hai fatto o stai per fare il tuo primo giro in
470, e hai una serata libera o se fuori dalla finestra il clima è piovoso, potresti trovare utili
ed illuminanti alcuni dei consigli contenuti in questo manuale.
Per diversi anni, nonostante un forte interesse e una comunità molto attiva, non
è stato pubblicato alcun manuale per il 470. Diverse ragioni hanno portato a
questa mancanza di lungimiranza. In primo luogo, scrivere un manuale richiede
del tempo e non offre molti ricavi. Inoltre, ci sono solo pochi aspetti tecnici
del 470 che sono distintivi e unici della barca. La maggior parte delle
informazioni contenute in questo manuale possono essere reperite in qualsiasi
libro di navigazione di livello intermedio. Tuttavia, la barca ha le sue
peculiarità e alcune idee che altrove vengono accennate devono essere
enfatizzate per navigare in 470. Alcune sezioni diventano piuttosto tecniche e
se non sai cosa sia la balumina della randa o una turbolenza, potresti rimanere molto
confuso. Questo manuale contiene pochi diagrammi riguardanti circuiti e messa a
punto della bara. Spesso ci sono diversi modi per raggiungere lo stesso
obiettivo nel montare una barca. L'evoluzione delle tecniche di navigazione e
dei relativi circuiti è talmente rapida che
qualsiasi schema di messa a punto potrebbe essere obsoleto prima ancora che
venga pubblicato. Verrà invece presentato uno schema generale delle cime che
controllano la forma e la regolazione della randa, del fiocco e dello spinnaker.
Se sei confuso su quali cime fanno cosa sulla barca che ti stai preparando a
navigare, passa qualche minuto a terra tirando le cime e osservando cosa
succede. In questo manuale inoltre, verranno presentati pochi trucchi su
come navigare in 470 (ad esempio, lascare il tesabase di circa 5cm con venti da 5
a 8 nodi). Più nel dettaglio invece, verrà presentata la teoria su come le tue vele dovrebbero
lavorare e cosa fanno tutti i comandi sulla forma delle vele. Sta a te capire
come tutto questo si applichi durante la navigazione in acqua.

Infine, il 470 è una barca da regata, molte delle informazioni contenute in
questo manuale riguardano l'ottenimento della massima velocità e prestazioni.
Come verrà evidenziato, il 470 è fondamentalmente una barca
piuttosto facile. Il trucco è farla navigare bene, che è ciò che la
regata richiede. Tuttavia, le tecniche descritte non finalizzate alla sola regata, poiché una
barca 470 ben condotta è notevolmente più divertente.
%
Giunto alla fine, dovresti essere un navigatore abbastanza competente per capire che la perfezione
di forma e funzionalità è il principale obiettivo di navigazione. È importante
ricordare che il 470 è una barca delicata e costosa. La manutenzione di essa
dipende da te. Ci sono troppe cose di cui il personale addetto, se presente, deve
tenere in ordine e che vengono prima delle riparazioni minori sul tuo 470. Eppure,
sono proprio i piccoli incidenti che peggiorano notevolmente la navigazione e
segnano la fine di una barca. Se prevedi di navigare in 470 per anni a venire,
metti in conto di dedicare del tempo per la riparazione e la manutenzione
preventiva della tua deriva. Se desideri vedere cambiamenti e miglioramenti
nell'attrezzatura,
nelle regate o nelle strutture, spetta a te fare in modo che qualcosa venga fatto.
% !TEX root = ../main.tex

\section{La Barca}
\label{sec:LaBarca}
% !TEX root = ../main.tex

\section{Armare e condurre il 470}
\label{sec:RegolazioneDellaBarca}
Dopo pochi secondi di ispezione del 470, ti renderai subito di quanto complicato
e al contempo delicato esso sia. Per garantire una navigazione sicura e
piacevole è opportuno prendere le adeguate precauzioni sia a terra che una volta
in mare. Dedica un po' di tempo a esaminare la barca a terra. Controlla ogni cima per vedere cosa fa e che funzioni correttamente.
Assicurati che non ci siano frizioni in nessuno dei dei circuiti della
barca. Se vedi un problema o un potenziale problema, risolvilo \textbf{prima} di andare a
navigare. Per una barca completamente attrezzata avrai bisogno di:
\begin{itemize}
    \item Randa e fiocco
    \item Tre stecche (1 lunga, 2 corte)
    \item Timone e barra
    \item Imbracatura per il trapezio
    \item Spinnaker e tangone
    \item Due giubbotti di salvataggio
\end{itemize}
Procedi d'apprima ad armare la randa inferendone la base nel boma e attaccando
dunque quest'ultimo all'albero tramite il \emph{corno di trozza} (perno presente
dull'albero). Inserisci a seguire, le stecche nella vela. Le due inferiori (corte)
sono abbastanza "normali" nel loro funzionamento. La stecca superiore lunga va
dall'inferitura della randa alla balumina. La sua tensione può essere regolata
per modificare la forma nelle sezioni superiori
della vela. In generale, con vento leggero e forte, la stecca dovrebbe essere
allentata (ma non così allentata da cadere). Con vento moderato, la stecca
dovrebbe essere abbastanza tesa. La tensione corretta tuttavia, dipende
dalla sua flessibilità, dallo stile di navigazione, dalle onde, dalle
condizioni della vela, dal peso dell'equipaggio e da una serie di altri fattori.
Per regolarla correttamente, devi guardare la tua vela e sperimentare per vedere
cosa ti sembra meglio. Come con tutti gli aggiustamenti della vela, se
hai domande, chiedi a un esperto locale.

Il fiocco 470 contiene al suo interno un cavo d'acciaio che costituisce lo
strallo dell'imbarcazione. Tale cavo non è inserito di base nella vela ma deve
essere inserito e rimosso prima e dopo ogni utilizzo. Quando il fiocco è issato,
l'albero è sostenuto da tale strallo e non cavo d'acciaio più sottile e
permanente (amichevolmente \emph{stralletto}) che si trova su altre
imbarcazioni. Quest'ultimo ha la sola funzione di evitare che l'albero cada
quando la barca non è armata. Il fiocco è dunque realizzato senza i comuni ganci
per lo strallo. Dopo aver attaccato il fiocco ad un grillo posizionato
a prua (solitamente più a poppa dello stralletto), aver attaccato la drizza
all'angolo di penna e le scotte alle balumine, hai la
scelta di tre opzioni per le estremità libere delle scotte del fiocco. La prima
opzione consiste nell'effettuare un nodo Savoia all'estremo. In tal caso, lascia
unaun margine di 6-8 pollici in modo da poterle afferrare se si
tirano sino allo strozzatore. In alternativa, le due estremità delle scotte del fiocco
possono essere legate insieme. Questo sistema continuo riduce l'incertezza del
prodiere poiché esiste solo una scotta da afferrare durante la virata. Una terza
possibilità è quella di far passare la scotta attraverso il pozzetto e legarla
alla maniglia del trapezio. Anche se questo sistema aumenta le probabilità
di inciampare nella barca, dopo poche sessioni di pratica diventa
sorprendentemente veloce ed efficiente. Sorprenderai i tuoi amici quando
vedranno quanto velocemente riesci a trovare la scotta giusta e a cazzare la
vela.

Sia la randa che il fiocco sono sostenute da drizze metalliche con terminazioni
tessili. L'inestensibilità di esse, fa sì che la posizione verticale delle vele
non vengano influenzate dalla
regolazione delle scotte o da venti forti. Le code tessili sono gli unici
elementi che ti permetteranno di non rovinarti le mani nell'issare le vele.
Questo sistema permette di non legare le vele, ma il cavo d'acciaio viene
agganciato ad un blocco di drizza (un elemento che permette di bloccare la
drizza) o alla ghinda. Dopo aver issato le vele e prima di uscire in acqua, scoprirai che è
una buona idea riporre le code delle drizze in una tasca di una delle tasche
dello spi. Le procedure di armatura e conservazione dello spi sono più facili a
vedersi che a spiegarsi (anche se essa verrà descritta dettagliatamente in una sezione successiva).

Alcuni accorgimenti da ricordare sono di controllore la drizza per assicurarti che sia
\emph{in chiaro} (non ingarbugliata) sin in cima all'albero. Controlla che sia completamente sciolta dalle sartie, non sia
bloccata tra uno spigolo e la randa e che lo spi sia riposto
all'esterno del fiocco e delle sartie. Probabilmente scoprirai che lo spi si
arma e ripone più facilmente se è montato prima del fiocco.

Prima di mettere la barca in acqua, attacca il tangone per verificare che il
\emph{carica alto} funzioni e sia a un'altezza approssimativamente corretta.
Dopo aver regolato l'altezza, riponi il tangone in modo sicuro nella barca in
modo che non sganciarsi e perdersi in caso di scuffia.

Il trapezio è un accessorio obbligatorio in tutte le condizioni di vento tranne
che in quelle più leggere. Fatta eccezione che per fini didattici, generalmente è
permesso solo un trapezio per barca. Esistono diverse scuole di pensiero sul suo
utilizzo. La maggior parte dei velisti concorda sul fatto che la cinghia in vita
debba essere abbastanza stretta. Il giubbotto di salvataggio può essere
indossato sia sotto le cinghie delle spalle, dove funge da cuscino e migliora la
vestibilità per equipaggi più piccoli, o sopra il trapezio per evitare che le
cinghie si incastrino sotto il boma. Alcuni velisti regoleranno le cinghie delle
spalle in base alla distribuzione del peso per ricercare il corretto supporto.

Se sei un principiante, mantieni la cinghia stretta per una sensazione di
sicurezza e tranquillità. Nel momento in cui diventerai più a tuo agio, prova ad
allentare la cinghia delle spalle con venti forti in modo da poterti allungare
di più.

Per il resto, l'abbigliamento del prodiere dovrebbe includere pantaloni lunghi e
una maglietta con colletto per evitare tagli e sfregamenti dalle cinghie del
trapezio. Le scarpe sono obbligatorie se intendi mantenere lo stesso paio di piedi
per il resto della tua vita. E, naturalmente, vestiti per il clima, che, in un
470 quando il vento è forte, significa vestiti per un continuo tuffo in acqua
che potrebbe essere piuttosto fredda.

La natura delicata del 470 lo rende particolarmente soggetto a danni quando
ormeggiato al molo.
Per minimizzare l'abuso che queste povere barche devono sopportare, lasca
completamente il
boomvavangng e alza la deriva ogni volta che la barca è ormeggiata. Ammaina le vele
e rimuovi il timone e la barra se la barca rimarrà al molo per più di qualche
minuto. Il dondolio e il rollio della barca potrebbero sbattere il timone contro
la barca e romperlo. Infine, non lasciare mai la tua barca
incustodita al molo, anche con le vele ammainate.
%!TEX root = ../main.tex

\section{La scuffia}
\label{sec:scuffia}
Se non scuffi main in 470, probabilmento stai facendo qualcosa di sbagliato!
Potrebbe essere che non stai spingendo la tua barca quanto potresti, o che non
esci in condizioni difficili come saresti in grado di affrontare, o che sei
eccessivamente cauto e non permetti alla barca di esprimere tutto il suo
potenziale. Uno dei piaceri di navigare su un 470 è la gestibilità della barca
dopo una scuffia. Molte barche sono impossibili da raddrizzare o tendono a
restare allagate dopo una scuffia, ma un 470 si rialza subito, quasi
completamente asciutto con solo un po' di pratica. La prima scuffia è sempre la
più difficile, e scuffiare con vento forte può essere piuttosto snervante,
indipendentemente da quanta pratica si abbia.

Il 470, ahimè, non è molto felice quando è sdraiato su un fianco. Preferirebbe
essere dritto o completamente capovolto. È quindi molto importante muoversi
rapidamente dopo ogni scuffia, se si vuole evitare il capovolgimento completo.
Il modo più sicuro per far capovolgere completamente un 470 dopo una scuffia è
sedersi sul bordo. Poiché il prodiere potrebbe aver bisogno di un po' di
tempo per sganciarsi, in molti casi sarà lui il colpevole. Pertanto, per ridurre
il tempo e lo sforzo necessari per rimettersi a navigare, è fondamentale far
scendere subito il prodiere dal bordo barca. Questo di solito implica tuffarsi
in acqua dal lato sottovento. Prima di prendere la decisione finale di tuffarsi,
il prodiere dovrebbe dare un'occhiata veloce al timoniere, se possibile, per
capire cosa ha intezione di fare.

Ci sono situazioni in cui conviene che il timoniere si tuffi in acqua e il
prodiere vada direttamente verso il centro della deriva. Questo accade quando il
timoniere è seduto \leeward o sbilanciato mentre la barca si inclina. Tuttavia,
nella maggior parte dei casi, il prodiere dovrebbe tuffarsi in acqua mentre lo
skipper si dirige verso la deriva il più velocemente possibile.

\emph{Avvertimento:} i 470 hanno svuotatori (bailers) con bordi estremamente
affilati che tendono a graffiare le gambe. Fai attenzione quando sali sulla
deriva e chiudi subito lo svuotatore con il palmo della mano prima di fare
qualsiasi altra cosa. Una volta che qualcuno è sulla deriva, la barca si
stabilizzerà e non si inclinerà ulteriormente. Dopo essere caduto in acqua,
il prodiere dovrebbe sganciarsi, districarsi dalle cime galleggianti intorno e
uscire dalla vela.

È buona norma urlare per sapere dove si trova l'altro. La comunicazione
diventa molto importante quando non riesci a vedere il tuo compagno attraverso
la barca.

A questo punto, la deriva dovrebbe essere estesa completamente verso il basso
per avere la massima leva. Se la tua deriva ha un sistema a frizione, è facile
estrarla dal fondo della barca. Sia il timoniere che il prodiere devono
ricordarsi di liberare la \sheet della randa, del fiocco, dello spinnaker e il
\vang, poiché tutti questi tendono a rendere la barca ingovernabile mentre viene
raddrizzata. La persona sulla deriva può dunque indietreggiare fino al bordo e
iniziare a tirare su la barca. Tenere la maniglia del trapezio ti dà maggiore
stabilità e leva mentre indietreggi. Quando la barca si solleva, l'equipaggio in
acqua dovrebbe agganciare il braccio intorno alla barra del carrello o sotto una
cinghia. Una volta che la barca comincia a sollevarsi, prende velocità e
potrebbe ribaltarsi completamente a meno che non venga fermata. Agganciandosi al
carrello, puoi mettere il tuo peso sul serbatoio (seduta laterale) non appena la
barca raggiunge la verticale e fermarne il rollio (anche tenersi alla maniglia
del trapezio funziona bene). Spesso la barca comincerà a navigare prima che
entrambi siate tornati a bordo, quindi sarà importante che la persona più vicina
alla poppa sia pronta a prendere il timone prima di risalire dall'acqua.

Non importa da che parte si trovi la barca quando viene raddrizzata. Appena la
testa della vela esce dall'acqua, il vento farà ruotare la barca e la farà
sollevare immediatamente. La persona in acqua deve essere molto veloce, poiché
sarà necessario tirare con forza per evitare che la barca si ribalti nell'altra
direzione.

Se la barca si capovolge completamente, sia il timoniere che il prodiere
potrebbero dover salire sul fondo della barca e tirare la deriva. Se la deriva è
scivolata all'interno della fessura, sarà necessario che uno di voi nuoti sotto
la barca e la spinga fuori. Una volta che la deriva è completamente estesa,
entrambi i velisti dovrebbero tirare delicatamente la barca inclinando il peso
sulla deriva. Saltare troppo forte può danneggiare sia la deriva che lo scafo.
Quando la punta dell'albero è appena in superficie, uno di voi deve entrare in
acqua e nuotare verso il lato in basso. La persona sulla deriva dovrebbe
stabilizzare la barca finché l'altro non è in posizione sul lato basso. A questo
punto, è esattamente come una scuffia normale.

Molte, se non la maggior parte, delle scuffie avvengono navigando in poppa,
quando è probabile che lo spinnaker sia issato. Quando ciò accade, la vela tende
a impigliarsi in modo intricato tra le crocette, le sartie o qualsiasi altra
cosa a portata di mano. Cerca di liberare completamente la vela prima di
raddrizzare la barca. Potrebbe essere necessario staccare la vela dalla sua
drizza e/o dalle scotte per districare il groviglio.

\emph{Nota:} Fissa sempre la drizza a qualcosa o almeno fai un nodo che prevenga
che essa rientri nell'albero.

Un modo sicuro per strappare uno spinnaker è strattonarlo quando è impigliato
dopo aver raddrizzato la barca. Se lo tiri su e non si libera completamente,
cerca di limitare i movimenti bruschi il più possibile mentre cerchi
delicatamente di liberare la vela.

Se scuffi in acque poco profonde, è fondamentale sollevare la punta dell'albero
dal fango lentamente per evitare di piegarlo. La chiave è evitare sforzi
eccessivi o salti sulla deriva. Sii paziente e il vento alla fine aiuterà la
barca a uscire dal fango, permettendo di raddrizzarla con poco sforzo. Fai
particolarmente attenzione quando scuffi vicino alla riva. Prima di rendertene
conto, potresti trovarti spinto in acque poco profonde.
%
%!TEX root = ../main.tex

\section{Appendice}
\label{sec:appendice}
Anche prima di mettere la barca in acqua, ci sono numerose decisioni che i
velisti da regata devono prendere per preparare le loro imbarcazioni. C'è,
ovviamente, la scelta dell'albero, che si basa principalmente sulle sue
caratteristiche di flessibilità (sia in avanti che all'indietro e da lato a
lato) e sul peso. A questo punto, probabilmente hai poche possibilità di scelta
dell'albero.

Un'altra cosa sono le crocette. Con le crocette si possono fare due regolazioni:
lunghezza e angolo. Nessuna di queste dovrebbe essere modificata fino a quando
non sei abbastanza sicuro di quello che stai facendo. I dettagli della
regolazione delle crocette vanno oltre lo scopo di questo manuale. Tuttavia, ci
sono un paio di concetti di base da tenere a mente. Crocette più lunghe rendono
l'albero più rigido lateralmente. Questo rende l'armo più potente ed è
preferito da equipaggi più pesanti, da coloro che non navigano negli oceani dove
ci sono grandi onde da affrontare, e da quelli che hanno vele appositamente
tagliate e piatte. L'angolo delle crocette influisce sulle caratteristiche di
curvatura in avanti e indietro. Fissare le punte delle crocette più verso poppa
aumenta la quantità di curvatura dell'albero causata dalla tensione sulle sartie
e dalla pressione della \sheet della randa.

Quando si configura l'albero, ci sono due aspetti da tenere a mente: la tensione
del rig e l'inclinazione. L'inclinazione non è la stessa cosa della curvatura.
L'inclinazione influisce sulla posizione del \ac{CE} e sul timone della
barca. Inclinare l'albero all'indietro sposta il \ac{CE} all'indietro e aumenta la
tendenza a sbandare al vento. L'inclinazione influisce anche sulla forma della
randa, determinando la tensione sulla \sheet della randa quando il boma è tirato
completamente dentro.

Infine, l'inclinazione dell'albero influisce sulla regolazione del fiocco,
cambiando la larghezza della fessura tra la randa e il fiocco. Inclinare
l'albero verso poppa ha lo stesso effetto sul fiocco di spostare la scotta del
fiocco verso poppa. Questo apre la fessura. Molti esperti hanno fatto
esperimenti per determinare la migliore inclinazione. Il consenso generale è che
l'albero dovrebbe essere impostato in modo che un metro a nastro fissato alla
drizza della randa e issato fino all'estremità dell'albero legga $6.76\si{m}$ al
\transom con la barca ghindata.
%
Nota che questo comporta un'inclinazione dell'albero verso poppa rispetto alla
verticale.

Con venti più forti, i ``grandi'' hanno stabilito che l'albero può essere
inclinato leggermente all'indietro e la misura dell'inclinazione ridotta a
$6.68\si{m}$ in condizioni di sovrapotenziamento.

La prossima decisione riguarda la tensione del rig. Una maggiore tensione sulle
sartie e sullo strallo impedisce all'albero di muoversi in modo incontrollato in
caso di onda, ``precurva'' l'albero per limitare la compressione e influisce sia
sull'entrata che sulla catenaria del fiocco (la catenaria è la forma che assume
una corda inestensibile appesa a due estremi e dipende dalla tensione ad essa
applicata). Mantenere l'armo saldamente ancorato è chiaramente un vantaggio
in condizioni di maltempo, ma l'importanza della tensione del rig sulla
forma delle vele più che rilevante in tutte le condizioni.

La maggior parte delle rande del 470 sono tagliate aspettandosi una determinata
curvatura dell'albero. Questo consente loro di funzionare nel modo più
efficiente possibile in una vasta gamma di condizioni di vento. In molte
situazioni, la curvatura dell'albero è indotta dalla tensione sulla scotta della
randa e sul vang, che, tirando sul bordo d'uscita della randa, tira indietro la
punta dell'albero. Tuttavia, in condizioni di vento leggero, la tensione sulla
randa dovrebbe essere molto limitata e il vang non dovrebbe essere utilizzato
affatto. La lieve curvatura causata dalla tensione delle sartie è necessaria
affinché la vela si posizioni correttamente.

Sartie tese causano anche una tensione sullo strallo, che impedisce al\theluff
del fiocco di cedere durante le raffiche. L'incurvarsi del\theluff riduce sia la
velocità che il controllo, poiché fa sì che il fiocco si gonfi di più durante le
raffiche. Con uno strallo teso, la forma del fiocco rimane costante durante
raffiche e bolle d'aria (vuoti di vento).

La tensione del rig influisce anche sull'entrata del fiocco. Troppo poca
tensione provoca un'entrata troppo piena e la barca risulta lenta. Troppa
tensione rende molto difficile mantenere la barca in rotta.
Fortunatamente, è quasi impossibile sovraccaricare il rig del 470. La maggior
parte dei velisti competitivi scopre che un rig super teso è estremamente
veloce e attrezzano le loro barche con un potente sistema di pulegge e cime per
tensionare lo strallo (tecnicamente la \textbf{ghinda}).

La tensione del rig e l'inclinazione dell'albero sono ovviamente influenzate
dagli stessi aggiustamenti. È importante modificare sia la lunghezza delle
sartie che la drizza del fiocco in linea con il tuo rig. Una volta impostate le
sartie, aumentare la tensione sullo strallo ha due effetti: 1) raddrizza
l'albero tirandolo in avanti; e 2) tira le sartie, aumentando
così la tensione del rig.

Piccole modifiche allo strallo hanno un effetto maggiore sull'inclinazione che
sulla tensione del rig. Cambiare la lunghezza delle sartie scegliendo un nuovo
foro sulla piastra di base ha un effetto maggiore sulla tensione che
sull'inclinazione. Pertanto, quando prepari la tua barca, è meglio considerare
lo strallo come una configurazione dell'inclinazione e le sartie della tensione.

Ricorda che per impostare correttamente il tuo armo, entrambi devono essere
regolati insieme. Ad esempio, se trovi necessario inclinare l'albero
all'indietro, lo strallo è il controllo principale, ma le sartie devono essere
fissate in un foro più basso per eliminare il ``gioco'' e mantenere la tensione
corretta. Alcuni dei migliori velisti da regata hanno perni a rilascio rapido
sulle loro sartie in modo da poter riconfigurarle in acqua, se necessario.

Le vele devono avere una forma aerodinamica per poter risalire il vento. Mettere
un foglio piatto sull'albero non funziona. Nel corso degli anni, i velai hanno
ideato due tecniche di costruzione per creare e controllare la forma della vela.
La broadseam (letteralmente cucitura larga) forza la profondità e la posizione
del profilo nella vela. Come suggerisce il nome, la broadseam comporta la
sovrapposizione delle cuciture dei pannelli nella vela.
%
La seconda tecnica, la curva del\theluff, consente di modificare facilmente la
pienezza della vela mentre la barca è in movimento. La curvatura viene forzata
in una vela con \aluff curva che è posizionata su un albero relativamente
dritto. Consentendo all'albero di piegarsi, si fa sì che la vela diventi sempre
più piatta fino a quando la quantità di curvatura non eguaglia la quantità di
curva tagliata nel\theluff e la curvatura del\theluff non contribuisce più alla
pienezza della vela. Piegare l'albero consente anche al \leech di ridurre lo
svergolamento, depotenziando la randa. L'albero può essere piegato oltre la
curva del\theluff, causando l'inversione della vela e grandi pieghe diagonali
che si irradiano dall'angolo di scotta fino allo \spreader. L'inversione dovrebbe
essere evitata in tutte le condizioni tranne quelle di vento molto forte. In
condizioni estreme, una randa invertita, sebbene quasi completamente
depotenziata, consente comunque alla barca di tenere la rotta e alla randa di
non essere tirata troppo.

Le rande dei 470, così come la maggior parte delle altre derive, presentano sia
broadseam che curve del\theluff per creare vele che siano efficaci in
un'ampia gamma di condizioni di vento e mare.

\end{document}