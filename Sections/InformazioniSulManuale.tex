% !TEX root = ../main.tex
\section{Informazioni sul manuale al 470}
\label{sec:InformazioniSulManuale}
Questo manuale al 470 costituisce un corso intensivo di livello
intermedio-avanzato. Nell'introdurre le molteplici sfaccettature del 470, tra
cui il trapezio, lo spinnaker, la migliorata capacità di bolinare e
le prestazioni in presenza di venti forti, tratteremo di diversi argomenti quali
la messa a punto delle vele, la navigazione con barca piatta, lo sfruttamento
del rollio (virata con rollio) e, di
fondamentale importanza, il lavoro di squadra tra timoniere e prodiere. Il
lettore di tale manuale dovrebbe avere già una conoscenza preliminare di nozioni
quali: messa a punto delle vele in navigazione a bolina, virata con rollio,
regole di precedenza in mare e teoria della vela in generale. Nonostante ciò, l'esperienza
pratica costruita su molte ore di navigazione è l'unico modo per imparare
veramente ad andare a vela.
% Manca la traduzione di questi due punti: sailing the lifts, coming-about on
% the knocks

\subsection{Regole base di navigazione in 470}
\label{sub:RegoleBaseDiNavigazioneIn470}
Poche regole di base sono essenziali per la navigazione in 470:
\begin{itemize}
    \item L'equipaggio è composto esattamente da \textbf{due persone}, un timoniere e un
          prodiere (Numero massimo e minimo di persone a bordo);
    \item Nessun giubbotto di supporto al galleggiamento deve rimanere riposto
          nelle sacche dello spinnaker;
    \item Il tangone dello spinnaker deve essere fissato con delle straps quando
          non è in uso e non deve essere lasciato libero per la barca;
    \item Quando si alza o abbassa la deriva, il vang deve sempre essere lascato.
\end{itemize}

\subsection{Regole per la conservazione a terra della barca}
\label{sub:RegolePerLaConservazioneATerraDellaBarca}
\begin{itemize}
    \item Tutte le cime devono essere raccolte e riposte in modo ordinato.
          Nessuna cima deve essere lasciata sulla superficie calpestabile dello scafo;
    \item Le vele devono essere arrotolate e riposte nelle loro borse;
    \item Le sacche dello spi devono essere rivoltate e lasciate aperte per
          permettere il passaggio dell'aria;
    \item Mai tagliare le cime in eccesso;
    \item Le scotte del fiocco sono sempre lasciate in barca;
    \item tutte le catene e componenti allentate devono essere attaccate
          all'albero o riposte in una sacca trasparente ed aperta;
    \item Ogni danno o rottura deve essere immediatamente riparato;
    \item Ogni problema nell'armare la barca deve essere risolto il più in
          fretta possibile.
\end{itemize}