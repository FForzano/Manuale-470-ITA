%!TEX root = ../main.tex

\section{Appendice}
\label{sec:appendice}
Anche prima di mettere la barca in acqua, ci sono numerose decisioni che i
velisti da regata devono prendere per preparare le loro imbarcazioni. C'è,
ovviamente, la scelta dell'albero, che si basa principalmente sulle sue
caratteristiche di flessibilità (sia in avanti che all'indietro e da lato a
lato) e sul peso. A questo punto, probabilmente hai poche possibilità di scelta
dell'albero.

Un'altra cosa sono le crocette. Con le crocette si possono fare due regolazioni:
lunghezza e angolo. Nessuna di queste dovrebbe essere modificata fino a quando
non sei abbastanza sicuro di quello che stai facendo. I dettagli della
regolazione delle crocette vanno oltre lo scopo di questo manuale. Tuttavia, ci
sono un paio di concetti di base da tenere a mente. Crocette più lunghe rendono
l'albero più rigido lateralmente. Questo rende l'armo più potente ed è
preferito da equipaggi più pesanti, da coloro che non navigano negli oceani dove
ci sono grandi onde da affrontare, e da quelli che hanno vele appositamente
tagliate e piatte. L'angolo delle crocette influisce sulle caratteristiche di
curvatura in avanti e indietro. Fissare le punte delle crocette più verso poppa
aumenta la quantità di curvatura dell'albero causata dalla tensione sulle sartie
e dalla pressione della \sheet della randa.

Quando si configura l'albero, ci sono due aspetti da tenere a mente: la tensione
del rig e l'inclinazione. L'inclinazione non è la stessa cosa della curvatura.
L'inclinazione influisce sulla posizione del \ac{CE} e sul timone della
barca. Inclinare l'albero all'indietro sposta il \ac{CE} all'indietro e aumenta la
tendenza a sbandare al vento. L'inclinazione influisce anche sulla forma della
randa, determinando la tensione sulla \sheet della randa quando il boma è tirato
completamente dentro.

Infine, l'inclinazione dell'albero influisce sulla regolazione del fiocco,
cambiando la larghezza della fessura tra la randa e il fiocco. Inclinare
l'albero verso poppa ha lo stesso effetto sul fiocco di spostare la scotta del
fiocco verso poppa. Questo apre la fessura. Molti esperti hanno fatto
esperimenti per determinare la migliore inclinazione. Il consenso generale è che
l'albero dovrebbe essere impostato in modo che un metro a nastro fissato alla
drizza della randa e issato fino all'estremità dell'albero legga $6.76\si{m}$ al
\transom con la barca ghindata.
%
Nota che questo comporta un'inclinazione dell'albero verso poppa rispetto alla
verticale.

Con venti più forti, i ``grandi'' hanno stabilito che l'albero può essere
inclinato leggermente all'indietro e la misura dell'inclinazione ridotta a
$6.68\si{m}$ in condizioni di sovrapotenziamento.

La prossima decisione riguarda la tensione del rig. Una maggiore tensione sulle
sartie e sullo strallo impedisce all'albero di muoversi in modo incontrollato in
caso di onda, ``precurva'' l'albero per limitare la compressione e influisce sia
sull'entrata che sulla catenaria del fiocco (la catenaria è la forma che assume
una corda inestensibile appesa a due estremi e dipende dalla tensione ad essa
applicata). Mantenere l'armo saldamente ancorato è chiaramente un vantaggio
in condizioni di maltempo, ma l'importanza della tensione del rig sulla
forma delle vele più che rilevante in tutte le condizioni.

La maggior parte delle rande del 470 sono tagliate aspettandosi una determinata
curvatura dell'albero. Questo consente loro di funzionare nel modo più
efficiente possibile in una vasta gamma di condizioni di vento. In molte
situazioni, la curvatura dell'albero è indotta dalla tensione sulla scotta della
randa e sul vang, che, tirando sul bordo d'uscita della randa, tira indietro la
punta dell'albero. Tuttavia, in condizioni di vento leggero, la tensione sulla
randa dovrebbe essere molto limitata e il vang non dovrebbe essere utilizzato
affatto. La lieve curvatura causata dalla tensione delle sartie è necessaria
affinché la vela si posizioni correttamente.

Sartie tese causano anche una tensione sullo strallo, che impedisce al\theluff
del fiocco di cedere durante le raffiche. L'incurvarsi del\theluff riduce sia la
velocità che il controllo, poiché fa sì che il fiocco si gonfi di più durante le
raffiche. Con uno strallo teso, la forma del fiocco rimane costante durante
raffiche e bolle d'aria (vuoti di vento).

La tensione del rig influisce anche sull'entrata del fiocco. Troppo poca
tensione provoca un'entrata troppo piena e la barca risulta lenta. Troppa
tensione rende molto difficile mantenere la barca in rotta.
Fortunatamente, è quasi impossibile sovraccaricare il rig del 470. La maggior
parte dei velisti competitivi scopre che un rig super teso è estremamente
veloce e attrezzano le loro barche con un potente sistema di pulegge e cime per
tensionare lo strallo (tecnicamente la \textbf{ghinda}).

La tensione del rig e l'inclinazione dell'albero sono ovviamente influenzate
dagli stessi aggiustamenti. È importante modificare sia la lunghezza delle
sartie che la drizza del fiocco in linea con il tuo rig. Una volta impostate le
sartie, aumentare la tensione sullo strallo ha due effetti: 1) raddrizza
l'albero tirandolo in avanti; e 2) tira le sartie, aumentando
così la tensione del rig.

Piccole modifiche allo strallo hanno un effetto maggiore sull'inclinazione che
sulla tensione del rig. Cambiare la lunghezza delle sartie scegliendo un nuovo
foro sulla piastra di base ha un effetto maggiore sulla tensione che
sull'inclinazione. Pertanto, quando prepari la tua barca, è meglio considerare
lo strallo come una configurazione dell'inclinazione e le sartie della tensione.

Ricorda che per impostare correttamente il tuo armo, entrambi devono essere
regolati insieme. Ad esempio, se trovi necessario inclinare l'albero
all'indietro, lo strallo è il controllo principale, ma le sartie devono essere
fissate in un foro più basso per eliminare il ``gioco'' e mantenere la tensione
corretta. Alcuni dei migliori velisti da regata hanno perni a rilascio rapido
sulle loro sartie in modo da poter riconfigurarle in acqua, se necessario.

Le vele devono avere una forma aerodinamica per poter risalire il vento. Mettere
un foglio piatto sull'albero non funziona. Nel corso degli anni, i velai hanno
ideato due tecniche di costruzione per creare e controllare la forma della vela.
La broadseam (letteralmente cucitura larga) forza la profondità e la posizione
del profilo nella vela. Come suggerisce il nome, la broadseam comporta la
sovrapposizione delle cuciture dei pannelli nella vela.
%
La seconda tecnica, la curva del\theluff, consente di modificare facilmente la
pienezza della vela mentre la barca è in movimento. La curvatura viene forzata
in una vela con \aluff curva che è posizionata su un albero relativamente
dritto. Consentendo all'albero di piegarsi, si fa sì che la vela diventi sempre
più piatta fino a quando la quantità di curvatura non eguaglia la quantità di
curva tagliata nel\theluff e la curvatura del\theluff non contribuisce più alla
pienezza della vela. Piegare l'albero consente anche al \leech di ridurre lo
svergolamento, depotenziando la randa. L'albero può essere piegato oltre la
curva del\theluff, causando l'inversione della vela e grandi pieghe diagonali
che si irradiano dall'angolo di scotta fino allo \spreader. L'inversione dovrebbe
essere evitata in tutte le condizioni tranne quelle di vento molto forte. In
condizioni estreme, una randa invertita, sebbene quasi completamente
depotenziata, consente comunque alla barca di tenere la rotta e alla randa di
non essere tirata troppo.

Le rande dei 470, così come la maggior parte delle altre derive, presentano sia
broadseam che curve del\theluff per creare vele che siano efficaci in
un'ampia gamma di condizioni di vento e mare.