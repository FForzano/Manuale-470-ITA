%!TEX root = ../main.tex

\section{La scuffia}
\label{sec:scuffia}
Se non scuffi main in 470, probabilmento stai facendo qualcosa di sbagliato!
Potrebbe essere che non stai spingendo la tua barca quanto potresti, o che non
esci in condizioni difficili come saresti in grado di affrontare, o che sei
eccessivamente cauto e non permetti alla barca di esprimere tutto il suo
potenziale. Uno dei piaceri di navigare su un 470 è la gestibilità della barca
dopo una scuffia. Molte barche sono impossibili da raddrizzare o tendono a
restare allagate dopo una scuffia, ma un 470 si rialza subito, quasi
completamente asciutto con solo un po' di pratica. La prima scuffia è sempre la
più difficile, e scuffiare con vento forte può essere piuttosto snervante,
indipendentemente da quanta pratica si abbia.

Il 470, ahimè, non è molto felice quando è sdraiato su un fianco. Preferirebbe
essere dritto o completamente capovolto. È quindi molto importante muoversi
rapidamente dopo ogni scuffia, se si vuole evitare il capovolgimento completo.
Il modo più sicuro per far capovolgere completamente un 470 dopo una scuffia è
sedersi sul bordo. Poiché il prodiere potrebbe aver bisogno di un po' di
tempo per sganciarsi, in molti casi sarà lui il colpevole. Pertanto, per ridurre
il tempo e lo sforzo necessari per rimettersi a navigare, è fondamentale far
scendere subito il prodiere dal bordo barca. Questo di solito implica tuffarsi
in acqua dal lato sottovento. Prima di prendere la decisione finale di tuffarsi,
il prodiere dovrebbe dare un'occhiata veloce al timoniere, se possibile, per
capire cosa ha intezione di fare.

Ci sono situazioni in cui conviene che il timoniere si tuffi in acqua e il
prodiere vada direttamente verso il centro della deriva. Questo accade quando il
timoniere è seduto \leeward o sbilanciato mentre la barca si inclina. Tuttavia,
nella maggior parte dei casi, il prodiere dovrebbe tuffarsi in acqua mentre lo
skipper si dirige verso la deriva il più velocemente possibile.

\emph{Avvertimento:} i 470 hanno svuotatori (bailers) con bordi estremamente
affilati che tendono a graffiare le gambe. Fai attenzione quando sali sulla
deriva e chiudi subito lo svuotatore con il palmo della mano prima di fare
qualsiasi altra cosa. Una volta che qualcuno è sulla deriva, la barca si
stabilizzerà e non si inclinerà ulteriormente. Dopo essere caduto in acqua,
il prodiere dovrebbe sganciarsi, districarsi dalle cime galleggianti intorno e
uscire dalla vela.

È buona norma urlare per sapere dove si trova l'altro. La comunicazione
diventa molto importante quando non riesci a vedere il tuo compagno attraverso
la barca.

A questo punto, la deriva dovrebbe essere estesa completamente verso il basso
per avere la massima leva. Se la tua deriva ha un sistema a frizione, è facile
estrarla dal fondo della barca. Sia il timoniere che il prodiere devono
ricordarsi di liberare la \sheet della randa, del fiocco, dello spinnaker e il
\vang, poiché tutti questi tendono a rendere la barca ingovernabile mentre viene
raddrizzata. La persona sulla deriva può dunque indietreggiare fino al bordo e
iniziare a tirare su la barca. Tenere la maniglia del trapezio ti dà maggiore
stabilità e leva mentre indietreggi. Quando la barca si solleva, l'equipaggio in
acqua dovrebbe agganciare il braccio intorno alla barra del carrello o sotto una
cinghia. Una volta che la barca comincia a sollevarsi, prende velocità e
potrebbe ribaltarsi completamente a meno che non venga fermata. Agganciandosi al
carrello, puoi mettere il tuo peso sul serbatoio (seduta laterale) non appena la
barca raggiunge la verticale e fermarne il rollio (anche tenersi alla maniglia
del trapezio funziona bene). Spesso la barca comincerà a navigare prima che
entrambi siate tornati a bordo, quindi sarà importante che la persona più vicina
alla poppa sia pronta a prendere il timone prima di risalire dall'acqua.

Non importa da che parte si trovi la barca quando viene raddrizzata. Appena la
testa della vela esce dall'acqua, il vento farà ruotare la barca e la farà
sollevare immediatamente. La persona in acqua deve essere molto veloce, poiché
sarà necessario tirare con forza per evitare che la barca si ribalti nell'altra
direzione.

Se la barca si capovolge completamente, sia il timoniere che il prodiere
potrebbero dover salire sul fondo della barca e tirare la deriva. Se la deriva è
scivolata all'interno della fessura, sarà necessario che uno di voi nuoti sotto
la barca e la spinga fuori. Una volta che la deriva è completamente estesa,
entrambi i velisti dovrebbero tirare delicatamente la barca inclinando il peso
sulla deriva. Saltare troppo forte può danneggiare sia la deriva che lo scafo.
Quando la punta dell'albero è appena in superficie, uno di voi deve entrare in
acqua e nuotare verso il lato in basso. La persona sulla deriva dovrebbe
stabilizzare la barca finché l'altro non è in posizione sul lato basso. A questo
punto, è esattamente come una scuffia normale.

Molte, se non la maggior parte, delle scuffie avvengono navigando in poppa,
quando è probabile che lo spinnaker sia issato. Quando ciò accade, la vela tende
a impigliarsi in modo intricato tra le crocette, le sartie o qualsiasi altra
cosa a portata di mano. Cerca di liberare completamente la vela prima di
raddrizzare la barca. Potrebbe essere necessario staccare la vela dalla sua
drizza e/o dalle scotte per districare il groviglio.

\emph{Nota:} Fissa sempre la drizza a qualcosa o almeno fai un nodo che prevenga
che essa rientri nell'albero.

Un modo sicuro per strappare uno spinnaker è strattonarlo quando è impigliato
dopo aver raddrizzato la barca. Se lo tiri su e non si libera completamente,
cerca di limitare i movimenti bruschi il più possibile mentre cerchi
delicatamente di liberare la vela.

Se scuffi in acque poco profonde, è fondamentale sollevare la punta dell'albero
dal fango lentamente per evitare di piegarlo. La chiave è evitare sforzi
eccessivi o salti sulla deriva. Sii paziente e il vento alla fine aiuterà la
barca a uscire dal fango, permettendo di raddrizzarla con poco sforzo. Fai
particolarmente attenzione quando scuffi vicino alla riva. Prima di rendertene
conto, potresti trovarti spinto in acque poco profonde.