% !TEX root = ../main.tex
\renewcommand{\subsectionsPath}{RegolazioneDellaBarca}

\section{Armare e condurre il 470}
\label{sec:RegolazioneDellaBarca}
Dopo pochi secondi di ispezione del 470, ti renderai subito di quanto complicato
e al contempo delicato esso sia. Per garantire una navigazione sicura e
piacevole è opportuno prendere le adeguate precauzioni sia a terra che una volta
in mare. Dedica un po' di tempo a esaminare la barca a terra. Controlla ogni cima per vedere cosa fa e che funzioni correttamente.
Assicurati che non ci siano frizioni in nessuno dei dei circuiti della
barca. Se vedi un problema o un potenziale problema, risolvilo \textbf{prima} di andare a
navigare. Per una barca completamente attrezzata avrai bisogno di:
\begin{itemize}
    \item Randa e fiocco
    \item Tre stecche (1 lunga, 2 corte)
    \item Timone e barra
    \item Imbracatura per il trapezio
    \item Spinnaker e tangone
    \item Due giubbotti di salvataggio
\end{itemize}
Procedi d'apprima ad armare la randa inferendone la base nel boma e attaccando
dunque quest'ultimo all'albero tramite il \emph{corno di trozza} (perno presente
dull'albero). Inserisci a seguire, le stecche nella vela. Le due inferiori (corte)
sono abbastanza "normali" nel loro funzionamento. La stecca superiore lunga va
dall'inferitura della randa alla balumina. La sua tensione può essere regolata
per modificare la forma nelle sezioni superiori
della vela. In generale, con vento leggero e forte, la stecca dovrebbe essere
allentata (ma non così allentata da cadere). Con vento moderato, la stecca
dovrebbe essere abbastanza tesa. La tensione corretta tuttavia, dipende
dalla sua flessibilità, dallo stile di navigazione, dalle onde, dalle
condizioni della vela, dal peso dell'equipaggio e da una serie di altri fattori.
Per regolarla correttamente, devi guardare la tua vela e sperimentare per vedere
cosa ti sembra meglio. Come con tutti gli aggiustamenti della vela, se
hai domande, chiedi a un esperto locale.

Il fiocco 470 contiene al suo interno un cavo d'acciaio che costituisce lo
strallo dell'imbarcazione. Tale cavo non è inserito di base nella vela ma deve
essere inserito e rimosso prima e dopo ogni utilizzo. Quando il fiocco è issato,
l'albero è sostenuto da tale strallo e non cavo d'acciaio più sottile e
permanente (amichevolmente \emph{stralletto}) che si trova su altre
imbarcazioni. Quest'ultimo ha la sola funzione di evitare che l'albero cada
quando la barca non è armata. Il fiocco è dunque realizzato senza i comuni ganci
per lo strallo. Dopo aver attaccato il fiocco ad un grillo posizionato
a prua (solitamente più a poppa dello stralletto), aver attaccato la drizza
all'angolo di penna e le scotte alle balumine, hai la
scelta di tre opzioni per le estremità libere delle scotte del fiocco. La prima
opzione consiste nell'effettuare un nodo Savoia all'estremo. In tal caso, lascia
unaun margine di 6-8 pollici in modo da poterle afferrare se si
tirano sino allo strozzatore. In alternativa, le due estremità delle scotte del fiocco
possono essere legate insieme. Questo sistema continuo riduce l'incertezza del
prodiere poiché esiste solo una scotta da afferrare durante la virata. Una terza
possibilità è quella di far passare la scotta attraverso il pozzetto e legarla
alla maniglia del trapezio. Anche se questo sistema aumenta le probabilità
di inciampare nella barca, dopo poche sessioni di pratica diventa
sorprendentemente veloce ed efficiente. Sorprenderai i tuoi amici quando
vedranno quanto velocemente riesci a trovare la scotta giusta e a cazzare la
vela.

Sia la randa che il fiocco sono sostenute da drizze metalliche con terminazioni
tessili. L'inestensibilità di esse, fa sì che la posizione verticale delle vele
non vengano influenzate dalla
regolazione delle scotte o da venti forti. Le code tessili sono gli unici
elementi che ti permetteranno di non rovinarti le mani nell'issare le vele.
Questo sistema permette di non legare le vele, ma il cavo d'acciaio viene
agganciato ad un blocco di drizza (un elemento che permette di bloccare la
drizza) o alla ghinda. Dopo aver issato le vele e prima di uscire in acqua, scoprirai che è
una buona idea riporre le code delle drizze in una tasca di una delle tasche
dello spi. Le procedure di armatura e conservazione dello spi sono più facili a
vedersi che a spiegarsi (anche se essa verrà descritta dettagliatamente in una sezione successiva).

Alcuni accorgimenti da ricordare sono di controllore la drizza per assicurarti che sia
\emph{in chiaro} (non ingarbugliata) sin in cima all'albero. Controlla che sia completamente sciolta dalle sartie, non sia
bloccata tra uno spigolo e la randa e che lo spi sia riposto
all'esterno del fiocco e delle sartie. Probabilmente scoprirai che lo spi si
arma e ripone più facilmente se è montato prima del fiocco.

Prima di mettere la barca in acqua, attacca il tangone per verificare che il
\emph{carica alto} funzioni e sia a un'altezza approssimativamente corretta.
Dopo aver regolato l'altezza, riponi il tangone in modo sicuro nella barca in
modo che non sganciarsi e perdersi in caso di scuffia.

Il trapezio è un accessorio obbligatorio in tutte le condizioni di vento tranne
che in quelle più leggere. Fatta eccezione che per fini didattici, generalmente è
permesso solo un trapezio per barca. Esistono diverse scuole di pensiero sul suo
utilizzo. La maggior parte dei velisti concorda sul fatto che la cinghia in vita
debba essere abbastanza stretta. Il giubbotto di salvataggio può essere
indossato sia sotto le cinghie delle spalle, dove funge da cuscino e migliora la
vestibilità per equipaggi più piccoli, o sopra il trapezio per evitare che le
cinghie si incastrino sotto il boma. Alcuni velisti regoleranno le cinghie delle
spalle in base alla distribuzione del peso per ricercare il corretto supporto.

Se sei un principiante, mantieni la cinghia stretta per una sensazione di
sicurezza e tranquillità. Nel momento in cui diventerai più a tuo agio, prova ad
allentare la cinghia delle spalle con venti forti in modo da poterti allungare
di più.

Per il resto, l'abbigliamento del prodiere dovrebbe includere pantaloni lunghi e
una maglietta con colletto per evitare tagli e sfregamenti dalle cinghie del
trapezio. Le scarpe sono obbligatorie se intendi mantenere lo stesso paio di piedi
per il resto della tua vita. E, naturalmente, vestiti per il clima, che, in un
470 quando il vento è forte, significa vestiti per un continuo tuffo in acqua
che potrebbe essere piuttosto fredda.

La natura delicata del 470 lo rende particolarmente soggetto a danni quando
ormeggiato al molo.
Per minimizzare l'abuso che queste povere barche devono sopportare, lasca
completamente il
boomvavangng e alza la deriva ogni volta che la barca è ormeggiata. Ammaina le vele
e rimuovi il timone e la barra se la barca rimarrà al molo per più di qualche
minuto. Il dondolio e il rollio della barca potrebbero sbattere il timone contro
la barca e romperlo. Infine, non lasciare mai la tua barca
incustodita al molo, anche con le vele ammainate.

% !TEX root = ../../main.tex

\subsection{Conduzione di Bolina}
\label{subsec:ConduzioneDiBolina}
Il 470 deve essere portato perfettamente piatto sull'acqua quando si naviga di
bolina, eccetto in condizioni di vento molto leggero, quando la barca può essere
leggermente sbandata sottovento per mantenere le vele gonfie. La forma dello
scafo del 470 lo rende più veloce, meno soggetto a scarroccio e più
controllabile quando completamente piatta. Mantenendo la barca piatta, si riduce
la tendenza della barca a orzare e poggiare da sola, riducendo così l'uso del
timone che rallenterebbe la barca.

\subsubsection{Il prodiere}
\label{subsubsec:IlProdiere}
Il principale compito del prodiere del 470 è mantenere la corretta
inclinazione dello scafo. Con vento leggero, ciò richiede di spostarsi agilmente
da un lato all'altro della barca. Con vento più forte, diventa necessario l'uso
del trapezio. Il lavoro del prodiere ha il fine di consentire al timoniere di
rimanere seduto comodamente in un punto in cui possa vedere sia la forma del
fiocco che l'acqua sopra e sottovento.

L'inclusione del trapezio su una barca a vela permette al progettista di
aumentare l'area velica e diminuire la larghezza della barca, riducendone così
la resistenza. Ancora più importante però, è il grado di liberà aggiuntivo che questo
dispositivo dà all'equipaggio. Durante le prime uscite, specialmente con vento
leggero, il timoniere farà di tutto per tenere il prodiere fuori al trapezio.
Nel momento in cui acquisterete esperienza, il timoniere potrà prestare
sempre meno attenzione alla posizione del prodiere, dedicandosi agli altri suoi
compiti. Ricordate, è compito del prodiere è quello di mantenere la barca
piatta.

Esistono due modi per uscire sul trapezio, da dentro la barca; un modo un po'
più lento ma facile e un modo più veloce. Per uscire nel modo semplice,
agganciati all'anello del trapezio mentre sei seduto sul bordo della barca e
cazza l'\emph{ascensore del trapezio} finché non sei sospeso appena sopra il
bordo della barca. Afferra dunque la maniglia
con la mano più a prua e posiziona la gamba a prua, piegata, sulla seduta. Metti
la mano a poppa sul bordo della barca appena dietro di te. Quando sei pronto
a uscire, metti il tuo peso sul filo, inclinati all'indietro e spingi con la
mano posteriore. Stendi la gamba a prua e porta la gamba a poppa sul bordo
della barca. Dovresti essere ora fuori sul trapezio!

Il modo impegnativo e veloce per uscire è più facile da descrivere: afferra la
maniglia con la mano più a prua, salta fuori e agganciati. Aspetta di essere
fuori dalla barca prima di agganciarti per ottenere virate molto veloci e di
classe che possono farti demolire gli avversari in una regata.

Durante le prime uscite sul trapezio, troverai più comodo appoggiare il piede a
prua contro la sartia, avere l'ascensore del trapezio completamente cazzato e posizionare
i piedi distanziati per mantenere l'equilibrio.

Quando diventerai più esperto, cerca di migliorare il tuo posizionamento e, come
conseguenza, le prestazioni della barca. Avvicina i piedi per spostare il tuo peso il più
possibile verso l'esterno. Per conferire ancora più stabilità alla barca, resta
sulle punte dei
piedi e lasca l'ascensore per abbassare il tuo peso e aumentare la forza
contro-sbandante.
Posizioni più alte dell'ascensore sono utilizzate maggiormente con
vento leggero o onde formate.

Infine, allontanati dalla sartia. Avere il peso
avanzato spinge la prua verso il basso e diminuisce dastricamente le prestazioni della barca.
Quando navighi in acque calme, posizionati a circa 60-90 cm a poppa della
sartia. Così come le onde e il vento aumentano, spostati a poppa fino a che
non ti trovi appena davanti al timoniere, che dovrebbe essere seduto proprio sopra il
carrello della randa. Come prodiere, la tua posizione esatta dipende
dal tuo peso e dal peso del timoniere. Come regola generale, con vento leggero e
acque calme, guarda avanti dove la prua taglia l'acqua. La curva della barca,
dove la prua si allarga e diventa il fondo dello scafo, dovrebbe appena sfiorare
le onde. In condizioni
di mare mosso, la barca dovrebbe sembrare come se stesse saltellando attraverso
le onde. Qualsiasi siano le condizioni, muoviti avanti e indietro per vedere gli
effetti del tuo peso. Annotati mentalmente il comportamento della barca allo
spostarsi del tuo peso ed in particolare se la barca tende a
planare più facilmente, a scavalcare le onde, se sembra più lenta, se tende a "puntare" più in
alto o a spruzzare acqua in modo strano. Chiedi inoltre al timoniere se
riesce a percepire una differenza nel timone in base al tuo spostamento.

L'aspetto critico di un buon lavoro sul trapezio è la fluidità. Troppo spesso i
principianti, e non solo, saltano fuori dalla barca quando la prima raffica
arriva, facendo sbandare la barca a sottovento, per poi rientrare velocemente a
bordo una volta bagnati. Questo continuo oscillare da un lato all'altro mentre
si naviga di bolina è generalmente considerato una cattiva pratica e non risulta
molto divertente.

Il primo requisito per un buon lavoro sul trapezio è che tu debba tenere gli
occhi fuori dalla barca e guardare da dove proviene il vento. Se vedi una grossa
raffica arrivare, puoi saltare fuori dalla barca abbastanza velocemente. D'altro
canto, se vedi che stai per essere colpito da una piccola raffica, sii pronto a
lasciare la scotta più lentamente, rientrando dolcemente.

Se sei già sul trapezio e il vento inizia a calmarsi, non saltare subito in
barca. Prima, siediti mantenendo le gambe dritte e piegati solo in vita. Se il
vento cala ancora, resta sul bordo e piegati in modo da poter rimanere seduto
sul bordo. Quando la prossima raffica arriva, puoi tornare fuori senza
dover passare per il fastidio di entrare ed uscire dalla barca. Ricorda, dal
momento in cui c'è una brezza moderata, il 470 deve essere condotto assolutamente
piatto. Presta attenzione a quanto la barca sta sbandando. Uno sguardo allo
specchio di poppa può aiutare a capire quanto la barca sia piatta.

In condizioni di vento appena sufficiente ad usare il trapezio, il lavoro del
prodiere richiede molta concentrazione e
pazienza. Sii pronto a regolare costantemente il tuo peso per mantenere la barca
in equilibrio. Spesso è una buona idea alzare l'ascensore abbastanza in alto da
tenerti appena fuori dal lato della barca quando sei seduto. Questo ti permette
di uscire facilmente senza dover sollevare il tuo peso ogni volta. Come il vento
aumenta, siediti sempre più fuori bordo mentre sei agganciato al trapezio. Se le
tue gambe sono abbastanza lunghe, sospeso direttamente sopra la deriva.

Altrimenti, tieniti a metà strada spingendoti indietro e spingendoti fuori dal lato
della barca con la mano a poppa. Sii pronto a mettere il piede anteriore sul
bordo della barca quando la raffica aumenta. Se necessario, tieni la scotta del
fiocco vicino per un'emergenza, rilassati e goditi il viaggio.

In caso di raffiche, puoi spostare il tuo peso per mantenere la barca piatta. Se fatto
armonicamente con il timoniere, questo movimento può essere uno strumento
estremamente potente con vento forte. Infatti, oltre a spingere la barca verso il basso,
il movimento fa flettere la cima dell'albero, permettendo di \emph{pompare} la parte
alta della vela (Con pompare si intende far fare un movimento brusco alla vela
che crea una spinta in avanti). Se la barca sbanda eccessivamente
troppo, lasca il fiocco per un istante e per poi cazzarlo di nuovo. Non lasciare il
fiocco libero di sventolare; ciò potrebbe portare alla tua rovina. Un 470 deve
essere tenuto sempre in movimento con vento forte. La barca può scuffiare,
anche con entrambe le vele sventolanti, se è ferma.

\subsubsection{Il timoniere}
\label{subsubsec:IlTimoniere}
Timonare di bolina in un 470, o in qualsiasi altra barca da regata, è
un compito non banale. Richiede concentrazione, osservazione, sperimentazione e
molta pratica. Quando inizi a timonare la barca, passerai molto tempo a
preoccuparti di dove si trova il tuo prodiere e come tenerlo fuori sul trapezio.
Per questo motivo, è meglio provare a navigare con la stessa persona per un po'
di tempo finché non vi abituate entrambi alla barca. Ricorda, finché il prodiere
non è completamente fuori sul trapezio, è sua responsabilità mantenere la barca
piatta e il timoniere deve rimanere seduto in una posizione comoda. Quando
navighi correttamente di bolina, il 470 è in grado di tenere rotte piuttosto
strette (con un angolo rispetto al vento relativamente piccolo).
Trovare tale angolo limite non è scontato ed è necessario passare molto tempo in barca e concentrarsi
quando si naviga di bolina. Un metodo per trovare la rotta più stretta
navigabile è il seguente. Cazzate a ferro tutte le vele (più dettagli a
riguardo più avanti) e timona per mantenere i bandierini del fiocco che
sventolano dritti. I bandierini del fiocco forniscono una misura molto precisa
del suo angolo di scotta. Il bandierino interno (sopravento) sventola prima che
la vela stessa lo faccia indicando dunque che essa è troppo lasca (o che
l'andatura è troppo stretta nel caso in cui sia cazzata a ferro); il bandierino
esterno che sventola, visto in ombra dietro la vela,
indica che la vela è eccessivamente cazzata. Il fiocco è al massimo della sua efficienza quando
entrambi i bandierini fileggiano dritti, senza essere soggetti a turbolenze.
Cazza se il bandierino interno
sventola; lasca la scotta se lo fa quello esterno. Se stai facendo un buon
lavoro, la barca avrà un timone quasi neutro. Una leggera tendenza orziera è
accettabile. Ciò ti permetterà di timonare attraverso le onde con pochissimo
movimento del timone. Prova a timonare con gli occhi chiusi per un po' e presto
sarai in grado di percepire la sensazione della barca quando è prua a vento.

Il 470 è un'imbarcazione facile da tenere piatta, ma è piuttosto difficile da
riportare in posizione una volta che è sbandata. Quando arriva una raffica,
sii pronto a lavorare duramente per un po' per riportare la barca in posizione.
La tecnica di base per raddrizzare la barca non è particolarmente complicata: lasca
la randa leggermente e orza leggermente. Quando
la barca è piatta, cazzate la randa di nuovo, tornate alla rotta corretta e
potete rilassarvi fino alla prossima raffica. In quasi tutte le condizioni, il
timoniere dovrebbe essere seduto il più appruato possibile, vicino all'attacco
della scotta randa. Come
prodiere, ciò mette il tuo peso nella parte più larga della barca, permettendoti
di raggiungere tutte le cime di controllo e di gestire l'inclinazione della
barca al meglio per superare le onde e virare rapidamente. C'è una forte tendenza per
i principianti a "scivolare" a poppa ad ogni possibile occasione. Cerca di
rimanere in avanti. Ricorda, continua a lavorare sulle vele per adattarle alle
condizioni variabili.

\subsubsection{La virata}
\label{subsubsec:LaVirata}
