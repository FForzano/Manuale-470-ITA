% !TEX root = ../main.tex

\section{Armare e condurre il 470}
\label{sec:RegolazioneDellaBarca}
Dopo pochi secondi di ispezione del 470, ti renderai subito di quanto complicato
e al contempo delicato esso sia. Per garantire una navigazione sicura e
piacevole è opportuno prendere le adeguate precauzioni sia a terra che una volta
in mare. Dedica un po' di tempo a esaminare la barca a terra. Controlla ogni cima per vedere cosa fa e che funzioni correttamente.
Assicurati che non ci siano frizioni in nessuno dei dei circuiti della
barca. Se vedi un problema o un potenziale problema, risolvilo \textbf{prima} di andare a
navigare. Per una barca completamente attrezzata avrai bisogno di:
\begin{itemize}
    \item Randa e fiocco
    \item Tre stecche (1 lunga, 2 corte)
    \item Timone e barra
    \item Imbracatura per il trapezio
    \item Spinnaker e tangone
    \item Due giubbotti di salvataggio
\end{itemize}
Procedi d'apprima ad armare la randa inferendone la base nel boma e attaccando
dunque quest'ultimo all'albero tramite il \emph{corno di trozza} (perno presente
dull'albero). Inserisci a seguire, le stecche nella vela. Le due inferiori (corte)
sono abbastanza "normali" nel loro funzionamento. La stecca superiore lunga va
dall'inferitura della randa alla balumina. La sua tensione può essere regolata
per modificare la forma nelle sezioni superiori
della vela. In generale, con vento leggero e forte, la stecca dovrebbe essere
allentata (ma non così allentata da cadere). Con vento moderato, la stecca
dovrebbe essere abbastanza tesa. La tensione corretta tuttavia, dipende
dalla sua flessibilità, dallo stile di navigazione, dalle onde, dalle
condizioni della vela, dal peso dell'equipaggio e da una serie di altri fattori.
Per regolarla correttamente, devi guardare la tua vela e sperimentare per vedere
cosa ti sembra meglio. Come con tutti gli aggiustamenti della vela, se
hai domande, chiedi a un esperto locale.

Il fiocco 470 contiene al suo interno un cavo d'acciaio che costituisce lo
strallo dell'imbarcazione. Tale cavo non è inserito di base nella vela ma deve
essere inserito e rimosso prima e dopo ogni utilizzo. Quando il fiocco è issato,
l'albero è sostenuto da tale strallo e non cavo d'acciaio più sottile e
permanente (amichevolmente \emph{stralletto}) che si trova su altre
imbarcazioni. Quest'ultimo ha la sola funzione di evitare che l'albero cada
quando la barca non è armata. Il fiocco è dunque realizzato senza i comuni ganci
per lo strallo. Dopo aver attaccato il fiocco ad un grillo posizionato
a prua (solitamente più a poppa dello stralletto), aver attaccato la drizza
all'angolo di penna e le scotte alle balumine, hai la
scelta di tre opzioni per le estremità libere delle scotte del fiocco. La prima
opzione consiste nell'effettuare un nodo Savoia all'estremo. In tal caso, lascia
unaun margine di 6-8 pollici in modo da poterle afferrare se si
tirano sino allo strozzatore. In alternativa, le due estremità delle scotte del fiocco
possono essere legate insieme. Questo sistema continuo riduce l'incertezza del
prodiere poiché esiste solo una scotta da afferrare durante la virata. Una terza
possibilità è quella di far passare la scotta attraverso il pozzetto e legarla
alla maniglia del trapezio. Anche se questo sistema aumenta le probabilità
di inciampare nella barca, dopo poche sessioni di pratica diventa
sorprendentemente veloce ed efficiente. Sorprenderai i tuoi amici quando
vedranno quanto velocemente riesci a trovare la scotta giusta e a cazzare la
vela.

Sia la randa che il fiocco sono sostenute da drizze metalliche con terminazioni
tessili. L'inestensibilità di esse, fa sì che la posizione verticale delle vele
non vengano influenzate dalla
regolazione delle scotte o da venti forti. Le code tessili sono gli unici
elementi che ti permetteranno di non rovinarti le mani nell'issare le vele.
Questo sistema permette di non legare le vele, ma il cavo d'acciaio viene
agganciato ad un blocco di drizza (un elemento che permette di bloccare la
drizza) o alla ghinda. Dopo aver issato le vele e prima di uscire in acqua, scoprirai che è
una buona idea riporre le code delle drizze in una tasca di una delle tasche
dello spi. Le procedure di armatura e conservazione dello spi sono più facili a
vedersi che a spiegarsi (anche se essa verrà descritta dettagliatamente in una sezione successiva).

Alcuni accorgimenti da ricordare sono di controllore la drizza per assicurarti che sia
\emph{in chiaro} (non ingarbugliata) sin in cima all'albero. Controlla che sia completamente sciolta dalle sartie, non sia
bloccata tra uno spigolo e la randa e che lo spi sia riposto
all'esterno del fiocco e delle sartie. Probabilmente scoprirai che lo spi si
arma e ripone più facilmente se è montato prima del fiocco.

Prima di mettere la barca in acqua, attacca il tangone per verificare che il
\emph{carica alto} funzioni e sia a un'altezza approssimativamente corretta.
Dopo aver regolato l'altezza, riponi il tangone in modo sicuro nella barca in
modo che non sganciarsi e perdersi in caso di scuffia.

Il trapezio è un accessorio obbligatorio in tutte le condizioni di vento tranne
che in quelle più leggere. Fatta eccezione che per fini didattici, generalmente è
permesso solo un trapezio per barca. Esistono diverse scuole di pensiero sul suo
utilizzo. La maggior parte dei velisti concorda sul fatto che la cinghia in vita
debba essere abbastanza stretta. Il giubbotto di salvataggio può essere
indossato sia sotto le cinghie delle spalle, dove funge da cuscino e migliora la
vestibilità per equipaggi più piccoli, o sopra il trapezio per evitare che le
cinghie si incastrino sotto il boma. Alcuni velisti regoleranno le cinghie delle
spalle in base alla distribuzione del peso per ricercare il corretto supporto.

Se sei un principiante, mantieni la cinghia stretta per una sensazione di
sicurezza e tranquillità. Nel momento in cui diventerai più a tuo agio, prova ad
allentare la cinghia delle spalle con venti forti in modo da poterti allungare
di più.

Per il resto, l'abbigliamento del prodiere dovrebbe includere pantaloni lunghi e
una maglietta con colletto per evitare tagli e sfregamenti dalle cinghie del
trapezio. Le scarpe sono obbligatorie se intendi mantenere lo stesso paio di piedi
per il resto della tua vita. E, naturalmente, vestiti per il clima, che, in un
470 quando il vento è forte, significa vestiti per un continuo tuffo in acqua
che potrebbe essere piuttosto fredda.

La natura delicata del 470 lo rende particolarmente soggetto a danni quando
ormeggiato al molo.
Per minimizzare l'abuso che queste povere barche devono sopportare, lasca
completamente il
boomvavangng e alza la deriva ogni volta che la barca è ormeggiata. Ammaina le vele
e rimuovi il timone e la barra se la barca rimarrà al molo per più di qualche
minuto. Il dondolio e il rollio della barca potrebbero sbattere il timone contro
la barca e romperlo. Infine, non lasciare mai la tua barca
incustodita al molo, anche con le vele ammainate.