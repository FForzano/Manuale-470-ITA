% !TEX root = ../main.tex

\section{La Barca}
\label{sec:LaBarca}
Il 470 è una deriva in vetroresina per due persone, progettata alla fine degli
anni '60 da André Corneau. Poco dopo la sua introduzione, il 470 fu accettato
come classe olimpica, soppiantando le derive Fireball. Il
debutto olimpico di tale deriva si ebbe alle Olimpiadi del 1976 a Kingston,
Ontario.

Il 470 è un monoscafo one-design con regolamenti di stazza piuttosto severi.
Nonostante molte aziende producano barche, vele e alberi, la forma, il peso e i
materiali dell'attrezzatura da regata del 470 sono attentamente regolamentati.
Uno dei migliori 470 al mondo fu prodotto dalla Vanguard Boat Works di Pewaukee,
Wisconsin. Attualmente esistono oltre 1700 esemplari di 470 solo negli Stati Uniti e circa 15.000
in tutto il mondo. Negli ultimi anni, sono state sviluppate diverse barche
simili per forma e tecnica richiesta al 470. Queste includono il 505, il 420, il
Laser II e il Flying Dutchman. Nonostante ciascuna di queste barche ha la
propria personalità e le proprie caratteristiche, esse condividono tutte la caratteristica comune di
essere derive leggere, con trapezio.

Il 470 è caratterizzato da un peso estremamente ridotto (120 kg completamente armato) e da una superficie velica
relativamente piccola; qui risiede la più grande differenza tra il 470 e la
maggior parte delle altre derive ad alte prestazioni che potreste incontrare. Il
470 è abbastanza facile da imparare portare, ma richiede una notevole
abilità per essere condotto al meglio. Una buona tecnica richiede di spingere la
barca al limite, pompare le vele e spostare il peso del corpo per facilitare la manovra.
La potenza limitata disponibile dal piano velico sottodimensionato deve essere
massimizzata mediante un attenta regolazione della curvatura dell'albero,
delle scotte e di qualsiasi altra cima di regolazione, in tutte le condizioni di
vento. Tuttavia, queste caratteristiche e l'utilizzo del trapezio rendono il 470
una barca che anche l'equipaggio più leggero e giovane può gestire anche in
presenza di vento forte. Queste ragioni portano il 470 ad essere la deriva migliore per i venti
forti (oltre i 25 nodi).

Il 470 è stato progettato come barca da regata per due persone e, a causa della
sua complessità, a timoniere e prodiere è richiesta una notevole
abilità. Inoltre, per navigare davvero bene, è richiesto un lavoro di squadra
impeccabile.

Assicuratevi di imparare a portare la barca sia come prodieri che
come timonieri. Se iniziate a regatare, cercate di navigare con lo stesso partner per un
po' (potreste scoprire che cambiare posto di tanto in tanto è molto utile). Dopo
solo poche regate, potreste essere in grado di battere atleti in generale più
bravi di voi, semplicemente perché averete affrontato meglio la regata e fatto
scelte tattiche migliori. Nei capitoli successivi verranno descritti più
esplicitamente i compiti di timoniere e prodiere, sia
in termini di tecnica di navigazione che di tattica. Ricordate però, avere un
prodiere competente vuol dire avere un cervello, due occhi e il doppio delle idee a bordo.
