% !TEX root = ../main.tex

\section{Introduzione}
\label{sec:Introduzione}
Non imparai mai a navigare leggendo un libro. Pertanto, se guardando fuori c'è
bel tempo, lascia perdere quello che stai facendo e vai a navigare. Il tempo che
trascorri su qualsiasi barca probabilmente ti aiuterà molto di più a navigare in
470 rispetto a quanto possa fare un qualsiasi libro di
navigazione. D'altra parte, se hai fatto o stai per fare il tuo primo giro in
470, e hai una serata libera o se fuori dalla finestra il clima è piovoso, potresti trovare utili
ed illuminanti alcuni dei consigli contenuti in questo manuale.
Per diversi anni, nonostante un forte interesse e una comunità molto attiva, non
è stato pubblicato alcun manuale per il 470. Diverse ragioni hanno portato a
questa mancanza di lungimiranza. In primo luogo, scrivere un manuale richiede
del tempo e non offre molti ricavi. Inoltre, ci sono solo pochi aspetti tecnici
del 470 che sono distintivi e unici della barca. La maggior parte delle
informazioni contenute in questo manuale possono essere reperite in qualsiasi
libro di navigazione di livello intermedio. Tuttavia, la barca ha le sue
peculiarità e alcune idee che altrove vengono accennate devono essere
enfatizzate per navigare in 470. Alcune sezioni diventano piuttosto tecniche e
se non sai cosa sia la balumina della randa o una turbolenza, potresti rimanere molto
confuso. Questo manuale contiene pochi diagrammi riguardanti circuiti e messa a
punto della bara. Spesso ci sono diversi modi per raggiungere lo stesso
obiettivo nel montare una barca. L'evoluzione delle tecniche di navigazione e
dei relativi circuiti è talmente rapida che
qualsiasi schema di messa a punto potrebbe essere obsoleto prima ancora che
venga pubblicato. Verrà invece presentato uno schema generale delle cime che
controllano la forma e la regolazione della randa, del fiocco e dello spinnaker.
Se sei confuso su quali cime fanno cosa sulla barca che ti stai preparando a
navigare, passa qualche minuto a terra tirando le cime e osservando cosa
succede. In questo manuale inoltre, verranno presentati pochi trucchi su
come navigare in 470 (ad esempio, lascare il tesabase di circa 5cm con venti da 5
a 8 nodi). Più nel dettaglio invece, verrà presentata la teoria su come le tue vele dovrebbero
lavorare e cosa fanno tutti i comandi sulla forma delle vele. Sta a te capire
come tutto questo si applichi durante la navigazione in acqua.

Infine, il 470 è una barca da regata, molte delle informazioni contenute in
questo manuale riguardano l'ottenimento della massima velocità e prestazioni.
Come verrà evidenziato, il 470 è fondamentalmente una barca
piuttosto facile. Il trucco è farla navigare bene, che è ciò che la
regata richiede. Tuttavia, le tecniche descritte non finalizzate alla sola regata, poiché una
barca 470 ben condotta è notevolmente più divertente.
%
Giunto alla fine, dovresti essere un navigatore abbastanza competente per capire che la perfezione
di forma e funzionalità è il principale obiettivo di navigazione. È importante
ricordare che il 470 è una barca delicata e costosa. La manutenzione di essa
dipende da te. Ci sono troppe cose di cui il personale addetto, se presente, deve
tenere in ordine e che vengono prima delle riparazioni minori sul tuo 470. Eppure,
sono proprio i piccoli incidenti che peggiorano notevolmente la navigazione e
segnano la fine di una barca. Se prevedi di navigare in 470 per anni a venire,
metti in conto di dedicare del tempo per la riparazione e la manutenzione
preventiva della tua deriva. Se desideri vedere cambiamenti e miglioramenti
nell'attrezzatura,
nelle regate o nelle strutture, spetta a te fare in modo che qualcosa venga fatto.