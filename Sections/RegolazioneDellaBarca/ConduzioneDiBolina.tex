% !TEX root = ../../main.tex

\subsection{Conduzione di Bolina}
\label{subsec:ConduzioneDiBolina}
Il 470 deve essere portato perfettamente piatto sull'acqua quando si naviga di
bolina, eccetto in condizioni di vento molto leggero, quando la barca può essere
leggermente sbandata sottovento per mantenere le vele gonfie. La forma dello
scafo del 470 lo rende più veloce, meno soggetto a scarroccio e più
controllabile quando completamente piatta. Mantenendo la barca piatta, si riduce
la tendenza della barca a orzare e poggiare da sola, riducendo così l'uso del
timone che rallenterebbe la barca.

\subsubsection{Il prodiere}
\label{subsubsec:IlProdiere}
Il principale compito del prodiere del 470 è mantenere la corretta
inclinazione dello scafo. Con vento leggero, ciò richiede di spostarsi agilmente
da un lato all'altro della barca. Con vento più forte, diventa necessario l'uso
del trapezio. Il lavoro del prodiere ha il fine di consentire al timoniere di
rimanere seduto comodamente in un punto in cui possa vedere sia la forma del
fiocco che l'acqua sopra e sottovento.

L'inclusione del trapezio su una barca a vela permette al progettista di
aumentare l'area velica e diminuire la larghezza della barca, riducendone così
la resistenza. Ancora più importante però, è il grado di liberà aggiuntivo che questo
dispositivo dà all'equipaggio. Durante le prime uscite, specialmente con vento
leggero, il timoniere farà di tutto per tenere il prodiere fuori al trapezio.
Nel momento in cui acquisterete esperienza, il timoniere potrà prestare
sempre meno attenzione alla posizione del prodiere, dedicandosi agli altri suoi
compiti. Ricordate, è compito del prodiere è quello di mantenere la barca
piatta.

Esistono due modi per uscire sul trapezio, da dentro la barca; un modo un po'
più lento ma facile e un modo più veloce. Per uscire nel modo semplice,
agganciati all'anello del trapezio mentre sei seduto sul bordo della barca e
cazza l'\emph{ascensore del trapezio} finché non sei sospeso appena sopra il
bordo della barca. Afferra dunque la maniglia
con la mano più a prua e posiziona la gamba a prua, piegata, sulla seduta. Metti
la mano a poppa sul bordo della barca appena dietro di te. Quando sei pronto
a uscire, metti il tuo peso sul filo, inclinati all'indietro e spingi con la
mano posteriore. Stendi la gamba a prua e porta la gamba a poppa sul bordo
della barca. Dovresti essere ora fuori sul trapezio!

Il modo impegnativo e veloce per uscire è più facile da descrivere: afferra la
maniglia con la mano più a prua, salta fuori e agganciati. Aspetta di essere
fuori dalla barca prima di agganciarti per ottenere virate molto veloci e di
classe che possono farti demolire gli avversari in una regata.

Durante le prime uscite sul trapezio, troverai più comodo appoggiare il piede a
prua contro la sartia, avere l'ascensore del trapezio completamente cazzato e posizionare
i piedi distanziati per mantenere l'equilibrio.

Quando diventerai più esperto, cerca di migliorare il tuo posizionamento e, come
conseguenza, le prestazioni della barca. Avvicina i piedi per spostare il tuo peso il più
possibile verso l'esterno. Per conferire ancora più stabilità alla barca, resta
sulle punte dei
piedi e lasca l'ascensore per abbassare il tuo peso e aumentare la forza
contro-sbandante.
Posizioni più alte dell'ascensore sono utilizzate maggiormente con
vento leggero o onde formate.

Infine, allontanati dalla sartia. Avere il peso
avanzato spinge la prua verso il basso e diminuisce dastricamente le prestazioni della barca.
Quando navighi in acque calme, posizionati a circa 60-90 cm a poppa della
sartia. Così come le onde e il vento aumentano, spostati a poppa fino a che
non ti trovi appena davanti al timoniere, che dovrebbe essere seduto proprio sopra il
carrello della randa. Come prodiere, la tua posizione esatta dipende
dal tuo peso e dal peso del timoniere. Come regola generale, con vento leggero e
acque calme, guarda avanti dove la prua taglia l'acqua. La curva della barca,
dove la prua si allarga e diventa il fondo dello scafo, dovrebbe appena sfiorare
le onde. In condizioni
di mare mosso, la barca dovrebbe sembrare come se stesse saltellando attraverso
le onde. Qualsiasi siano le condizioni, muoviti avanti e indietro per vedere gli
effetti del tuo peso. Annotati mentalmente il comportamento della barca allo
spostarsi del tuo peso ed in particolare se la barca tende a
planare più facilmente, a scavalcare le onde, se sembra più lenta, se tende a "puntare" più in
alto o a spruzzare acqua in modo strano. Chiedi inoltre al timoniere se
riesce a percepire una differenza nel timone in base al tuo spostamento.

L'aspetto critico di un buon lavoro sul trapezio è la fluidità. Troppo spesso i
principianti, e non solo, saltano fuori dalla barca quando la prima raffica
arriva, facendo sbandare la barca a sottovento, per poi rientrare velocemente a
bordo una volta bagnati. Questo continuo oscillare da un lato all'altro mentre
si naviga di bolina è generalmente considerato una cattiva pratica e non risulta
molto divertente.

Il primo requisito per un buon lavoro sul trapezio è che tu debba tenere gli
occhi fuori dalla barca e guardare da dove proviene il vento. Se vedi una grossa
raffica arrivare, puoi saltare fuori dalla barca abbastanza velocemente. D'altro
canto, se vedi che stai per essere colpito da una piccola raffica, sii pronto a
lasciare la scotta più lentamente, rientrando dolcemente.

Se sei già sul trapezio e il vento inizia a calmarsi, non saltare subito in
barca. Prima, siediti mantenendo le gambe dritte e piegati solo in vita. Se il
vento cala ancora, resta sul bordo e piegati in modo da poter rimanere seduto
sul bordo. Quando la prossima raffica arriva, puoi tornare fuori senza
dover passare per il fastidio di entrare ed uscire dalla barca. Ricorda, dal
momento in cui c'è una brezza moderata, il 470 deve essere condotto assolutamente
piatto. Presta attenzione a quanto la barca sta sbandando. Uno sguardo allo
specchio di poppa può aiutare a capire quanto la barca sia piatta.

In condizioni di vento appena sufficiente ad usare il trapezio, il lavoro del
prodiere richiede molta concentrazione e
pazienza. Sii pronto a regolare costantemente il tuo peso per mantenere la barca
in equilibrio. Spesso è una buona idea alzare l'ascensore abbastanza in alto da
tenerti appena fuori dal lato della barca quando sei seduto. Questo ti permette
di uscire facilmente senza dover sollevare il tuo peso ogni volta. Come il vento
aumenta, siediti sempre più fuori bordo mentre sei agganciato al trapezio. Se le
tue gambe sono abbastanza lunghe, sospeso direttamente sopra la deriva.

Altrimenti, tieniti a metà strada spingendoti indietro e spingendoti fuori dal lato
della barca con la mano a poppa. Sii pronto a mettere il piede anteriore sul
bordo della barca quando la raffica aumenta. Se necessario, tieni la scotta del
fiocco vicino per un'emergenza, rilassati e goditi il viaggio.

In caso di raffiche, puoi spostare il tuo peso per mantenere la barca piatta. Se fatto
armonicamente con il timoniere, questo movimento può essere uno strumento
estremamente potente con vento forte. Infatti, oltre a spingere la barca verso il basso,
il movimento fa flettere la cima dell'albero, permettendo di \emph{pompare} la parte
alta della vela (Con pompare si intende far fare un movimento brusco alla vela
che crea una spinta in avanti). Se la barca sbanda eccessivamente
troppo, lasca il fiocco per un istante e per poi cazzarlo di nuovo. Non lasciare il
fiocco libero di sventolare; ciò potrebbe portare alla tua rovina. Un 470 deve
essere tenuto sempre in movimento con vento forte. La barca può scuffiare,
anche con entrambe le vele sventolanti, se è ferma.

\subsubsection{Il timoniere}
\label{subsubsec:IlTimoniere}
Timonare di bolina in un 470, o in qualsiasi altra barca da regata, è
un compito non banale. Richiede concentrazione, osservazione, sperimentazione e
molta pratica. Quando inizi a timonare la barca, passerai molto tempo a
preoccuparti di dove si trova il tuo prodiere e come tenerlo fuori sul trapezio.
Per questo motivo, è meglio provare a navigare con la stessa persona per un po'
di tempo finché non vi abituate entrambi alla barca. Ricorda, finché il prodiere
non è completamente fuori sul trapezio, è sua responsabilità mantenere la barca
piatta e il timoniere deve rimanere seduto in una posizione comoda. Quando
navighi correttamente di bolina, il 470 è in grado di tenere rotte piuttosto
strette (con un angolo rispetto al vento relativamente piccolo).
Trovare tale angolo limite non è scontato ed è necessario passare molto tempo in barca e concentrarsi
quando si naviga di bolina. Un metodo per trovare la rotta più stretta
navigabile è il seguente. Cazzate a ferro tutte le vele (più dettagli a
riguardo più avanti) e timona per mantenere i bandierini del fiocco che
sventolano dritti. I bandierini del fiocco forniscono una misura molto precisa
del suo angolo di scotta. Il bandierino interno (sopravento) sventola prima che
la vela stessa lo faccia indicando dunque che essa è troppo lasca (o che
l'andatura è troppo stretta nel caso in cui sia cazzata a ferro); il bandierino
esterno che sventola, visto in ombra dietro la vela,
indica che la vela è eccessivamente cazzata. Il fiocco è al massimo della sua efficienza quando
entrambi i bandierini fileggiano dritti, senza essere soggetti a turbolenze.
Cazza se il bandierino interno
sventola; lasca la scotta se lo fa quello esterno. Se stai facendo un buon
lavoro, la barca avrà un timone quasi neutro. Una leggera tendenza orziera è
accettabile. Ciò ti permetterà di timonare attraverso le onde con pochissimo
movimento del timone. Prova a timonare con gli occhi chiusi per un po' e presto
sarai in grado di percepire la sensazione della barca quando è prua a vento.

Il 470 è un'imbarcazione facile da tenere piatta, ma è piuttosto difficile da
riportare in posizione una volta che è sbandata. Quando arriva una raffica,
sii pronto a lavorare duramente per un po' per riportare la barca in posizione.
La tecnica di base per raddrizzare la barca non è particolarmente complicata: lasca
la randa leggermente e orza leggermente. Quando
la barca è piatta, cazzate la randa di nuovo, tornate alla rotta corretta e
potete rilassarvi fino alla prossima raffica. In quasi tutte le condizioni, il
timoniere dovrebbe essere seduto il più appruato possibile, vicino all'attacco
della scotta randa. Come
prodiere, ciò mette il tuo peso nella parte più larga della barca, permettendoti
di raggiungere tutte le cime di controllo e di gestire l'inclinazione della
barca al meglio per superare le onde e virare rapidamente. C'è una forte tendenza per
i principianti a "scivolare" a poppa ad ogni possibile occasione. Cerca di
rimanere in avanti. Ricorda, continua a lavorare sulle vele per adattarle alle
condizioni variabili.

\subsubsection{La virata}
\label{subsubsec:LaVirata}
Ci sono tre elementi fattori da dovere considerare quando si vuole virare in 470: il
timone, le vele e lo scafo. Ovviamente puoi tirare o spingere il timone per far
puntare la barca nella giusta direzione. Non così ovvio, ma altrettanto
importante per direzionare la barca è la regolazione dello scafo e delle vele. Su
tutte le imbarcazioni a vela, tutte le forze del vento e dell'acqua, possono
essere considerate applicate in singoli punti sulle vele (Center of Effort-CE) e sullo scafo
(Center of Lateral Resistance-CLR) rispettivamente.

Quando una barca ha timone neutro, il CE è direttamente sopra il CLR e la barca naviga in linea
retta senza pressione sul timone. I velisti di 470 e di altre imbarcazioni ad
alte prestazioni cercano di regolare le loro imbarcazioni per raggiungere questa
situazione. La riduzione del movimento del timone non solo rende la barca più
reattiva, ma la rende effettivamente più veloce grazie alla riduzione della sua
resistenza. Anche se la forma della vela, l'inclinazione dell'albero,
la forma della deriva ecc., influenzano la posizione del CE e del CLR, i
cambiamenti più significativi che possono essere effettuati durante la navigazione riguardano la
regolazione delle vele e lo sbandamento della barca. Oltre a ridurre la quantità
di movimento del timone necessaria, la regolazione della barca e delle vele può
aiutarti a mantenere la barca sotto controllo in condizioni di vento forte, o
durante grandi cambi di rotta come le virate in boa.

Il 470 ha un design del timone notoriamente inefficiente a causa delle
stringenti regole di classe. Non è raro che il flusso d'acqua si separi dalla
lama del timone. Questo fenomeno è noto come stallo. Quando ciò accade, ad
esempio, qunado il timone è girato troppo, esso diventa quasi completamente
inefficace.
In condizioni di vento forte, il timone può creare turbolenze che contribuiscono
a portare la barca alla straorza e a renderla non controllabile. Con vento
leggero, la barca sembrerà lenta e senza controllo mentre deriva verso l'angolo
morto.

Lascando leggermente la randa, il
CE si sposterà in avanti e la barca poggerà. Questo accade poiché il fiocco
fornirà più potenza e forza di rotazione. Al contrario, cazzare leggermente la
randa sposta il CE in avanti e porterà la barca a orzare.

Sbandare la barca cambia la posizione del CLR. Quando la barca sbanda
sottovento, come accade durante una raffica, il CLR si sposta sottovento
portando la barca a orzare. Pertanto, quando una raffica colpisce e la barca
si inclina, è imperativo che tu laschi leggermente la randa per neutralizzare il
timone e far tornare la barca sotto controllo. Semplicemente girare il timone
non è spesso sufficiente.

Quando diventa necessario fare cambiamenti importanti di
direzione, come durante una virata, un'abbattuta o in partenza,
le vele e l'inclinazione possono essere utilizzate per rendere il lavoro del
timone molto più facile. In generale, quando si intende poggiare, lasca la
randa e inclina la barca sopravento. Quando si vuole orzare, permetti
alla barca di sbandare e cazza rapidamente la randa. Queste tecniche al timone
non dovrebbero essere considerate come fronzoli o tecniche avanzate. Piuttosto,
sono strumenti essenziali per navigare in 470.

Per apprezzare quanto detto e sviluppare un buon controllo delle vele e del peso, è
necessaria pratica. In una giornata con una brezza moderata (prodiere e
timoniere seduti sul bordo sopravento ma non al trapezio), allenta la presa
sullo stick del timone e cerca di mantenere la barca in una rotta rettilinea.
Osserva l'effetto che lo sbandamento e la regolazione delle vele hanno sul
direzionamento della barca. Dopo averlo
fatto per un po', lascia completamente il timone! Assicurati di farlo lontano da
altre barche poiché probabilmente perderai il controllo e navigherai in cerchio
per un po'. Dopo aver padroneggiato la conduzione in linea retta, togli il timone dalla barca
e prova a navigare in in un circuito. Dovrai alzare leggermente la deriva per
controbilanciare l'effetto che la rimozione del timone ha sul CLR. Quando senti
di avere la barca sotto controllo, e dopo aver rimesso il timone al suo posto,
prova a navigare in cerchio intorno ad una boa. Anche se adesso puoi usare il
timone, scoprirai che più preciso sarai nello sbandare la barca e
nel regolare le vele, più stretti saranno i tuoi cerchi.

Così come per la gran parte degli aspetti della conduzione del 470, esistono due
modi per effettuare una virata: il modo facile e il modo veloce. Il modo facile,
chiamato virata piatta, non è molto diverso da virare qualsiasi altra barca.
Lasca leggermente randa e fiocco mentre navighi di bolina, inizia ad orzare e,
quando le mura a vento saranno cambiate (ovvero superi con la prua la direzione
del vento) cazza nuovamente le vele mentre la barca si posiziona nella nuova
andatura di bolina.
Se il prodiere è fuori sul trapezio, il timoniere deve comunicare l'intenzione a
virare dicendo chiaramente  "pronti a virare" e aspettare che il prodiere si
sposti, si sganci e laschi la scotta del fiocco. È responsabilità del timoniere
aspettare che il prodiere sia pronto prima di virare.

Il metodo veloce per virare, la virata con rollio, coinvolge invece lo
spostamento attivo del peso dell'equipaggio per forzare la barca ad attraversare
l'angolo morto. Una virata con rollio ben fatta oltre a diminuire il tempo in
cui la barca si trova prua a vento, produce un'accelerazione della stessa. La
virata con rollio inizia lasciando la barca sbandare sopravento, spingendo
quindi la barca ad orzare (come descritto in precedenza). Combinando tale
sbandamento con il movimento del timone si ottiene un passaggio al vento
estremamente veloce. Durante il passaggio a vento, a differenza della virata
piatta, timoniere e prodiere devono rimanere sul lato (vecchio) di bolina
lasciando la barca sbandare completamente. Per chiudere la virata, una volta
attraversato l'angolo morto, timoniere e prodiere devono rapidamente e in
sincronia spostarsi sul nuovo sopravento e sporgersi per raddrizzare la barca mentre cazzano le
vele.

La virata con rollio è leggermente più difficile in condizioni di vento che
richiedano il trapezio. Quando il timoniere dice "pronti a virare", il prodiere
si sgancia dal trapezio e lasca la scotta del fiocco mentre è ancora fuori dalla
barca. Quando è pronto, lo comunica al timoniere dicendo "pronto" e il timoniere mette subito il
timone all'orza. Con questo tipo di virata, il prodiere è responsabile della sua
sicurezza; se il timoniere fosse costretto a controllare il prodiere,
quest'ultimo dovrebbe rimanere sospeso, sganciato, per un paio di secondi in
più.

Dal momento in cui la barca inizia a virare, il prodiere dovrebbe cercare di
aspettare un secondo in più prima di spostarsi in modo da velocizzare il
passaggio a vento della barca. La nuova scotta del fiocco dovrebbe essere afferrata il
più vicino possibile al carrello in modo che un solo movimento la cazzi quasi
del tutto. La maggior parte dei prodieri di 470 virano guardando in avanti, ma è
possibile che alcuni la effettuino rivolti a poppa. Prova entrambi i metodi
per valutare quello che ti è più congeniale. In ogni caso, una volta che avrai virato,
cazza e blocca la scotta del fiocco, afferra la maniglia con la mano che sarà in
a prua, girati e metti i piedi sul bordo. A questo punto, il timoniere dovrebbe
essere passato dall'altra parte della barca e star cazzando la randa per
bilanciare la barca. Il prodiere dovrebbe essere a questo punto agganciato e cazzare il fiocco.
Se tutto ciò avviene in sincronia, la barca supererà tutte le altre che stanno
facendo le solite e noiose virate piatte.

Ovviamente, la virata con rollio richiede molta pratica per essere eseguita in
modo efficace e con il giusto tempismo. Un buon esercizio per migliorare la
velocità dell'equipaggio è navigare senza agganciarsi al trapezio. Questo
costringe il prodiere a imparare ad uscire e rientrare senza essere dipendente
dal trapezio.

Una volta che avrai padroneggiato la virata con rollio e ti verrà fluida, prova a eseguire una
seconda virata immediatamente dopo aver completato la prima. Se la barca non si
ferma completamente in acqua, stai eseguendo bene al manovra. Se non riesci a
farlo, continua a fare pratica. La doppia virata, oltre ad essere un buon
esercizio, è un'ottima difesa contro un avversario che ti toglie il vento. Per
di più, è molto efficace per impressionare gli altri velisti. Se sei
orientato alle regate in 470, estendi l'esercizio facendo il maggior numero di
virate con rollio consecutive possibili. È importante non navigare tra una
virata e l'altra per un secondo o due, poiché ciò rende molto più facile
l'esecuzione. In teoria, dovresti essere in grado di virare con rollio da un lato
all'altro del lago (o comunque un numero arbitrario di volte). In pratica, se
riesci a fare con fluidità doppie e triple
virate, puoi ritenerti piuttosto soddisfatto.

\subsubsection{Regolazione delle vele}
\label{subsubsec:RegolazioneDelleVele}
Fino a questo punto, non è stato menzionato nulla su come regolare le varie cime
di controllo del 470. È impossibile dire, ad esempio, che con 12 nodi di vento,
il cunningham dovrebbe essere tirato giù di 1-1/4 pollici. In molti casi, il
modo in cui vengono regolate le cime di controllo dipende dal tuo peso, dall'età
e dal taglio della vela, dal tipo di albero e persino dal tuo particolare stile
di navigazione. Quindi, piuttosto che riassumere i migliori modi per regolare le
vele, ecco alcune informazioni tecniche sul "motore" del 470. Ecco alcuni
suggerimenti di base su come regolare le vele, ma spetta a te uscire ed
esperimentare per osservare cosa succede. Inoltre, non esitare a chiedere a
velisti più esperti la loro opinione su determinati problemi di regolazione.

Le vele hanno una forma tridimensionale piuttosto complicata che è piuttosto
difficile da interpretare senza diversi anni di esperienza in vela. Molto
probabilmente, se hai navigato con un velista esperto, sei stato frustrato dai
suoi continui aggiustamenti delle cime di controllo e dai suoi mormorii di "non
sembrava giusto" in risposta alle domande. Non preoccuparti troppo! Ci sono dei
modi per sviluppare un buon occhio. Sebbene sia difficile discutere della vela
nel suo complesso, ci sono tre aree della vela che indicano, in generale, come
la vela nel suo complesso stia lavorando.

La prima area importante è la parte alta della balumina (ricorda che la balumina
è il "lato obliquo" della randa). Questa deve essere
regolata per presentare il corretto svergolamento (con svergolamento si intente
la torsione della balumina sottovento). Una vela
troppo chiusa non presenterà svergolamento e dunque le stecche più alte punteranno quasi
completamente all'indietro o leggermente a vento. D'altra parte, la sezione alta di una vela con
molto svergolamento (molto aperta), cederà "aprendosi" sottovento (ricorda che
durante l'andatura di bolina il vento arriva lateralmente rispetto alla vela),
depotenziando la vela.
%
La seconda regione da considerare per una corretta regolazione delle vele è la
parte più profonda della vela. Il nome tecnico con cui ci si riferisce a questo
punto è il \emph{grasso} della vela.
%
Infine, la terza aree importante da controllare è l'entrata della vela (la
regione vicina all'inferitura sull'albero). L'angolo di entrata della vela
influenza drasticamente il flusso d'aria sul resto di essa.

Ci sono diversi modi per valutare le tue vele. Prima di tutto, guardale
mentre navighi. A volte potrebbe essere utile infilare la testa al di sotto della randa
per dare un'occhiata al lato sottovento delle vele. In secondo luogo, osserva le
altre barche che navigano vicino a te, ma più veloci. Guarda le balumine delle
loro vele, l'altezza del piede del fiocco sopra il bordo, la posizione del boma
e qualsiasi altro indizio su cosa si stia facendo bene e cosa male. Infine,
metti una barca a terra quando c'è una brezza leggera. Assicurati di farlo con
abbastanza persone intorno per mantenere la barca sotto controllo. Gioca con le
cime di controllo e cammina intorno alla barca per vedere come cambia la forma
delle vele.

% Up to this point, nothing has been mentioned about how to adjust the various
% control lines of the 470. It is impossible to say, for example, that in 12 knots of
% wind, the cunningham should be pulled down 1-1/4 inches. In many cases, the
% way the controls are adjusted is dependent on your weight, sail age and cut, the
% type of mast, and even your particular style of sailing. So, rather than summarize
% the best ways to adjust your sails, here is some technical information on the
% "power plant" of the 470. Here are some basic hints on how to tune the sails, but
% its up to you to go out and experiment to observe what happens. Also, don't be
% shy about asking more experienced sailors for their opinion on certain trim
% problems.
% Sails have a rather complicated three-dimensional shape which is quite difficult
% to interpret without several years of sailing experience. In all likelihood, if you've
% sailed with an experienced sailor, you've been frustrated by his or her constant
% adjustments of controls and mutterings of "it just didn't look right" in response to
% queries. Don't fret too much! There are short cuts to developing a good eye.
% While its difficult to discuss the sail as a whole, there are three areas of the sail
% which indicate, in general, how the whole sail is working.
% The first area is the upper leech. This adjusts for the proper amount of "twist". A
% sail with little twist has the leech, and the give-away top batten, pointing straight
% back or hooked slightly to windward. The leech of a sail with lots of twist, falls off
% "opening up" and depowers the sail.
% The second important area of sail trim concerns the location and depth of the
% deepest part of the sail. The fancy name for this is the point of maximum camber
% or more simply, the draft.
% The third area is the luff of the sail. The amount of fullness, or "entry" of the sail,
% dramatically influences the flow of air over the rest of the sail. There are several
% ways to evaluate your sails. First, look at the sails while you are sailing. It might
% be helpful at times to stick your head under the main to get a good look at the
% leeward side of the sails. Second, watch other boats sailing near, but faster, than
% you . Look at the leeches of their sails, the height of the foot of the jib above the
% splash rail, the location of the boom, and any other clues of what is being done
% right or wrong. Finally, set a boat up on shore sometime when there is a light
% breeze. Be sure to do this with enough people around to keep the boat under
% control. Play with the controls and walk around the boat to see what the sails look
% like.
% As far as 470 sailors are concerned, the wind comes in three basic "flavors." -
% very light, medium and heavy - each of which requires a fairly different sail set up.
% In light air, strangely enough, it is necessary to sail with fairly flat sails but very
% little tension on the sheets. Wind, going all the way around a full sail, tends to
% become separated from the sail. The loss of the laminar flow from the sail, or the
% rudder, is called stall. It's important to remember that sails are made of cloth and
% that changes in wind strength causes them to change shape. For example,
% increasing wind strength automatically causes the camber to move aft and the
% leech to twist off. If a sail is set perfectly in moderate air, what happens in light air
% is the leech will tend to be too tight and the draft goes too far forward.