% !TEX root = ../../main.tex

\subsection{Conduzione di Bolina}
\label{subsec:ConduzioneDiBolina}
Il 470 deve essere portato perfettamente piatto sull'acqua quando si naviga di
bolina, eccetto in condizioni di vento molto leggero, quando la barca può essere
leggermente sbandata sottovento per mantenere le vele gonfie. La forma dello
scafo del 470 lo rende più veloce, meno soggetto a scarroccio e più
controllabile quando completamente piatta. Mantenendo la barca piatta, si riduce
la tendenza della barca a orzare e poggiare da sola, riducendo così l'uso del
timone che rallenterebbe la barca.

\subsubsection{Il prodiere}
\label{subsubsec:IlProdiere}
Il principale compito del prodiere del 470 è mantenere la corretta
inclinazione dello scafo. Con vento leggero, ciò richiede di spostarsi agilmente
da un lato all'altro della barca. Con vento più forte, diventa necessario l'uso
del trapezio. Il lavoro del prodiere ha il fine di consentire al timoniere di
rimanere seduto comodamente in un punto in cui possa vedere sia la forma del
fiocco che l'acqua sopra e sottovento.

L'inclusione del trapezio su una barca a vela permette al progettista di
aumentare l'area velica e diminuire la larghezza della barca, riducendone così
la resistenza. Ancora più importante però, è il grado di liberà aggiuntivo che questo
dispositivo dà all'equipaggio. Durante le prime uscite, specialmente con vento
leggero, il timoniere farà di tutto per tenere il prodiere fuori al trapezio.
Nel momento in cui acquisterete esperienza, il timoniere potrà prestare
sempre meno attenzione alla posizione del prodiere, dedicandosi agli altri suoi
compiti. Ricordate, è compito del prodiere è quello di mantenere la barca
piatta.

Esistono due modi per uscire sul trapezio, da dentro la barca; un modo un po'
più lento ma facile e un modo più veloce. Per uscire nel modo semplice,
agganciati all'anello del trapezio mentre sei seduto sul bordo della barca e
cazza l'\emph{ascensore del trapezio} finché non sei sospeso appena sopra il
bordo della barca. Afferra dunque la maniglia
con la mano più a prua e posiziona la gamba a prua, piegata, sulla seduta. Metti
la mano a poppa sul bordo della barca appena dietro di te. Quando sei pronto
a uscire, metti il tuo peso sul filo, inclinati all'indietro e spingi con la
mano posteriore. Stendi la gamba a prua e porta la gamba a poppa sul bordo
della barca. Dovresti essere ora fuori sul trapezio!

Il modo impegnativo e veloce per uscire è più facile da descrivere: afferra la
maniglia con la mano più a prua, salta fuori e agganciati. Aspetta di essere
fuori dalla barca prima di agganciarti per ottenere virate molto veloci e di
classe che possono farti demolire gli avversari in una regata.

Durante le prime uscite sul trapezio, troverai più comodo appoggiare il piede a
prua contro la sartia, avere l'ascensore del trapezio completamente cazzato e posizionare
i piedi distanziati per mantenere l'equilibrio.

Quando diventerai più esperto, cerca di migliorare il tuo posizionamento e, come
conseguenza, le prestazioni della barca. Avvicina i piedi per spostare il tuo peso il più
possibile verso l'esterno. Per conferire ancora più stabilità alla barca, resta
sulle punte dei
piedi e lasca l'ascensore per abbassare il tuo peso e aumentare la forza
contro-sbandante.
Posizioni più alte dell'ascensore sono utilizzate maggiormente con
vento leggero o onde formate.

Infine, allontanati dalla sartia. Avere il peso
avanzato spinge la prua verso il basso e diminuisce dastricamente le prestazioni della barca.
Quando navighi in acque calme, posizionati a circa 60-90 cm a poppa della
sartia. Così come le onde e il vento aumentano, spostati a poppa fino a che
non ti trovi appena davanti al timoniere, che dovrebbe essere seduto proprio sopra il
carrello della randa. Come prodiere, la tua posizione esatta dipende
dal tuo peso e dal peso del timoniere. Come regola generale, con vento leggero e
acque calme, guarda avanti dove la prua taglia l'acqua. La curva della barca,
dove la prua si allarga e diventa il fondo dello scafo, dovrebbe appena sfiorare
le onde. In condizioni
di mare mosso, la barca dovrebbe sembrare come se stesse saltellando attraverso
le onde. Qualsiasi siano le condizioni, muoviti avanti e indietro per vedere gli
effetti del tuo peso. Annotati mentalmente il comportamento della barca allo
spostarsi del tuo peso ed in particolare se la barca tende a
planare più facilmente, a scavalcare le onde, se sembra più lenta, se tende a "puntare" più in
alto o a spruzzare acqua in modo strano. Chiedi inoltre al timoniere se
riesce a percepire una differenza nel timone in base al tuo spostamento.

L'aspetto critico di un buon lavoro sul trapezio è la fluidità. Troppo spesso i
principianti, e non solo, saltano fuori dalla barca quando la prima raffica
arriva, facendo sbandare la barca a sottovento, per poi rientrare velocemente a
bordo una volta bagnati. Questo continuo oscillare da un lato all'altro mentre
si naviga di bolina è generalmente considerato una cattiva pratica e non risulta
molto divertente.

Il primo requisito per un buon lavoro sul trapezio è che tu debba tenere gli
occhi fuori dalla barca e guardare da dove proviene il vento. Se vedi una grossa
raffica arrivare, puoi saltare fuori dalla barca abbastanza velocemente. D'altro
canto, se vedi che stai per essere colpito da una piccola raffica, sii pronto a
lasciare la scotta più lentamente, rientrando dolcemente.

Se sei già sul trapezio e il vento inizia a calmarsi, non saltare subito in
barca. Prima, siediti mantenendo le gambe dritte e piegati solo in vita. Se il
vento cala ancora, resta sul bordo e piegati in modo da poter rimanere seduto
sul bordo. Quando la prossima raffica arriva, puoi tornare fuori senza
dover passare per il fastidio di entrare ed uscire dalla barca. Ricorda, dal
momento in cui c'è una brezza moderata, il 470 deve essere condotto assolutamente
piatto. Presta attenzione a quanto la barca sta sbandando. Uno sguardo allo
specchio di poppa può aiutare a capire quanto la barca sia piatta.

In condizioni di vento appena sufficiente ad usare il trapezio, il lavoro del
prodiere richiede molta concentrazione e
pazienza. Sii pronto a regolare costantemente il tuo peso per mantenere la barca
in equilibrio. Spesso è una buona idea alzare l'ascensore abbastanza in alto da
tenerti appena fuori dal lato della barca quando sei seduto. Questo ti permette
di uscire facilmente senza dover sollevare il tuo peso ogni volta. Come il vento
aumenta, siediti sempre più fuori bordo mentre sei agganciato al trapezio. Se le
tue gambe sono abbastanza lunghe, sospeso direttamente sopra la deriva.

Altrimenti, tieniti a metà strada spingendoti indietro e spingendoti fuori dal lato
della barca con la mano a poppa. Sii pronto a mettere il piede anteriore sul
bordo della barca quando la raffica aumenta. Se necessario, tieni la scotta del
fiocco vicino per un'emergenza, rilassati e goditi il viaggio.

In caso di raffiche, puoi spostare il tuo peso per mantenere la barca piatta. Se fatto
armonicamente con il timoniere, questo movimento può essere uno strumento
estremamente potente con vento forte. Infatti, oltre a spingere la barca verso il basso,
il movimento fa flettere la cima dell'albero, permettendo di \emph{pompare} la parte
alta della vela (Con pompare si intende far fare un movimento brusco alla vela
che crea una spinta in avanti). Se la barca sbanda eccessivamente
troppo, lasca il fiocco per un istante e per poi cazzarlo di nuovo. Non lasciare il
fiocco libero di sventolare; ciò potrebbe portare alla tua rovina. Un 470 deve
essere tenuto sempre in movimento con vento forte. La barca può scuffiare,
anche con entrambe le vele sventolanti, se è ferma.

\subsubsection{Il timoniere}
\label{subsubsec:IlTimoniere}
Timonare di bolina in un 470, o in qualsiasi altra barca da regata, è
un compito non banale. Richiede concentrazione, osservazione, sperimentazione e
molta pratica. Quando inizi a timonare la barca, passerai molto tempo a
preoccuparti di dove si trova il tuo prodiere e come tenerlo fuori sul trapezio.
Per questo motivo, è meglio provare a navigare con la stessa persona per un po'
di tempo finché non vi abituate entrambi alla barca. Ricorda, finché il prodiere
non è completamente fuori sul trapezio, è sua responsabilità mantenere la barca
piatta e il timoniere deve rimanere seduto in una posizione comoda. Quando
navighi correttamente di bolina, il 470 è in grado di tenere rotte piuttosto
strette (con un angolo rispetto al vento relativamente piccolo).
Trovare tale angolo limite non è scontato ed è necessario passare molto tempo in barca e concentrarsi
quando si naviga di bolina. Un metodo per trovare la rotta più stretta
navigabile è il seguente. Cazzate a ferro tutte le vele (più dettagli a
riguardo più avanti) e timona per mantenere i bandierini del fiocco che
sventolano dritti. I bandierini del fiocco forniscono una misura molto precisa
del suo angolo di scotta. Il bandierino interno (sopravento) sventola prima che
la vela stessa lo faccia indicando dunque che essa è troppo lasca (o che
l'andatura è troppo stretta nel caso in cui sia cazzata a ferro); il bandierino
esterno che sventola, visto in ombra dietro la vela,
indica che la vela è eccessivamente cazzata. Il fiocco è al massimo della sua efficienza quando
entrambi i bandierini fileggiano dritti, senza essere soggetti a turbolenze.
Cazza se il bandierino interno
sventola; lasca la scotta se lo fa quello esterno. Se stai facendo un buon
lavoro, la barca avrà un timone quasi neutro. Una leggera tendenza orziera è
accettabile. Ciò ti permetterà di timonare attraverso le onde con pochissimo
movimento del timone. Prova a timonare con gli occhi chiusi per un po' e presto
sarai in grado di percepire la sensazione della barca quando è prua a vento.

Il 470 è un'imbarcazione facile da tenere piatta, ma è piuttosto difficile da
riportare in posizione una volta che è sbandata. Quando arriva una raffica,
sii pronto a lavorare duramente per un po' per riportare la barca in posizione.
La tecnica di base per raddrizzare la barca non è particolarmente complicata: lasca
la randa leggermente e orza leggermente. Quando
la barca è piatta, cazzate la randa di nuovo, tornate alla rotta corretta e
potete rilassarvi fino alla prossima raffica. In quasi tutte le condizioni, il
timoniere dovrebbe essere seduto il più appruato possibile, vicino all'attacco
della scotta randa. Come
prodiere, ciò mette il tuo peso nella parte più larga della barca, permettendoti
di raggiungere tutte le cime di controllo e di gestire l'inclinazione della
barca al meglio per superare le onde e virare rapidamente. C'è una forte tendenza per
i principianti a "scivolare" a poppa ad ogni possibile occasione. Cerca di
rimanere in avanti. Ricorda, continua a lavorare sulle vele per adattarle alle
condizioni variabili.

\subsubsection{La virata}
\label{subsubsec:LaVirata}
Ci sono tre elementi fattori da dovere considerare quando si vuole virare in 470: il
timone, le vele e lo scafo. Ovviamente puoi tirare o spingere il timone per far
puntare la barca nella giusta direzione. Non così ovvio, ma altrettanto
importante per direzionare la barca è la regolazione dello scafo e delle vele. Su
tutte le imbarcazioni a vela, tutte le forze del vento e dell'acqua, possono
essere considerate applicate in singoli punti sulle vele (Center of Effort-CE) e sullo scafo
(Center of Lateral Resistance-CLR) rispettivamente.

Quando una barca ha timone neutro, il CE è direttamente sopra il CLR e la barca naviga in linea
retta senza pressione sul timone. I velisti di 470 e di altre imbarcazioni ad
alte prestazioni cercano di regolare le loro imbarcazioni per raggiungere questa
situazione. La riduzione del movimento del timone non solo rende la barca più
reattiva, ma la rende effettivamente più veloce grazie alla riduzione della sua
resistenza. Anche se la forma della vela, l'inclinazione dell'albero,
la forma della deriva ecc., influenzano la posizione del CE e del CLR, i
cambiamenti più significativi che possono essere effettuati durante la navigazione riguardano la
regolazione delle vele e lo sbandamento della barca. Oltre a ridurre la quantità
di movimento del timone necessaria, la regolazione della barca e delle vele può
aiutarti a mantenere la barca sotto controllo in condizioni di vento forte, o
durante grandi cambi di rotta come le virate in boa.

Il 470 ha un design del timone notoriamente inefficiente a causa delle
stringenti regole di classe. Non è raro che il flusso d'acqua si separi dalla
lama del timone. Questo fenomeno è noto come stallo. Quando ciò accade, ad
esempio, qunado il timone è girato troppo, esso diventa quasi completamente
inefficace.
In condizioni di vento forte, il timone può creare turbolenze che contribuiscono
a portare la barca alla straorza e a renderla non controllabile. Con vento
leggero, la barca sembrerà lenta e senza controllo mentre deriva verso l'angolo
morto.

Lascando leggermente la randa, il
CE si sposterà in avanti e la barca poggerà. Questo accade poiché il fiocco
fornirà più potenza e forza di rotazione. Al contrario, cazzare leggermente la
randa sposta il CE in avanti e porterà la barca a orzare.

Sbandare la barca cambia la posizione del CLR. Quando la barca sbanda
sottovento, come accade durante una raffica, il CLR si sposta sottovento
portando la barca a orzare. Pertanto, quando una raffica colpisce e la barca
si inclina, è imperativo che tu laschi leggermente la randa per neutralizzare il
timone e far tornare la barca sotto controllo. Semplicemente girare il timone
non è spesso sufficiente.

Quando diventa necessario fare cambiamenti importanti di
direzione, come durante una virata, un'abbattuta o in partenza,
le vele e l'inclinazione possono essere utilizzate per rendere il lavoro del
timone molto più facile. In generale, quando si intende poggiare, lasca la
randa e inclina la barca sopravento. Quando si vuole orzare, permetti
alla barca di sbandare e cazza rapidamente la randa. Queste tecniche al timone
non dovrebbero essere considerate come fronzoli o tecniche avanzate. Piuttosto,
sono strumenti essenziali per navigare in 470.

Per apprezzare quanto detto e sviluppare un buon controllo delle vele e del peso, è
necessaria pratica. In una giornata con una brezza moderata (prodiere e
timoniere seduti sul bordo sopravento ma non al trapezio), allenta la presa
sullo stick del timone e cerca di mantenere la barca in una rotta rettilinea.
Osserva l'effetto che lo sbandamento e la regolazione delle vele hanno sul
direzionamento della barca. Dopo averlo
fatto per un po', lascia completamente il timone! Assicurati di farlo lontano da
altre barche poiché probabilmente perderai il controllo e navigherai in cerchio
per un po'. Dopo aver padroneggiato la conduzione in linea retta, togli il timone dalla barca
e prova a navigare in in un circuito. Dovrai alzare leggermente la deriva per
controbilanciare l'effetto che la rimozione del timone ha sul CLR. Quando senti
di avere la barca sotto controllo, e dopo aver rimesso il timone al suo posto,
prova a navigare in cerchio intorno ad una boa. Anche se adesso puoi usare il
timone, scoprirai che più preciso sarai nello sbandare la barca e
nel regolare le vele, più stretti saranno i tuoi cerchi.


% As with most other aspects of 470 sailing, there are two ways to tack the boat-the
% easy way and a fast way. The easy way, called the flat tack, is not unlike tacking
% any other boat. Uncleat the main and jib when close hauled, push the tiller
% across, change sides and trim in the sails as the boat falls off onto the new tack.
% If the crew is out on the wire, the skipper should say "ready about" and wait until
% the crew swings in, unhooks and uncleats the jib. Its the skipper's responsibility
% to wait until the crew is unhooked and ready before putting the helm down.
% The flashy method of tacking, the roll tack, involves actively shifting crew weight
% to force the boat through the eye of the wind. A properly done roll tack actually
% accelerates the boat. The roll tack is started by allowing the boat to heel up,
% giving it a strong weather helm. Coupled with the turning of the tiller, this allows
% the boat to spin very fast. Unlike during the flat tack, the skipper and crew should
% stay on the (old) weather side after the skipper pushes the tiller over. In fact, they
% should hike hard to roll the boat through the tack. After coming across the wind,
% skipper and crew rapidly scurry to the high side and hike the boat flat while
% trimming in the sails.
% A roll tack is slightly more difficult during trepanning conditions. When the
% skipper says "ready about", the crew unhooks from the trapeze and uncleats the
% jib while still out of the boat. The crew, when ready, says "ready", and the skipper
% immediately puts the helm down. With this sort of tack, the crew is responsible
% for their own well being; if the skipper were forced to check the crew, the crew
% would have to hang, unhooked, for an extra couple of seconds.
% Once the boat begins to tack, the crew should try to wait for an extra second
% while swinging in to roll the boat through the wind. The new jib sheet should be
% grabbed as close to the fair lead as possible so that one pull will get the jib
% almost trimmed. The vast majority of 470 crews tack facing forwards but some
% face aft during a tack. Give both a try before developing any habits. Either way, as
% you come across the boat, trim and cleat the jib, grab the handle with what will be
% your forward hand, pivot around and get your feet out on the rail. By this point,
% the skipper should be across the boat and trimming the main to balance the boat.
% The crew should be hooked up and trimming in the jib. All at the same time, the
% boat races past all the boats who were doing dull old flat tacks.
% Obviously, this type of tack requires lots of practice in order to get the timing
% down. One of the best drills to speed up the crew is sailing without the use of the
% harness. This forces the crew to learn to get out and come in with out being
% dependent on being hooked up.
% Once you can do a smooth roll tack try "double tacking. Do a second tack the
% instantly after finishing a tack. If your boat has not stopped dead in the water,
% you're doing OK. If not, keep practicing. The double tack, in addition to being a
% fine drill, is a useful defense against a tight cover in a race. It is also very effective
% in impressing other sailors. If you are serious about racing 470's, extend the drill
% by doing as many consecutive roll tacks as possible. It is important not to sail
% between tacks, even for a second or two, because this makes it much easier to do
% them. In theory, you should be able to roll tack your way across the lake. In
% practice, if you can consistently do smooth double and triple tacks then you're
% doing pretty well.