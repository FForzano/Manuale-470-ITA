% !TEX root = ../../main.tex

\subsection{Conduzione di Bolina}
\label{subsec:ConduzioneDiBolina}
Il 470 deve essere portato perfettamente piatto sull'acqua quando si naviga di
bolina, eccetto in condizioni di vento molto leggero, quando la barca può essere
leggermente sbandata sottovento per mantenere le vele gonfie. La forma dello
scafo del 470 lo rende più veloce, meno soggetto a scarroccio e più
controllabile quando completamente piatta. Mantenendo la barca piatta, si riduce
la tendenza della barca a orzare e poggiare da sola, riducendo così l'uso del
timone che rallenterebbe la barca.

\subsubsection{Il prodiere}
\label{subsubsec:IlProdiere}
Il principale compito del prodiere del 470 è mantenere la corretta
inclinazione dello scafo. Con vento leggero, ciò richiede di spostarsi agilmente
da un lato all'altro della barca. Con vento più forte, diventa necessario l'uso
del trapezio. Il lavoro del prodiere ha il fine di consentire al timoniere di
rimanere seduto comodamente in un punto in cui possa vedere sia la forma del
fiocco che l'acqua sopra e sottovento.

L'inclusione del trapezio su una barca a vela permette al progettista di
aumentare l'area velica e diminuire la larghezza della barca, riducendone così
la resistenza. Ancora più importante però, è il grado di liberà aggiuntivo che questo
dispositivo dà all'equipaggio. Durante le prime uscite, specialmente con vento
leggero, il timoniere farà di tutto per tenere il prodiere fuori al trapezio.
Nel momento in cui acquisterete esperienza, il timoniere potrà prestare
sempre meno attenzione alla posizione del prodiere, dedicandosi agli altri suoi
compiti. Ricordate, è compito del prodiere è quello di mantenere la barca
piatta.

Esistono due modi per uscire sul trapezio, da dentro la barca; un modo un po'
più lento ma facile e un modo più veloce. Per uscire nel modo semplice,
agganciati all'anello del trapezio mentre sei seduto sul bordo della barca e
cazza l'\emph{ascensore del trapezio} finché non sei sospeso appena sopra il
bordo della barca. Afferra dunque la maniglia
con la mano più a prua e posiziona la gamba a prua, piegata, sulla seduta. Metti
la mano a poppa sul bordo della barca appena dietro di te. Quando sei pronto
a uscire, metti il tuo peso sul filo, inclinati all'indietro e spingi con la
mano posteriore. Stendi la gamba a prua e porta la gamba a poppa sul bordo
della barca. Dovresti essere ora fuori sul trapezio!

Il modo impegnativo e veloce per uscire è più facile da descrivere: afferra la
maniglia con la mano più a prua, salta fuori e agganciati. Aspetta di essere
fuori dalla barca prima di agganciarti per ottenere virate molto veloci e di
classe che possono farti demolire gli avversari in una regata.

Durante le prime uscite sul trapezio, troverai più comodo appoggiare il piede a
prua contro la sartia, avere l'ascensore del trapezio completamente cazzato e posizionare
i piedi distanziati per mantenere l'equilibrio.

Quando diventerai più esperto, cerca di migliorare il tuo posizionamento e, come
conseguenza, le prestazioni della barca. Avvicina i piedi per spostare il tuo peso il più
possibile verso l'esterno. Per conferire ancora più stabilità alla barca, resta
sulle punte dei
piedi e lasca l'ascensore per abbassare il tuo peso e aumentare la forza
contro-sbandante.
Posizioni più alte dell'ascensore sono utilizzate maggiormente con
vento leggero o onde formate.

Infine, allontanati dalla sartia. Avere il peso avanzato spinge la prua verso il
basso e diminuisce dastricamente le prestazioni della barca. Quando navighi in
acque calme, posizionati a circa 60-90 cm a poppa della sartia. Così come le
onde e il vento aumentano, spostati a poppa fino a che non ti trovi appena
davanti al timoniere, che dovrebbe essere seduto proprio sopra il carrello della
randa. Come prodiere, la tua posizione esatta dipende dal tuo peso e dal peso
del timoniere. Come regola generale, con vento leggero e acque calme, guarda
avanti dove la prua fende l'acqua. La parte bassa dello scafo, in corrispondenza
di dove la prua si allarga e diventa il fondo dello scafo, dovrebbe appena
sfiorare le onde. In condizioni di mare mosso, la barca dovrebbe sembrare come
se stesse saltellando attraverso le onde. Qualsiasi siano le condizioni, muoviti
avanti e indietro per vedere gli effetti del tuo peso. Annota mentalmente il
comportamento della barca allo spostarsi del tuo peso ed in particolare se la
barca tende a planare più facilmente, a scavalcare le onde, se sembra più lenta,
se tende a "puntare" più in alto o a spruzzare acqua in modo strano. Chiedi
inoltre al timoniere se riesce a percepire una differenza nel timone in base al
tuo spostamento.

L'aspetto critico di un buon lavoro sul trapezio è la fluidità. Troppo spesso i
principianti, e non solo, saltano fuori dalla barca quando la prima raffica
arriva, facendo sbandare la barca a sottovento, per poi rientrare velocemente a
bordo una volta bagnati. Questo continuo oscillare da un lato all'altro mentre
si naviga di bolina è generalmente considerato una cattiva pratica e non risulta
molto divertente.

Il primo requisito per un buon lavoro sul trapezio è che tu debba tenere gli
occhi fuori dalla barca e guardare da dove proviene il vento. Se vedi una grossa
raffica arrivare, puoi saltare fuori dalla barca abbastanza velocemente. D'altro
canto, se vedi che stai per essere colpito da una piccola raffica, sii pronto a
lasciare la scotta più lentamente, rientrando dolcemente.

Se sei già sul trapezio e il vento inizia a calmarsi, non saltare subito in
barca. Prima, siediti mantenendo le gambe dritte e piegati solo in vita. Se il
vento cala ancora, resta sul bordo e piegati in modo da poter rimanere seduto
sul bordo. Quando la prossima raffica arriva, puoi tornare fuori senza incorrere
nel fastidio di entrare ed uscire dalla barca. Ricorda, dal momento in cui c'è
una brezza moderata, il 470 deve essere condotto assolutamente piatto. Presta
attenzione a quanto la barca sta sbandando. Uno sguardo allo specchio di poppa
può aiutare a capire quanto la barca sia piatta.

In condizioni di vento appena sufficiente ad usare il trapezio, il lavoro del
prodiere richiede molta concentrazione e pazienza. Sii pronto a regolare
costantemente il tuo peso per mantenere la barca in equilibrio. Spesso è una
buona idea alzare l'ascensore abbastanza in alto da tenerti appena fuori dal
lato della barca quando sei seduto. Questo ti permette di uscire facilmente
senza dover sollevare il tuo peso ogni volta. Come il vento aumenta, siediti
sempre più fuori bordo mentre sei agganciato al trapezio. Se le tue gambe sono
abbastanza lunghe, sospeso direttamente sopra la deriva.

Altrimenti, tieniti a metà strada spingendoti indietro e spingendoti fuori dal lato
della barca con la mano a poppa. Sii pronto a mettere il piede anteriore sul
bordo della barca quando la raffica aumenta. Se necessario, tieni la scotta del
fiocco vicino per un'emergenza, rilassati e goditi il viaggio.

In caso di raffiche, puoi spostare il tuo peso per mantenere la barca piatta. Se fatto
armonicamente con il timoniere, questo movimento può essere uno strumento
estremamente potente con vento forte. Infatti, oltre a spingere la barca verso il basso,
il movimento fa flettere la cima dell'albero, permettendo di \emph{pompare} la parte
alta della vela (Con pompare si intende far fare un movimento brusco alla vela
che crea una spinta in avanti). Se la barca sbanda eccessivamente
troppo, lasca il fiocco per un istante e per poi cazzarlo di nuovo. Non lasciare il
fiocco libero di sventolare; ciò potrebbe portare alla tua rovina. Un 470 deve
essere tenuto sempre in movimento con vento forte. La barca può scuffiare,
anche con entrambe le vele sventolanti, se è ferma.

\subsubsection{Il timoniere}
\label{subsubsec:IlTimoniere}
Timonare di bolina in un 470, o in qualsiasi altra barca da regata, è un compito
non banale. Richiede concentrazione, osservazione, sperimentazione e molta
pratica. Quando inizi a timonare la barca, passerai molto tempo a preoccuparti
di dove si trova il tuo prodiere e come tenerlo fuori sul trapezio. Per questo
motivo, è meglio provare a navigare con la stessa persona per un po' di tempo
finché non vi abituate entrambi alla barca. Ricorda, finché il prodiere non è
completamente fuori sul trapezio, è sua responsabilità mantenere la barca piatta
e il timoniere deve rimanere seduto in una posizione comoda. Quando navighi
correttamente di bolina, il 470 è in grado di tenere rotte piuttosto strette
(con un angolo rispetto al vento relativamente piccolo). Trovare tale angolo
limite non è scontato ed è necessario passare molto tempo in barca e
concentrarsi quando si naviga di bolina. Un metodo per trovare la rotta più
stretta navigabile è il seguente. Cazzate a ferro tutte le vele (più dettagli a
riguardo più avanti) e timona per mantenere i telltales (filetti solitamente di
lana attaccati alle vele) del fiocco che sventolano dritti. 

I telltales del fiocco forniscono una misura molto precisa del suo ``stato di
regolazione''. Il telltale interno (sopravento) sventola prima che la vela
stessa lo faccia indicando dunque che essa è troppo lasca (o che l'andatura è
troppo stretta nel caso in cui sia cazzata a ferro); il telltale esterno che
sventola, visto in ombra dietro la vela, indica che la vela è eccessivamente
cazzata. Il fiocco è al massimo della sua efficienza quando entrambi i
bandierini fileggiano dritti, senza essere soggetti a turbolenze. Cazza se il
bandierino interno sventola; lasca la scotta se lo fa quello esterno. Se stai
facendo un buon lavoro, la barca avrà un timone quasi neutro. Una leggera
tendenza orziera è accettabile. Ciò ti permetterà di timonare attraverso le onde
con pochissimo movimento del timone. Prova a timonare con gli occhi chiusi per
un po' e presto sarai in grado di percepire la sensazione della barca quando è
prua a vento.

Il 470 è un'imbarcazione facile da tenere piatta, ma è piuttosto difficile da
riportare in posizione una volta che è sbandata. Quando arriva una raffica, sii
pronto a lavorare duramente per un po' per riportare la barca in posizione. La
tecnica di base per raddrizzare la barca non è particolarmente complicata: lasca
un po' la randa e orza leggermente. Quando la barca è piatta, cazzate la randa
di nuovo, tornate alla rotta corretta e potete rilassarvi fino alla prossima
raffica. In quasi tutte le condizioni, il timoniere dovrebbe essere seduto il
più appruato possibile, vicino all'attacco della scotta randa. Ciò mette il tuo
peso nella parte più larga della barca, permettendoti di raggiungere tutte le
cime di controllo e di gestire l'inclinazione della barca al meglio per superare
le onde e virare rapidamente. C'è una forte tendenza per i principianti a
``scivolare'' a poppa ad ogni possibile occasione. Cerca di rimanere in avanti.
Ricorda, continua a lavorare sulle vele per adattarle alle condizioni variabili.

\subsubsection{La virata}
\label{subsubsec:LaVirata}
Ci sono tre elementi fattori da dovere considerare quando si vuole virare in
470: il timone, le vele e lo scafo. Ovviamente puoi tirare o spingere il timone
per far puntare la barca nella giusta direzione. Non così ovvio, ma altrettanto
importante per direzionare la barca è la regolazione dello scafo e delle vele.
Su tutte le imbarcazioni a vela, tutte le forze del vento e dell'acqua, possono
essere considerate applicate in singoli punti sulle vele, \ac{CE}, e sullo
scafo, \ac{CLR}, rispettivamente.

Quando una barca ha timone neutro, il \ac{CE} è direttamente sopra il \ac{CLR} e
la barca naviga in linea retta senza pressione sul timone. I velisti di 470 e di
altre imbarcazioni ad alte prestazioni cercano di regolare le loro imbarcazioni
per raggiungere questa situazione. La riduzione del movimento del timone non
solo rende la barca più reattiva, ma la rende effettivamente più veloce grazie
alla riduzione della sua resistenza. Anche se la forma della vela,
l'inclinazione dell'albero, la forma della deriva ecc., influenzano la posizione
del \ac{CE} e del \ac{CLR}, i cambiamenti più significativi che possono essere
effettuati durante la navigazione riguardano la regolazione delle vele e lo
sbandamento della barca. Oltre a ridurre la quantità di movimento del timone
necessaria, la regolazione della barca e delle vele può aiutarti a mantenere la
barca sotto controllo in condizioni di vento forte, o durante grandi cambi di
rotta come le virate in boa.

Il 470 ha un design del timone notoriamente inefficiente a causa delle
stringenti regole di classe. Non è raro che il flusso d'acqua si separi dalla
lama del timone. Questo fenomeno è noto come stallo. Quando ciò accade, ad
esempio, qunado il timone è girato troppo, esso diventa quasi completamente
inefficace. In condizioni di vento forte, il timone può creare turbolenze che
contribuiscono a portare la barca alla straorza e a renderla non controllabile.
Con vento leggero, la barca sembrerà lenta e senza controllo mentre deriva verso
l'angolo morto.

Lascando leggermente la randa, il \ac{CE} si sposterà in avanti e la barca poggerà.
Questo accade poiché il fiocco fornirà più potenza e forza di rotazione. Al
contrario, cazzare leggermente la randa sposta il \ac{CE} in avanti e porterà la
barca a orzare.

Sbandare la barca cambia la posizione del \ac{CLR}. Quando la barca sbanda
sottovento, come accade durante una raffica, il \ac{CLR} si sposta sottovento
portando la barca a orzare. Pertanto, quando una raffica colpisce e la barca si
inclina, è imperativo lascare leggermente la randa per neutralizzare il
timone e far tornare la barca sotto controllo. Semplicemente girare il timone
non è spesso sufficiente.

Quando diventa necessario fare cambiamenti importanti di direzione, come durante
una virata, un'abbattuta o in partenza, le vele e l'inclinazione possono essere
utilizzate per rendere il lavoro del timone molto più facile. In generale,
quando si intende poggiare, lasca la randa e inclina la barca sopravento. Quando
si vuole orzare, permetti alla barca di sbandare e cazza rapidamente la randa.
Queste tecniche al timone non dovrebbero essere considerate come fronzoli o
tecniche avanzate. Piuttosto, sono strumenti essenziali per navigare in 470.

Per apprezzare quanto detto e sviluppare un buon controllo delle vele e del
peso, è necessaria pratica. In una giornata con una brezza moderata (prodiere e
timoniere seduti sul bordo sopravento ma non al trapezio), allenta la presa
sullo stick del timone e cerca di mantenere la barca in una rotta rettilinea.
Osserva l'effetto che lo sbandamento e la regolazione delle vele hanno sul
direzionamento della barca. Dopo averlo fatto per un po', lascia completamente
il timone! Assicurati di farlo lontano da altre barche poiché probabilmente
perderai il controllo e navigherai in cerchio per un po'. Dopo aver
padroneggiato la conduzione in linea retta, togli il timone dalla barca e prova
a navigare in in un circuito. Dovrai alzare leggermente la deriva per
controbilanciare l'effetto che la rimozione del timone ha sul \ac{CLR}. Quando
senti di avere la barca sotto controllo, e dopo aver rimesso il timone al suo
posto, prova a navigare in cerchio intorno ad una boa. Anche se adesso puoi
usare il timone, scoprirai che più preciso sarai nello sbandare la barca e nel
regolare le vele, più stretti saranno i tuoi cerchi.

Così come per la gran parte degli aspetti della conduzione del 470, esistono due
modi per effettuare una virata: il modo facile e il modo veloce. Il modo facile,
chiamato virata piatta, non è molto diverso da virare qualsiasi altra barca.
Lasca leggermente randa e fiocco mentre navighi di bolina, inizia ad orzare e,
quando le mura a vento saranno cambiate (ovvero superi con la prua la direzione
del vento) cazza nuovamente le vele mentre la barca si posiziona nella nuova
andatura di bolina. Se il prodiere è fuori sul trapezio, il timoniere deve
comunicare l'intenzione a virare dicendo chiaramente  "pronti a virare" e
aspettare che il prodiere si sposti, si sganci e laschi la scotta del fiocco. È
responsabilità del timoniere aspettare che il prodiere sia pronto prima di
virare.

Il metodo veloce per virare, la virata con rollio, coinvolge invece lo
spostamento attivo del peso dell'equipaggio per forzare la barca ad attraversare
l'angolo morto. Una virata con rollio ben fatta oltre a diminuire il tempo in
cui la barca si trova prua a vento, produce un'accelerazione della stessa. La
virata con rollio inizia lasciando la barca sbandare sopravento, spingendo
quindi la barca ad orzare (come descritto in precedenza). Combinando tale
sbandamento con il movimento del timone si ottiene un passaggio al vento
estremamente veloce. Durante il passaggio a vento, a differenza della virata
piatta, timoniere e prodiere devono rimanere sul lato (vecchio) di bolina
lasciando la barca sbandare completamente. Per chiudere la virata, una volta
attraversato l'angolo morto, timoniere e prodiere devono rapidamente e in
sincronia spostarsi sul nuovo sopravento e sporgersi per raddrizzare la barca
mentre cazzano le vele.

La virata con rollio è leggermente più difficile in condizioni di vento che
richiedano il trapezio. Quando il timoniere dice "pronti a virare", il prodiere
si sgancia dal trapezio e lasca la scotta del fiocco mentre è ancora fuori dalla
barca. Quando è pronto, lo comunica al timoniere dicendo "pronto" e il timoniere
mette subito il timone all'orza. Con questo tipo di virata, il prodiere è
responsabile della sua sicurezza; se il timoniere fosse costretto a controllare
il prodiere, quest'ultimo dovrebbe rimanere sospeso, sganciato, per un paio di
secondi in più.

Dal momento in cui la barca inizia a virare, il prodiere dovrebbe cercare di
aspettare un secondo in più prima di spostarsi in modo da velocizzare il
passaggio a vento della barca. La nuova scotta del fiocco dovrebbe essere
afferrata il più vicino possibile al carrello in modo che un solo movimento la
cazzi quasi del tutto. La maggior parte dei prodieri di 470 virano guardando in
avanti, ma è possibile che alcuni la effettuino rivolti a poppa. Prova entrambi
i metodi per valutare quello che ti è più congeniale. In ogni caso, una volta
che avrai virato, cazza e blocca la scotta del fiocco, afferra la maniglia con
la mano che sarà in a prua, girati e metti i piedi sul bordo. A questo punto, il
timoniere dovrebbe essere passato dall'altra parte della barca e star cazzando
la randa per bilanciare la barca. Il prodiere dovrebbe essere a questo punto
agganciato e cazzare il fiocco. Se tutto ciò avviene in sincronia, la barca
supererà tutte le altre che stanno facendo le solite e noiose virate piatte.

Ovviamente, la virata con rollio richiede molta pratica per essere eseguita in
modo efficace e con il giusto tempismo. Un buon esercizio per migliorare la
velocità dell'equipaggio è navigare senza agganciarsi al trapezio. Questo
costringe il prodiere a imparare ad uscire e rientrare senza essere dipendente
dal trapezio.

Una volta che avrai padroneggiato la virata con rollio e ti verrà fluida, prova
a eseguire una seconda virata immediatamente dopo aver completato la prima. Se
la barca non si ferma completamente in acqua, stai eseguendo bene al manovra. Se
non riesci a farlo, continua a fare pratica. La doppia virata, oltre ad essere
un buon esercizio, è un'ottima difesa contro un avversario che ti toglie il
vento. Per di più, è molto efficace per impressionare gli altri velisti. Se sei
orientato alle regate in 470, estendi l'esercizio facendo il maggior numero di
virate con rollio consecutive possibili. È importante non navigare tra una
virata e l'altra per un secondo o due, poiché ciò rende molto più facile
l'esecuzione. In teoria, dovresti essere in grado di virare con rollio da un
lato all'altro del lago (o comunque un numero arbitrario di volte). In pratica,
se riesci a fare con fluidità doppie e triple virate, puoi ritenerti piuttosto
soddisfatto.

\subsubsection{Regolazione delle vele}
\label{subsubsec:RegolazioneDelleVele}
Fino a questo punto, non è stato menzionato nulla su come regolare le varie cime
di controllo del 470. È impossibile dire, ad esempio, che con 12 nodi di vento,
il \cunningham dovrebbe essere tirato giù di 1-1/4 pollici. In molti casi, il
modo in cui vengono regolate le cime di controllo dipende dal tuo peso, dall'età
e dal taglio della vela, dal tipo di albero e persino dal tuo particolare stile
di navigazione. Quindi, piuttosto che riassumere i migliori modi per regolare le
vele, ecco alcune informazioni tecniche sul ``motore'' del 470. Ecco alcuni
suggerimenti di base su come regolare le vele, ma spetta a te uscire ed
esperimentare per osservare cosa succede. Inoltre, non esitare a chiedere a
velisti più esperti la loro opinione su determinati problemi di regolazione.

Le vele hanno una forma tridimensionale piuttosto complicata che è piuttosto
difficile da interpretare senza diversi anni di esperienza in vela. Molto
probabilmente, se hai navigato con un velista esperto, ti sei sentito frustrato
dai suoi continui aggiustamenti delle cime di controllo e dai suoi mormorii di
``non sembrava giusto'' in risposta alle domande. Non preoccuparti troppo! Ci
sono dei modi per sviluppare un buon occhio. Sebbene sia difficile discutere
della vela nel suo complesso, ci sono tre aree della vela che indicano, in
generale, come la vela nel suo complesso stia lavorando.

La prima area importante è la \upperleech (ricorda che la \leech è il ``lato
obliquo'' della randa). Questa deve essere regolata per presentare il corretto
\twist (con \twist si intente la torsione della \leech sottovento). Una vela
troppo chiusa non presenterà \twist e dunque le stecche più alte punteranno
quasi completamente all'indietro o leggermente \windward. D'altra parte, la
sezione alta di una vela con molto \twist (molto aperta), cederà, ``aprendosi''
\leeward (ricorda che durante l'andatura di bolina il vento arriva lateralmente
rispetto alla vela), depotenziando la vela.
%
La seconda regione da considerare per una corretta regolazione delle vele è la
parte più profonda della vela. Il nome tecnico con cui ci si riferisce a questo
punto è il \emph{\draft} della vela.
%
Infine, la terza aree importante da controllare è l'\luff (la regione vicina
attaccata all'albero). L'angolo di entrata della vela influenza drasticamente il
flusso d'aria sul resto di essa.

Ci sono diversi modi per valutare le tue vele. Prima di tutto, guardale mentre
navighi. A volte potrebbe essere utile infilare la testa al di sotto della randa
per dare un'occhiata al lato \leeward delle vele. In secondo luogo, osserva le
altre barche che navigano vicino a te, ma più veloci. Guarda le \leeches delle
loro vele, l'altezza della base del fiocco, la posizione del boma e qualsiasi
altro indizio su cosa si stia facendo bene e cosa male. Infine, metti una barca
a terra quando c'è una brezza leggera. Assicurati di farlo con abbastanza
persone intorno per mantenere la barca sotto controllo. Gioca con le cime di
controllo e cammina intorno alla barca per vedere come cambia la forma delle
vele.

Per gli equipaggi del 470, il vento può essere classificato in tre categorie:
molto leggero, medio e forte. Ognuna di queste categorie richiede una
regolazione delle vele piuttosto diversa. In condizioni di \textbf{vento
leggero}, abbastanza stranamente, è necessario navigare con vele piatte ma con
pochissima tensione sulle \sheets. Il vento, essendo attraversato da una vela
piena, tende a separarsi da essa. La perdita del flusso laminare sulla vela, o
sul timone, è chiamata stallo. È importante ricordare che le vele sono fatte di
tessuto e che i cambiamenti nella forza del vento le fanno cambiare forma. Per
esempio, un aumento della forza del vento causa automaticamente lo spostamento
del \draft verso poppa e un incremento dello \twist. Una vela che è regolata
perfettamente in condizioni di vento moderato, in condizioni di vento leggero
apparirà con troppo poco \twist (vela troppo chiusa) e il \draft sarà troppo
avanti.

Per aumentare lo \twist, chiudere la vela e spostare il \draft verso la
posizione corretta; l'albero deve essere flesso in condizioni di vento leggero.
Purtroppo, i due strumenti più importanti per flettere l'albero, la \sheet della
randa e il \vang, non possono essere utilizzati a tal fine perché una tensione
eccessiva su queste linee appiattirebbe troppo la vela e chiuderebbe
eccessivamente la \leech. La conoscenza del \prebend dell'albero risulta
pertanto essenziale in queste condizioni (solitamente il \prebend dell'albero si
ottiene tramite una corretta configurazione di crocette e tensione del rig). Il
\cunningham d'altro canto, non deve essere cazzato, poiché tende a spostare il
\draft in avanti e ad appiattire la vela. Il \outhaul invece, può essere cazzato
ma non eccessivamente.

Il miglior indicatore per regolare la randa è costituito dai \telltales sulla
\leech della vela. Se la vela si trova in condizione di stallo, i \telltales
sventoleranno o si arricceranno sul lato \leeward della vela. In condizioni di
vento leggero, è meglio avere i \telltales più in alto in stallo per circa
$1/3$-$1/2$ del tempo e tesi all'indietro il resto del tempo. I controlli più
importanti su cui agire per ottenere tale risultato sono la \sheet e il carrello
della randa. Se giocare con questi controlli o con il resto delle cime non dà i
risultati sperati, prova ad ammainare la vela e a controllare la tensione delle
stecche più in alto. Una maggiore tensione rende la vela più piena, la \leech
più tesa e i \telltales più spesso in stallo.

Negli anni il sistema del carrello della randa del 470 ha subito una grande
evoluzione. In passato, tutte le barche erano equipaggiate con un carrello
convenzionale costituito da un singolo binario. Molti velisti sono poi passati
ad un sistema fisso (senza carrello), che rendeva più facile regolare la vela
con vento forte. Diversi metodi sono stati provati per rendere il punto di
\sheet regolabile per l'uso con vento leggero, ma non si è giunti ad un
risultato soddisfacente fino a quando non è stato sviluppato il sistema a due
carrelli, con due binari separati, uno a dritta e uno a sinistra.

In condizioni di vento leggero con un solo carrello, la tensione della \leech è
facilmente regolabile con una combinazione di posizione del carrello, di solito
\windward, e tensione della \sheet. Quando si naviga di bolina, la \sheet non
agisce più facendo entrare e uscire il boma. La maggior parte della tensione
della \sheet è invece diretta verso il basso e di conseguenza agisce più che
altro sulla tensione sulla \leech (e di conseguenza sullo \twist), causando lo
stallo dei \telltales per la maggior parte del tempo.

Una volta che il corretto \twist è stato impostato per la randa, la posizione
del boma può essere regolata con il carrello. Con punto di \sheet fisso (senza
carrello), questo tipo di regolazione diventa più complicata. La vela avrà il
giusto \twist ma una regolazione sbagliata se la \sheet è lasciata andare, o si
troverà ad essere troppo chiusa se la vela è tirata tutta dentro (scotta
cazzata).

Con un sistema a due carrelli, il carrello \leeward può essere cazzato con vento
leggero rendendo di fatto operativo solo il carrello \windward. Ciò implica che
se stai imparando e non vuoi preoccuparti dei carrelli, probabilmente ti
conviene lasciare entrambi i carrelli fuori cazzandone le relative cime.
Analogamente, se hai un solo carrello, non preoccuparti di esso finché non ti
senti a tuo agio in barca.

Potresti voler dimostrare a te stesso che il carrello \leeward ha un effetto
drastico sulla forma della vela guardando una barca messa a terra. Cazza
completamente entrambi i carrelli e cazzata la randa tutta dentro. Se ti sposti
sul lato \leeward della barca e guardi la \leech, vedrai che è molto chiusa.
Ora, lascia andare il carrello \leeward e guarda la differenza nella vela.
Prova a modificare la posizione del boma con il carrello \windward e a cambiare
la tensione della \sheet contemporaneamente. Infine, guarda gli elementi della
vela che ti permettono di valutare come essa stia lavorando (come spiegato in
precedenza). Suggerimento: oltre al comportamento dei \telltales della \leech,
l'angolo delle stecche rispetto al boma ti dà spesso un'immagine precisa della
forma della balumina.

Condizioni di \textbf{vento medio} richiedono differenti regolazioni delle vele.
Le vele possono essere regolate per lavorare al massimo delle loro capacità e
non c'è un serio rischio di stallo. Il vento è considerato medio da quando
l'equipaggio è appena seduto sul serbatoio \windward fino al punto in cui il
prodiere è completamente steso al trapezio e non si riesce a mantenere la
barca piatta senza lasciare andare la randa.

Per ottenere il massimo potere dalle tue vele, lasca leggermente il \outhaul e
tieni la randa cazzata. Molti equipaggi principianti tendono a tenere le
vele poco cazzate. La scotta dovrebbe essere cazzata sino a che i rimandi di
essa sul boma (puleggie/carrucole della \sheet) raggiungono la verticale dei
rimandi sul carrello.

Una volta che il prodiere è sul trapezio, puoi iniziare a mettere in tensione il
\cunningham. Presta attenzione a come il \cunningham regola la forma della vela.
La tensione sulla parte anteriore sposta il punto di massimo spessore in avanti
e rende l'ingresso più pieno. Se le tue vele sono vecchie e gonfie, il punto di
massimo spessore sarà spostato indietro e potresti dover cazzare il \cunningham
prima che il prodiere sia sul trapezio. Il \vang può iniziare dunque ad essere
cazzato dopo che il prodiere è sul trapezio. Questo manterrà la \leech tesa,
in particolare nella parte bassa, impedendo al boma di alzarsi con l'aumento del
vento. Il \vang, purtroppo, spinge il boma in avanti e fa piegare l'albero e
depotenziare la vela. In condizioni di vento medio, si vuole ottenere il massimo
potere possibile dalla vela e impedire all'albero di piegarsi. Questo richiede
di bloccare la flessione dell'albero tramite il \mastpull, un cavo d'acciaio
posizionato nella fessura sullo scafo attraverso la quale l'albero passa,
che blocca l'albero impedendogli di flettersi troppo verso prua.

Dal momento in cui il prodiere è al trapezio, un sistema senza carrello risulta
essere vantaggioso rispetto a uno con carrello. Con entrambi i carrelli cazzati
all'esterno, la tensione della balumina è regolata unicamente dal vang. A causa
di questo trasferimento di funzione della scotta della randa al vang, questo
sistema è talvolta chiamato \emph{vang sheeting}. La balumina dovrebbe essere
regolata per passare gradualmente da una situazione di stallo per $1/3$-$1/2$
del tempo totale, indicato dal telltale della balumina, quando l'equipaggio è
seduto sul bordo, a una situazione in cui è in stallo costantemente quando il
prodiere è completamente steso al trapezio. Così come il vento aumenta, sarà
necessario cazzare leggermente tutte le cime per mantenere la forma della vela.
Con vento leggero, la balumina rimarrà tesa da sola, ma con l'aumento del vento
la vela si aprirà sempre di più. L'apertura della vela diventerebbe eccessiva se
non si utilizzasse il vang come controllo.

Considera che il vento spesso non è stabile per più
di un paio di secondi alla volta. Se si utilizza la tecnica suggerita di cazzare molto
la randa, è necessario essere pronti a lasciare andare la vela nelle
raffiche e a cazzarla nuovamente nei momenti di calma. Navigando di bolina è
molto importante mantenere la barca costantemente livellata e avere le vele ben
regolate.

Quando il vento diventa abbastanza forte da richiedere una riduzione
di potenza, l'albero dovrebbe essere lasciato flettere lascando le regolazioni
fini della ghinda. La tensione del
vang e della randa farà flettere l'albero, appiattendo la vela e
aprendo la balumina. Il tesabase può essere cazzato abbastanza forte e sarà
necessario cazzare molto \cunningham per spostare il grasso in avanti. Ricorda, è
il vento forte che spinge il punto di massimo spessore indietro, tu vuoi solo
che rimanga al suo posto.

Entrambi i carrelli della scotta randa dovrebbero essere cazzati (spostati al massimo
verso l'esterno) per il vang sheeting. Una tensione sostanziale del vang è
necessaria per controllare la tensione della balumina, appiattire la vela e far
flettere l'albero. Quando la randa è lasciata andare nelle raffiche, il vang
permetterà alla barca di continuare a navigare impedendo alla balumina di
svergolare troppo.

In condizioni di vento forte, le vele dovrebbero essere sempre regolate per
ottenere un flusso stazionario (non in stallo). Prova a vedere tu stesso cosa succede quando provi
a farle stallo. Come già detto, il
470 è facile da mantenere piatto ma difficile da appiattire. Le raffiche devono
essere gestite lasciando andare la randa e ricazzando rapidamente per mantenere
la barca in movimento. Può essere d'aiuto sollevare leggermente la deriva dalla
sua posizione normale perpendicolare allo scafo (in questo modo si avrà un po'
più di scarroccio laterale ma diminuirà lo sbandamento).

Quando ti senti abbastanza a tuo agio con il vento forte, puoi provare una
tecnica leggermente diversa dal solito stile di vela. Se le condizioni sono
giuste, il 470 può essere fatto planare a bolina. Questo significa che la barca
è condotta come se fosse al traverso. Ecco cosa devi fare:
\begin{itemize}
      \item Cazza il vang e lasca la randa per bilanciare lo sbandamento;
      \item Mantieni la barca piatta;
      \item Sfrutta il trapezio;
      \item Sposta i carrelli del fiocco a poppa in modo da aumentare il flusso
            d'aria sulla randa;
      \item Lasca la ghinda per avere una vela piatta e una balumina aperta;
      \item Quando non riesci a controllare la barca, lasca un po' la randa e orza
            leggermente.
\end{itemize}
L'ultimo passaggio potrebbe portare ad un bagno a meno che non si facciano tutti
gli altri passaggi correttamente. Tuttavia, se il vento e il tuo tempismo sono
giusti, la barca decollerà.

Un'ultima considerazione prima di passare alla regolazione del fiocco. Le
condizioni del mare sono importanti quanto quelle del vento per influenzare la
regolazione delle vele. Le onde grandi richiedono vele più piene rispetto a un
mare piatto per poterle attraversare. A volte il vento e le onde cooperano per
rendere la tua scelta facile. Ad esempio, con vento forte e mare calmo, in
genere una brezza di terra, le vele dovrebbero essere molto piatte. A volte, le
condizioni non sono così favorevoli da permettere scelte dirette e semplici. Un
esempio è quando il mare è mosso ma il vento sta calando. Esci e fai molti tentativi per
vedere come risponde la tua barca. Non ci sono suggerimenti facili se non la
pratica e l'esperienza.

La regolazione del fiocco segue in generale la
regolazione della randa; pieno per avere potenza in condizioni di vento medio e
mare mosso, piatto in mare calmo e in condizioni di vento molto leggero o molto forte.

Come detto in precedenza, la tensione del rig (strallo e sartie) influenza la
forma del fiocco controllandone lo svergolamento e l'angolo di entrata. È
difficile che si abbia troppa tensione sul rig, indipendentemente dalle condizioni. In
alcuni fiocchi, la tensione del tessuto può essere regolata in modo abbastanza
indipendente dalla tensione del rig con una piccola cima di regolazione
nell'angolo di mura o in testa alla vela. Questa cima funziona come un
\cunningham per il fiocco, anche se è quasi impossibile regolarla mentre la barca
è in movimento.

I due controlli a disposizione per la regolazione del fiocco sono la posizione
del carrello e la tensione della scotta. La tensione della scotta influisce
sullo svergolamento della balumina e sull'angolo di entrata. Non sono controlli
indipendenti: cazzare la scotta simultaneamente chiude la balumina e appiattisce
la vela. Un avvertimento: quando si tratta di regolare il carrello, il fatto che
i binari siano lunghi 30 cm non significa che sia mai necessario un movimento di
30 cm. Di solito, un movimento di 7-10 cm è il massimo necessario per qualsiasi
fiocco. Il binario lungo permette una varietà di tagli di vela diversi. Come
guida per la regolazione del fiocco, traccia due segni con una matita spessa sul
bordo della barca a 35 e 40 cm dal bordo. Il piede del fiocco sarà regolato tra
questi segni, più esterno quando il fiocco deve essere pieno per il mare mosso e
il vento medio, o quando la balumina deve essere aperta per il vento forte, e
più interno in condizioni moderate quando è richiesta la massima potenza. 

Quattro coppie di telltales facilitano la regolazione del fiocco. Tre coppie
sono posizionate lungo il lato di mura e la quarta in alto sulla balumina come
indicatore di stallo. Quando imposti il fiocco per la prima volta, mettiti a
sottovento e guarda la balumina. Il telltale della balumina dovrebbe essere
regolato in modo simile a quello della randa, stallo per $1/3$-$1/2$ del tempo
in condizioni di vento leggero e medio, e meno in altre situazioni. Cazzare la
scotta chiude la balumina e fa aumenta lo stallo dei telltales. Analogamente,
spostare il carrello in avanti mette più forza verso il basso sulla balumina e
fa aumentare lo stallo della vela. 

Poiché non sono controlli indipendenti,
cazzare la scotta chiude la balumina e appiattisce la vela. Bilancia la tensione
della scotta e la posizione del carrello fino a quando non pensi che la pienezza
e lo svergolamento siano corretti. Ora, sposta lo sguardo dalla balumina alla
mura e fai orzare leggermente il timoniere. Tutte e tre le coppie di telltales
dovrebbero sventolare contemporaneamente. Se il telltale superiore sventola per
primo, è un'indicazione che il carrello è troppo indietro e non abbastanza
tensione sulla scotta viene trasmessa alla parte superiore della vela. Sposta il
carrello in avanti e ricalibra la scotta per ottenere il giusto stallo della
balumina. Se il telltale inferiore sventola, sposta il carrello indietro e
ricalibra. Una volta che la vela è regolata correttamente, torna sul lato
sopravento della barca e fatti un'idea di dove il fiocco è regolato rispetto al
bordo della barca in modo da poterlo rapidamente ricalibrare dopo la prossima
virata. 

In condizioni di vento molto leggero, lo spazio tra la randa e il fiocco dovrebbe
essere il maggiore possibile. Il vento leggero riesce ad attraversare uno spazio
stretto. Potrebbe essere necessario che l'equipaggio, seduto a sottovento, tenga
le scotte del fiocco fuori bordo e leggermente in alto. In ogni caso, la
tensione della scotta dovrebbe essere molto poca. Le condizioni moderate di vento
richiedono la massima attenzione alla regolazione del fiocco. Come per la randa,
il fiocco non dovrebbe essere troppo tensionato con brezze moderata ma comunque
abbastanza leggere. Questo causerà alla vela di essere stallo troppo spesso. Con
l'aumento del vento, il fiocco può essere regolato in modo da chiudere la
balumina per ottenere potenza. 

In condizioni di vento molto forte, il fiocco
dovrebbe essere piatto e avere la balumina aperta. Questo si può ottenere
tirando il carrello indietro e
cazzando molto la scotta. Sii pronto a lascare rapidamente il fiocco
e a rimetterlo a punto dopo una raffica improvvisa. 