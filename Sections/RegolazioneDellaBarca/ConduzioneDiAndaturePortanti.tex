% !TEX root = ../../main.tex

\subsection{Conduzione con andature portanti}
\label{sec:conduzione_andature_portanti}
Un 470 si comporta in modo molto diverso in andature portanti (dal lasco alla
poppa) rispetto a quando è a bolina. Risalendo il vento, la barca richiede uno
sforzo costante e continuo, piccoli aggiustamenti della vela e pochi movimenti
del corpo. La barca è una piattaforma abbastanza stabile che tende costantemente
a tirare nella direzione del vento. ``Scendendo il vento'', il 470 tende ad
essere molto più instabile. Condurre la barca richiede molto più movimento e
cambiamenti drastici della vela.

Le regolazioni discendendo il vento sono abbastanza semplici. Nella maggior
parte dei casi, le cime di controllo possono essere lascate. Il \cunningham è
facile da rilasciare mentre il \outhaul è spesso lasciato invariato a meno che
tu non sia in una regata lunga o abbia un meccanismo che consente un rilascio
rapido. La cosa importante da ricordare è la regolazione del \vang. Se avevi il
\vang cazzato durante la navigazione di bolina e inizi a poggiare verso un
traverso o lasco, assicurati di lascarlo completamente. Se lasciato troppo teso,
il \vang rallenterà la barca, peggiorerà significativamente la manovrabilità, non
permetterà alla vela di essere rapidamente rilasciata nelle raffiche e, tirando
il boma verso il basso, può causare dolorose strambate (abbattute involontarie).

Durante la navigazione di lasco, il \vang dovrebbe essere utilizzato come un
``acceleratore''. Un \vang più teso porta a meno \twist e più potenza sino a
quando la vela non risulti troppo piatta. Di lasco, la randa dovrebbe prendere
vento in modo uniforme, né prima in alto né prima in basso, e le stecche
dovrebbero essere allineate se si ha la giusta tensione del \vang. Al traverso e
al lasco, un \vang risulta sovraccaricato renderà la tua barca più lenta senza
ridurre significativamente le forze di sbandamento. Infine, quando ti prepari ad
abbattere, rilascia un po' il \vang. Questo permetterà al boma di alzarsi e
manterrà la barca sotto controllo e non sovraccaricata mentre concludi la
manovra. Quando ti sarai sistemato di nuovo, cazza il \vang (il giusto) e
riparti. Fino a quando non viene issato lo spinnaker, la regolazione delle
scotte della randa e del fiocco sono gli aspetti più importanti durante la
navigazione in andature portanti. Entrambe le vele dovrebbero essere tese in
modo che lascandole leggermente si osservi uno ``sventolamento'' o una
contropancia (tecnicamente fintanto che le vele non \emph{fileggino}). Una buona
regolazione della randa richiede che la vela venga occasionalmente lascata un
po' e poi ricazzata fino a quando le contropancie scompaiono.

A differenza della navigazione risalendo il vento, non è sufficiente regolare e
bloccare le vele mentre si naviga in tali andature. Il 470 mostrerà cambiamenti
abbastanza significativi nella velocità di discesa del vento dovuti a raffiche,
onde, planate e lievi cambiamenti di rotta. Questi cambiamenti causano
spostamenti del vento apparente che devono essere compensati da cambiamenti
nella regolazione della vela.

Il vento apparente è il vento che si sente in faccia e rispetto al quale le vele
sono regolate ed è influenzato sia dal vento reale, come misurato a terra, sia
dal vento causato dal movimento della barca. Questo è esattamente lo stesso
vento che si sentirebbe se si fosse su un motoscafo che si muove a 10 mph.
Questo vento viene quindi sommato al vento reale e produce il vento apparente
rispetto al quale devi regolare le vele.

I cambiamenti sia nella velocità che nella direzione del vento, così come nella
velocità e nella direzione della barca, causano cambiamenti nel vento apparente.
Ad esempio, in una barca a vela di lasco, quando una raffica colpisce per la
prima volta, sembra che il vento si sposti verso poppa a causa dell'aumento di
forza del vento, quindi le vele vengono leggermente lascate. Nel momento in cui
la barca accelera, le vele devono essere ricazzate per compensare l'effetto
aumentato della velocità della barca.

Sono necessari molto più movimento del corpo e del timone nelle andature di
lasco e traverso rispetto alle boline. Per quanto possa sembrare sciocco, c'è un
vecchio detto che consiglia di ``tenere la barca sotto l'albero'' quando navighi
di lasco. Oltre al suo significato ovvio, questo consiglio fornisce tutte le
informazioni di cui avrai bisogno per essere un abile navigatore in questa
andatura. Se la punta dell'albero va a sottovento, come quando arriva una
raffica, porta la barca a sottovento fino a quando la punta dell'albero non è
proprio sopra la barca. Allo stesso modo, quando la barca si inclina a
sopravento in una calma, porta la barca a sopravento e mettila sotto l'albero.
Questa tecnica, se fatta in modo fluido, non solo è veloce ma rende anche la
barca più comoda e controllabile riducendo al minimo le possibilità di scuffia.

Una buona parte dei paragrafi precedenti è stata dedicata a descrivere come
condurre la barca sfruttando il peso e le vele. L'utilizzo di tali tecniche
diventa ancora più importante quando si naviga discendendo il vento poiché i
cambiamenti di rotta tendono ad essere più netti e repentini. Quando una raffica
colpisce, spostati duramente per far stare la barca piatta, o anche leggermente
inclinata a sopravento, lasca la randa e poi poggia leggermente col timone. Le
orzate sono meglio gestite permettendo alla barca di inclinarsi leggermente
(sottovento) e accompagnando il movimento del timone cazzando un po' le vele. In
condizioni di vento forte, non mantenere ``la barca sotto l'albero'', non usare
vele e non sfruttare lo sbandamento per poggiare durande raffiche sono le
principali cause di scuffia. Come accennato in precedenza, navigare senza il
timone è un ottimo esercizio per sviluppare un buon feeling con la barca.

\subsubsection{Lo spinnaker}
\label{sec:spinnaker}
Come probabilmente sai, lo spinnaker è una vela ausiliaria utilizzata
per la navigazione di lasco. L'aggiunta di uno spinnaker nel design di una barca
permette alla barca di essere più veloce nel discendere il vento. Quando si
naviga di lasco e poppa, lo spinnaker praticamente raddoppia la superficie
velica. Questa specializzazione permette alla randa e al fiocco di essere
progettati in modo più efficiente per la navigazione di bolina. Inoltre, lo
spinnaker dà al prodiere un importante compito nelle tratte di lasco in regata.
Questa responsabilità aggiuntiva e il coordinamento necessario durante l'issata,
la strambata e l'ammainata dello spinnaker, richiedono che la barca sia
condotta nella perfetta armonia dell'equipaggio.

L'utilizzo di uno spinnaker aumenta notevolmente il numero di cime con cui
bisogna fare i conti. La prima è la drizza dello spinnaker. Un'estremità è
attaccata alla penna dello spinnaker e l'altra è rimandata al timoniere. Come
tutte le drizze, la drizza dello spinnaker è utilizzata per issare la vela. A
differenza di altre vele, lo spinnaker è usato solo durante il lasco e deve
essere issato e ammainato più volte durante una giornata di navigazione.
Un'aggiunta utile alla drizza dello spinnaker è il sistema di paranco inverso.
Tirando un metro di drizza si issa la vela di tre metri, anche se ciò richiede
tre volte la forza. Oltre ad accelerare l'issata, l'utilizzo di un paranco
inverso lascia solo un terzo della cima in giro nel pozzetto quando lo spinnaker
è issato. Lo spinnaker ha anche una scotta che, come sempre, è utilizzata per
regolare la vela. Tuttavia, lo spinnaker è una vela simmetrica e, a prima vista,
sembra avere due scotte. Una volta issata la vela, questo problema scompare:
solo una cima agisce come scotta mentre l'altra è utilizzata per posizionare il
tangone dello spinnaker ed è chiamata \emph{braccio}. Il tangone gestisce il
posizionamento del piede dello spinnaker ed è sempre montato sul lato opposto
del boma. Nota che quando strambi, il tangone deve essere spostato da un lato
all'altro e il vecchio braccio diventa la nuova scotta e viceversa.

Il tangone dello spinnaker ha tre punti di attacco. Un'estremità è agganciata al
braccio e l'altra estremità è attaccata all'anello dello spinnaker sull'albero.
L'occhiello al centro del tangone è attaccato a due cime che prendono il nome di
\emph{carica alto} e \emph{carica basso}. Di queste solitamente, una è tenuta in
tensione da un circuito elastico mentre l'altra è un effettivo controllo
utilizzato per gestire l'altezza del tangone. Quando non è in uso, lo spinnaker
è riposto, armato e pronto all'uso in una delle sacche vicino all'albero.

Per riporre lo spinnaker, individua entrambe le \emph {bugne} (punti di attacco
di scotta e braccio) mentre la vela è ancora nella sua sacca di stivaggio e
assicurati di poter passare da un lato all'altro senza che ci siano nodi nel
bordo inferiore. Infila il bordo inferiore nella sacca dello spinnaker ma lascia
le due bugne fuori. Continua a infilare la vela assicurandoti che le strisce
rinforzate non presentino torsioni o nodi. Quando leghi la drizza, guarda in
alto per assicurarti che non sia aggrovigliata e che essa scorra sullo stesso
lato dello strallo. Infine, ricorda che una delle scotte deve essere fatta
passare attorno allo strallo prima di essere attaccata allo spinnaker.
Probabilmente ti accorgerai che riporre lo spinnaker prima di armare il fiocco
ridurrà alcuni dei problemi di aggrovigliamento. Ricorda, lo spinnaker deve
essere armato all'esterno del fiocco.

Ci sono due tecniche diverse necessarie per issare lo spinnaker in tutte le
condizioni. L'uso di uno o dalla sacca in cui è riposto lo spinnaker prima
dell'issata. Se la vela è nella sacca di sinistra mentre il vento arriva da
dritta, lo spinnaker, una volta issato, sarà ``all'ombra'' (si intende il vento
colpisce prima le altre vele) delle vele bianche (randa e fiocco). In questa
condizione, appena issata, la vela si gonfiera senza problemi senza trovare
ostacoli. In tal caso si parla di \emph{issata normale} o \emph{a \leeward}.
%
Se lo spinnaker è nella sacca a dritta, si possono invece riscontrare più
problemi. Il possibile rischio consiste in una incontrollata e repentina issata
della vela che, in presenza di raffiche, può portare la barca ad essere
``tirata'' alla straorza. La tecnica necessaria in questo caso è chiamata
\emph{a sopravento} poiché lo spinnaker sarà a monte delle vele bianche prima di
essere issato. L'issata a sottovento, essendo più facile e meno incline a
problemi, è il primo tipo di issata dello spinnaker che dovresti provare. Se
ti trovi a navigare e ti serve un'issata a sopravento, stramba per posizionare
lo spinnaker nella giusta posizione.

Per issare lo spinnaker, poggia per raggiungere un lasco e lasca il \vang e la
randa. Il fiocco può essere lasciato lasco ma spesso è utile cazzarlo un po' più
del necessario in modo che non interferisca con la manovra. Durante tutta
l'issata, il prodiere sarà occupato al centro della barca e dunque spetta al
timoniere bilanciare la barca e mantenerla in moto.

Prima di issare, il prodiere dovrebbe assicurarsi che la drizza sia libera di
salire. Alcuni velisti preferiscono far passare la drizza attorno al \chainplate
per evitare si annodi quando non è in uso. Questo richiede al prodiere di andare
a \leeward e sganciare la drizza prima di ripetere la manovra. In ogni caso, è
spesso una buona idea tirare la testa della vela di un metro o due fuori dalla
sacca per aiutarla a salire.

Se la tua barca ha i \twinglines dello spinnaker, assicurati che quello
\windward sia strozzato e quello \leeward libero. Il prodiere dovrebbe
assicurarsi che la \sheet dello spinnaker sia correttamente bloccata e collegata
al \guy. Sfila il \pole dal pozzetto e blocca un estremità al \guy (tramite una
delle apposite varee). Assicurati che il \pole sia posizionato davanti alla
sartia. Inizia a spingere in fuori il \pole fintanto che non raggiungi
l'aggancio del carica alto-basso, a questo punto puoi agganciarlo. Spingi del
tutto fuori il \pole ed aggancia la seconda varea all'anello dello spinnaker.

Nel momento in cui il \pole è armato, il prodiere deve gridare ``issa'' al
timoniere ed essere pronto a bilanciare la barca quando il timoniere si
appresterà ad afferrare la drizza. Mentre il timoniere issa la vela, il
prodiere cazza il braccio fintanto che la brugna dello spinnaker incontra la
varea del \pole. Il \pole è regolato in modo che sia allineato al boma, a questo
punto può essere fissato. Strozzare il \guy può richiedere di voltarsi a causa
del posizionamento dello strozzatore vicino al carrello . Altre volte lo
strozzatore è invece posizionato vicino alle sartie. Infine, allunga il braccio
per afferrare la scotta. Un buon timoniere può liberare e regolare la scotta
dello spinnaker fintanto che il prodiere è impegnato a regolare il braccio.

Nota: Durante tutta l'issata, tutti e tre gli angoli della vela sono tensionati.
La penna è tirata dalla drizza, le mura dal braccio e la scotta è bloccata per
mantere la bugna vicina. Questo permette di ridurre il rischio di
``incaramellamento'' (torsione) della vela. La scotta viene sbloccata solamente
nel momento in cui la vela è completamente issata e sovra-cazzata. Sii pronto ad
a raddoppiare la superficie velica in circa 2 secondi. Quando questo avviene, il
timoniere deve essere in grado di poggiare e il prodiere deve bilanciare la
barca con forza.

l'issata \windward è più complicata poichè lo spinnaker potrebbe essere spinto
contro il fiocco e incastrarsi. Durante un'issata a \leeward, la vela verrà
spinta via dalla barca senza problemi. Prima di ogni issata, il prodiere è
tenuto a verificare la vela e a informare il timoniere sul tipo di issata.
Quando ci si prepara a un'issata \windward, il prodiere deve sbloccare sia il
\guy che la \sheet, e dare uno ``strattone'' al braccio per ottenere un po' di
lasco nella cima. Senza questo margine, la vela non riuscirà a passare intorno
al fiocco abbastanza velocemente.

Successivamente, il \pole è attaccato al \guy ma non è spinto fuori. Lo
spinnaker è interamente tirato fuori dalla sacca e dopo aver gridato ``vai'', è
lanciato dal prodiere verso lo strallo. Il timoniere deve a questo punto, issare
la vela il più velocemente possibile. Il lancio deve essere sufficientemente
deciso da far passare lo spinnaker al lato sottovento della barca, altrimenti
dovrai staccarlo dal fiocco e ripetere il lancio. Quando lo spinnaker è libero
dallo strallo, il \pole può essere complettamente attaccato e regolato
normalmente.

È meglio provare questo tipo di issata in andature di gran lasco e poppa. Queste sono
solitamente tranquille perchè la vela tenderà a gonfiarsi di fronte al fiocco e
a non andargli addosso. Le issate più difficili sono in giornate ventose con
andature di lasco stretto, quasi traverso. Lo spinnaker, come le altre vele, è
regolato con la \sheet. La \sheet deve essere lascata fintanto che la vela non
inizi a creare una contropancia dal lato del \pole.

Attenzione: Gli spinnaker sono vele insidiose che collasseranno completamente se
distoglierai lo sguardo da esse anche solo per un secondo. Se lo spinnaker si
affloscia, regola rapidamente la scotta finché la vela non si riempie di nuovo,
poi allenta di nuovo la scotta.

Il tangone viene posizionato all'incirca perpendicolare al vento. Con
un'andatura al gran lasco o poppa, il tangone dovrebbe essere posizionato circa
a due terzi della possibile corsa verso poppa, mentre con un'andatura al
traverso può essere allentato il più possibile in avanti senza appoggiarsi allo
strallo di prua. La posizione del tangone viene regolata con il braccio, che può
essere fissato vicino al carrello (quadrato) o vicino allo strallo di prua
(strallato).
%
Se quadrato, l'equipaggio sarà costretto a distogliere lo sguardo dalla vela
mentre regola il tangone. Utilizzando strozzatori anteriori, il \guy può
rimanere regolato mentre si gioca con la scotta. L'altezza del \pole è regolata
con il carica alto. Essa controlla anche la forma della vela e la larghezza
della ``fessura'' tra lo spinnaker e le vele bianche.

Le regole della classe 470 richiedono un \pole per spinnaker che sia più
piccolo dell'ottimale. A causa di ciò, è comune che il \pole si allontani dalla
brugna della vela e sempre per tale ragione, deve spesso essere regolato più
alto che in altre classi. In generale, il \pole non deve mai essere più in basso
della perpendicolare con l'albero e può essere sollevato sino alla
perpendicolare con lo strallo. Solitamente, il \pole dovrebbe essere più basso
in condizione di vento leggero e in andature molto larghe, mentre dovrebbe
essere più alto con andature più strette e vento forte.
%
Una regola pratica per regolare l'altezza del \pole consiste nel posizionare
quest'ultimo ad un'altezza tale per cui \theluff (il lato dello spinnaker tra la
penna e il \guy) inizi a creare la contropancia esattamente a metà tra la
penna e la brugna del \guy.

Il caricabasso del 470 è costituito da una corda elastica che non è abbastanza
forte da prevenire che \thepole di alzarsi durante una raffica. Tuttavia,
tenendo \theguy basso, \thepole si riesce a mantenere sotto controllo. Ciò può
essere fatto o tramite dei rinvii per \theguy posizionati sulla coperta, che con
un sistema di \twinglines. Il \guy \textbf{dovrebbe} essere bloccato (strozzato) se si
naviga dal traverso al gran lasco e \textbf{può} essere stozzato con vento di
poppa ad eccezione di condizioni di vento molto debole. Se la tua barca prevede
i \twinglines, assicurati che quello \leeward sia sempre bloccato. Tirare in
basso il punto di scotta ha il solo effetto di soffocare lo spinnaker.

Lo spinnaker del 470 può inoltre essere ``pompato'' per ottenere un incremento
temporaneo di potenza, utile per iniziare un planata o per surfare su un'onda.
Il pompaggio funziona al meglio con venti forti, di poppa o gran lasco, con onde
formate. La vela viene pompata tirando simultaneamente e rapidamente \guy e
\sheet, lasciandoli poi tornare alla loro posizione originaria. Allo stesso
tempo, il timoniere tirerà energicamante la randa e manterrà la barca piatta (la
randa può essere tirata o cazzando energicamente la \sheet o afferrando
direttamente il circuito). Questa tecnica richiede molta pratica per essere
eseguita correttamente. È richiesto un ottimo feeling per capire quando la barca
è pronta per essere pompata. Un buon modo per avere un'idea generale di cosa sta
succedendo è uscire in barca con qualcuno che conosce bene la barca ed è
abituato a percepirne lo stato.

Quando si naviga discendendo il vento, specialmente in regata, il compito del
prodiere è tenere lo spinnaker a segno. Il timoniere è responabile della tattica
e del bilanciamento della barca. Questo è essenzialmente l'inverso di ciò che
avviene durante il lato di bolina, nel quale il timoniere si concentra sulla
velocità della barca e il prodiere decide le manovre e mantiene la barca piatta.
Al lasco, il prodiere dovrebbe sedere \windward per avere una visione completa
dello spi. Il timoniere solitamente siede \leeward in modo da vedere bene il
vento in arrivo ed eventuali raffiche. Alcuni equipaggi lasciano il timoniere
sedueto \windward e il prodiere a certrobarca. Questo compromette la visuale sia
del timoniere che del prodiere e concentra il peso al centro, rendendo la barca
più instabile.

Il massimo dell'adrenalina navigando in 470 è uscire al trapezio con lo spinnaker
alzato. Sii pronto a finire in acqua le prime volte che proverai. Il timoniere
deve essere pronto a saltare in giro per la barca per tenere ``le cose'' sotto
controllo. Il prodiere deve essere pronto a passare dall'essere completamente
disteso al trapezio all'essere seduto in barca, all'essere di nuovo
completamente fuori, tutto mantenendo lo spinnaker regolato. Il passaggio più
complicato è passare dall'essere in barca all'uscire al trapezio. A differenza
di quando si naviga risalendo il vento, dovrai impegnarti a mantenere la vela a
segno mentre esci al cavo e una mano sarà impegnata a tenere la \sheet. Questo
ti impedirà di usare la maniglia del trapezio. Tenere l'ascensore completamente
cazzato, ti renderà più facile lo spostamento. Scoprirai inoltre, che sarà
necessario tenere le gambe più aperte poichè le andature sono più
movimentate rispetto alla bolina. Infine ricorda, mentre starai cercando di
uscire, la \sheet dello spinnaker ti tirerà dentro\dots Buona fortuna!

Quando si plana con lo spinnaker dal trapezio, sono necessari molti continui
aggiustamenti e regolazioni. Quando una raffica colpisce le vele, il vento
apparente si sposta verso poppa e lo spinnaker risulta sovra-cazzato. Inoltre,
il timoniere deve essere in grado di orzare per riportare la barca sotto
controllo. Se lo spinnaker non viene rilasciato adeguatamente, le forze in gioco
impediscono al timoniere di poggiare.

Dunque, appena percepisci una raffica, lasca un po' la \sheet dello spinnaker e
stenditi al trapezio per mantenere la barca piatta. Cerca di non lascare
troppo la \sheet poichè questo potrebbe portare al collasso della vela e ad un
conseguente brusco sbandamento verso il vento. Quando la barca si appiattisce e
inizia a planare, lo spinnaker dovrà essere nuovamente regolato. Tutto questo
può richiedere molto lavoro e non sarebbe strano se ti trovassi senza fiato dopo
alcuni laschi veloci fatti in questo modo. Abbattere in 470 è solitamente facile
nella maggior parte delle condizioni. Le ``abbattute volanti'' sono la norma. Il
timoniere afferra il circuito \sheet randa o il prodiere il circuito \vang e
passano il boma da una parte all'altra. Abbattere in condizioni di vento forte
con lo spinnaker diventa un po' impegnativo a causa della velocità della
manovra. Sii pronto a muovere velocemente il peso per mantenere la barca
stabile. Tieni il controllo del timone e non permettere alla barca di orzare
troppo dopo l'abbattuta.

Esistono due modi per abbattere con lo spinnaker issato. Il primo, l'abbattuta
normale che è usata per abbattere di poppa o con brezze leggere. Il secondo,
l'abbattuta da lasco a lasco. L'abbattuta da lasco a lasco è una tecnica per
abbattere velocemente passando da un trapezio all'altro.
%
Per l'abbattuta normale, il timoniere informa il prodiere dell'intenzione di
eseguire la manovra e prende il controllo della \sheet dello spinnaker. Le
\sheets dello spinnaker sul 470, attraversano il pozzetto della barca. Il
timoniere è in grado di raggiungere la \sheet senza troppo sforzo. Il timoniere
non passa l'estremo della \sheet indietro, oltre il carrello della randa per non
causare inutile confusione. Nel momento in cui il timoniere ha la \sheet,
prenderà \theguy analogamente. Il timoniere può sedersi a cavalcioni del timone
e governare la barca. Nel momento in cui il prodiere fa abbattere la randa dal
circuito vang,il timoniere porta lo spinnaker sulle altre mura e lo gestisce
fintanto che il prodiere non ha fatto abbattere \thepole.
%
Questo è fatto sganciando \thepole dall'albero, agganciandolo al nuovo \guy,
sganciandolo dal vecchio \guy e riagganciandolo all'anello spinnaker sull'albero
(dall'altra varea). Il \guy viene dunque bloccato, lo spinnaker si rigonfia e i
\twinglines vengono reimpostati. I due aspetti più impegnativi dell'abbattuta
sono coordinare il posizionamento del peso di timoniere e prodiere mentre si
tiene lo spinnaker gonfio. È utile che il prodiere, nel riattaccare \thepole
all'anello sull'albero, lo spinga fuori in avanti quanto più vicino alla giusta
altezza. Questo aiuta il timoniere a mantere lo spinnaker in volo durante tutta
l'abbattuta.

Un buon esercizio consiste nel navigare con il vento in poppa e strambare ogni
10 lunghezze di barca. Inoltre, prova a far volare lo spinnaker senza il tangone
mentre il timoniere sta in piedi, governa e gestisce sia \theguy che la
\sheet, effettuando strambate da una parte e dall'altra.

La strambata da lasco a lasco è più impegnativa e solitamente viene usata
solamente una volta durante una ragata (in corrispondenza della boa sottovento).
Se eseguita correttamente tuttavia, può portare a superare 3 o 4 barche in boa.
Va detto che questo tipo di strambata richiede molta pratica, ma vale
decisamente lo sforzo.

La strambata da lasco a lasco inizia con timoniere e prodiere entrambi
completamente appesi fuori bordo e la barca in planata su un'andatura di lasco
stretto. Il timoniere prende la scotta dal prodiere, si appende con forza e
poggia leggermente per permettere al prodiere di rientrare. Mentre il prodiere
rientra, sgancia il trapezio, fa uscire il fiocco dal suo strozzascotte e, se
possibile, tira il \twingline \leeward il più possibile con un'unica tirata. A
questo punto, il timoniere lasca la \sheet, lasciando lo spinnaker sventolare,
poggia ancora e abbatte la randa lanciando il boma dall'altra parte della barca
utilizzando il circuito della randa. Mentre la barca abbatte, il
prodiere sgancia il \twingline del ``vecchio'' \windward e tira lo
spinnaker attorno al fiocco cazzando il vecchio \guy. Una volta che la randa
ha abbattuto, il timoniere può iniziare a cazzare la \sheet della randa e
raggiungere dall'altra parte per afferrare la \sheet dello spinnaker.
%
Nel frattempo, il prodiere si sposta sul lato \windward, cazzando completamente
il nuovo \twingline \windward e bloccando \theguy nella strozzascotte vicino
alle sartie, in un punto predeterminato che manterrà il tangone appena fuori
dallo strallo di prua (questo punto può essere cercato con barca a terra e
segnato per mettere più velocemente a segno \thepole). Infine, il tangone viene
spostato sul lato corretto. Non appena il timoniere vede il tangone agganciato
all'anello sull'albero, cazza la \sheet dello spinnaker. Il prodiere afferra la
maniglia, esce fuori al trapezio, si aggancia e riprende la \sheet dello
spinnaker dal timoniere. Una volta stabilizzati, timoniere e prodiere tirano
entrambi un breve sospiro di sollievo, si mettono appesi fuori bordo, il
timoniere regola correttamente la randa, il prodiere regola di nuovo il fiocco e
la barca accelera. Tutto qui!

L'ammainata è semplicemente l'opposto dell'issata, ma più facile. Il timoniere
prende la \sheet dal prodiere e la regola mentre il prodiere sgancia \theguy.
Nel momento in cui \thepole viene sganciato dall'albero, la \sheet può essere
rilasciata. Così come \thepole viene completamente sganciato e portato a bordo,
il prodiere afferra \theguy. La base della vela viene dunque raccolta facendo
collassare lo spinnaker. Il prodiere a questo punto urla al timoniere di
ammainare la vela. Mentre il prodiere ripone lo spinnaker nella sacca, il
timoniere dovrebbe controllare la discesa della vela per evitare sia troppo
veloce.

Nota che lo spinnaker è sempre raccolto \windward. In condizioni di vento forte,
lo spinnaker potrebbe risultare più maneggevole cazzando completamente entrambi
i \twinglines prima dell'ammainata.

Un utile consiglio per le regate è preparare la barca per la prossima bolina
prima di ammainare lo spinnaker. Ad esempio, abbassare la deriva, tirare il
\cunningham, ritensionare il \outhaul, giusto per nominare alcune cose. Così,
dopo l'ammainata, puoi concentrarti completamente a fare una buona
virata e mettere la barca al vento, mentre i tuoi concorrenti sono ancora
impegnati a trafficare con le loro cime.