% !TEX root = ../../main.tex

\subsection{Conduzione con andature portanti}
\label{sec:conduzione_andature_portanti}
Un 470 si comporta in modo molto diverso in andature portanti (dal lasco alla poppa)
rispetto a quando è a bolina. Risalendo il vento, la barca richiede uno sforzo
costante e continua, piccoli aggiustamenti della vela e pochi movimenti
del corpo. La barca è una piattaforma abbastanza stabile che tende costantemente
a tirare nella direzione del vento. ``Scendendo il vento'', il 470 tende ad essere molto più instabile.
Condurre la barca richiede molto più movimento e cambiamenti drastici della
vela.

Le regolazioni discendendo il vento sono abbastanza semplici. Nella maggior parte
dei casi, le cime di controllo possono essere lascate. Il cunningham è facile
da rilasciare mentre il tesabase è spesso lasciato invariato a meno che tu non sia in
una regata lunga o abbia un meccanismo che consente un rilascio rapido. La cosa importante da
ricordare è la regolazione del vang. Se avevi il vang cazzato durante la
navigazione di
bolina e inizi a poggiare verso un traverso o lasco, assicurati di lascarlo
completamente. Se lasciato troppo teso, il vang rallenterà la barca, peggiorerà significativamente la
manovrabilità, non permetterà alla vela di essere rapidamente rilasciata nelle
raffiche e, tirando il boma verso il basso, può causare dolorose strambate
(abbattute involontarie).

Durante la navigazione di lasco, il vang dovrebbe essere utilizzato come un
``acceleratore''. Un vang più teso porta a
meno svergolamento e più potenza sino a quando la vela non risulti troppo
piatta. Di lasco, la randa dovrebbe prendere vento in modo uniforme, né prima in
alto né prima in basso, e le stecche dovrebbero essere allineate se si ha la
giusta tensione del vang. Al traverso e al lasco, un vang risulta sovraccaricato renderà la tua barca più
lenta senza ridurre significativamente le forze di sbandamento. Infine, quando
ti prepari ad abbattere, rilascia un po' il vang. Questo permetterà al boma di
alzarsi e manterrà la barca sotto controllo e non sovraccaricata mentre concludi
la manovra. Quando ti sarai sistemato di nuovo, cazza il vang (il giusto) e
riparti. Fino a quando non viene issato lo spinnaker, la regolazione delle
scotte della randa e del fiocco è la considerazione più importante quando si è
in andature portanti. Entrambe le vele dovrebbero essere tese in modo che
lascandole leggermente si osservi uno ``sventolamento'' o una contropancia
(tecnicamente fintanto che le vele non \emph{fileggino}). Una
buona regolazione della randa richiede che la vela venga occasionalmente
lascata un po' e poi ricazzata fino a quando le contropancie scompaiono.

A differenza della navigazione risalendo il vento, non è sufficiente regolare e bloccare le
vele mentre si naviga in tali andature. Il 470 mostrerà cambiamenti abbastanza
significativi nella velocità di discesa del vento dovuti a raffiche, onde,
planate e lievi cambiamenti di rotta. Questi cambiamenti causano spostamenti del
vento apparente che devono essere compensati da cambiamenti nella regolazione
della vela.

Il vento apparente è il vento che si sente in faccia e rispetto al quale le vele sono
regolate ed è influenzato sia dal vento reale, come misurato a terra, sia dal
vento causato dal movimento della barca. Questo è esattamente lo stesso vento
che si sentirebbe se si fosse su un motoscafo che si muove a 10 mph. Questo
vento viene quindi sommato al vento reale e produce il vento apparente rispetto al quale
devi regolare le vele.

I cambiamenti sia nella velocità che nella direzione del vento, così come nella
velocità e nella direzione della barca, causano cambiamenti nel vento apparente.
Ad esempio, in una barca a vela di lasco, quando una raffica colpisce per la
prima volta, sembra che il vento si sposti verso poppa a causa dell'aumento di
forza del vento, quindi le vele vengono leggermente lascate. Nel momento in cui
la barca accelera, le vele devono essere ricazzate per compensare l'effetto
aumentato della velocità della barca.

Sono necessari molto più movimento del corpo e del timone sulle andature di
lasco e traverso rispetto alle boline. Per quanto possa sembrare sciocco, c'è un vecchio
detto che consiglia di ``tenere la barca sotto l'albero'' quando navighi di
lasco. Oltre al suo significato ovvio, questo consiglio fornisce tutte le
informazioni di cui avrai bisogno per essere un abile navigatore in questa andatura. Se la
punta dell'albero va a sottovento, come quando arriva una raffica, porta la
barca a sottovento fino a quando la punta dell'albero non è proprio sopra la
barca. Allo stesso modo, quando la barca si inclina a sopravento in una calma,
porta la barca a sopravento e mettila sotto l'albero. Questa tecnica, se fatta
in modo fluido, non solo è veloce ma rende anche la barca più comoda e
controllabile riducendo al minimo le possibilità di scuffia.

Una buona parte dei paragrafi precedenti è stata dedicata a descrivere come
condurre la barca sfruttando il peso e le vele. L'utilizzo di tali tecniche
diventa ancora più importante quando si naviga discendendo il vento poiché i cambiamenti di rotta
tendono ad essere più netti e repentini. Quando una
raffica colpisce, spostati duramente per far stare la barca piatta, o anche
leggermente inclinata a sopravento, lasca la randa e poi poggia leggermente col
timone. Le orzate sono meglio gestite permettendo alla
barca di inclinarsi leggermente (sottovento) e accompagnando il movimento del
timone cazzando un po' le vele. In condizioni di vento forte, non mantenere ``la barca
sotto l'albero'', non usare vele e non sfruttare lo sbandamento per poggiare durande raffiche
sono le principali cause di scuffia. Come accennato in precedenza,
navigare senza il timone è un ottimo esercizio per sbiluppare un buon feeling
con la barca.

\subsubsection{Lo spinnaker}
\label{sec:spinnaker}
Come probabilmente sai, lo spinnaker è una vela ausiliaria utilizzata
per la navigazione di lasco. L'aggiunta di uno spinnaker nel design di una barca
permette alla barca di essere più veloce nel discendere il vento. Quando si
naviga di lasco e poppa, lo spinnaker praticamente raddoppia la superficie
velica. Questa specializzazione permette alla randa e al fiocco di essere
progettati in modo più efficiente per la navigazione di bolina. Inoltre, lo
spinnaker dà al prodiere un importante compito nelle tratte di lasco in regata.
Questa responsabilità aggiuntiva e il coordinamento necessario durante l'issata,
la strambata e l'ammainata dello spinnaker, richiedono che la barca sia
condotta nella perfetta armonia dell'equipaggio.

L'utilizzo di uno spinnaker aumenta notevolmente il numero di cime con cui
bisogna fare i conti. La prima è la drizza dello spinnaker. Un'estremità è attaccata
alla penna dello spinnaker e l'altra è rimandata al timoniere. Come tutte le
drizze, la drizza dello spinnaker è utilizzata per issare la vela. A differenza
di altre vele, lo spinnaker è usato solo durante il lasco e deve essere issato e
ammainato più volte durante una giornata di navigazione. Un'aggiunta utile alla
drizza dello spinnaker è il sistema di paranco inverso. Tirando un metro di
drizza si issa la vela di tre metri, anche se ciò richiede tre volte la forza.
Oltre ad accelerare l'issata, l'utilizzo di un paranco inverso lascia solo un
terzo della cima in giro nel pozzetto quando lo spinnaker è issato. Lo spinnaker
ha anche una scotta che, come sempre, è utilizzata per regolare la vela.
Tuttavia, lo spinnaker è una vela simmetrica e, a prima vista, sembra avere due
scotte. Una volta issata la vela, questo problema scompare: solo una cima agisce
come scotta mentre l'altra è utilizzata per posizionare il tangone dello
spinnaker ed è chiamata \emph{braccio}. Il tangone gestisce il posizionamento del piede dello
spinnaker ed è sempre montato sul lato opposto del boma. Nota che quando
strambi, il tangone deve essere spostato da un lato all'altro e il vecchio braccio
diventa la nuova scotta e viceversa.

Il tangone dello spinnaker ha tre punti di attacco. Un'estremità è agganciata al
braccio e l'altra estremità è attaccata all'anello dello spinnaker sull'albero.
L'occhiello al centro del tangone è attaccato a due cime che prendono il nome di
\emph{carica alto} e \emph{carica basso}. Di queste solitamente una è tenuta in
tensione da un circuito elastico mentre l'altra è un effettivo controllo utilizzato per gestire
l'altezza del tangone. Quando non è in uso, lo spinnaker è riposto, armato e
pronto all'uso in una delle sacche vicino all'albero.

Per riporre lo spinnaker, individua entrambe le \emph {bugne} (punti di attacco
di scotta e braccio) mentre la vela è ancora nella sua sacca di stivaggio e
assicurati di poter passare da un lato all'altro senza che ci siano nodi nel
bordo inferiore. Infila il bordo inferiore nella sacca dello spinnaker ma lascia
le due bugne fuori. Continua a infilare la vela assicurandoti che le strisce
rinforzate non presentino torsioni o nodi. Quando leghi la drizza, guarda in
alto per assicurarti che non sia aggrovigliata e che essa scorra sullo stesso
lato dello strallo. Infine, ricorda che una delle scotte deve essere fatta passare
attorno allo strallo prima di essere attaccata allo spinnaker. Probabilmente ti
accorgerai che riporre lo spinnaker prima di armare il fiocco ridurrà alcuni dei
problemi di aggrovigliamento. Ricorda, lo spinnaker deve essere armato
all'esterno del fiocco.

Ci sono due tecniche diverse necessarie per issare lo spinnaker in tutte le
condizioni. L'uso di uno o dell'altro è dipendente dalla posizione dello
spinnaker riposto prima dell'issata. Se la vela è nella sacca di sinistra mentre


% As you probably know, the spinnaker, or chute, is an auxiliary sail used for off
% wind work. The addition of a spinnaker into a boat's design allows the boat to be
% faster down wind. When reaching and running, the spinnaker practically doubles
% the sail area. This specialization allows the main and jib to be designed most
% efficiently for sailing on the wind. In addition, the spinnaker gives the crew an
% important job on the off wind legs of a race. This added responsibility, and the
% coordination needed while raising, jibing and lowering the chute, demand that the
% boat be sailed as a team.
% The use of a spinnaker greatly increases the number of lines that must be
% contended with. First is the spinnaker halyard. One end is attached to the head of
% the spinnaker and the other end led to the skipper. Like all halyards, the
% spinnaker halyard is used to raise the sail. Unlike other sails however, the
% spinnaker is only flown off the wind and must be raised and lowered in the course
% of a day's sailing. A handy addition to the chute halyard is the reverse purchase.
% Pulling one foot of halyard raises the sail three feet, though it requires three times
% the strength. In addition to speeding up the hoist, use of a reverse purchase
% leaves only a third as much line kicking around in the cockpit when the chute is
% up. The spinnaker also has a sheet which, as always, is used to trim the sail.
% However, the spinnaker is a symmetrical sail and appears to have two sheets.
% Once the sail is raised this problem disappears. Only one line acts as a sheet
% while the other is used to position the spinnaker pole and is called the guy. The
% pole positions the foot of the spinnaker and is always set out on the side
% opposite the boom. Note that when you jibe, the pole must be switched from side
% to side and the old guy becomes the new sheet and vice versa.
% The spinnaker pole has three attachment points. One end is hooked to the guy
% and the other end is attached to the spinnaker ring on the mast. The eye at the
% center of the pole is attached to the topping lift, a line used to control the height
% at which the pole is flown. When not in use, the spinnaker is stowed, rigged and
% ready to go in one of the bags near the mast.
% To pack the chute, find both clews while the sail is still in its storage bag and
% make sure that you can run from one side to the other without any tangles in the
% foot. Stuff the foot into the spinnaker bag but leave both clews hanging out.
% continue to stuff the sail making sure that the luff tapes are not twisted. When
% tying on the halyard, look aloft to make sure that its not tangled and the halyard
% runs on the same side of the forestay as the sail. Finally, remember that one of
% the sheets must be led around the forestay before being attached to the
% spinnaker. You'll probably find that packing the chute before the jib is rigged will
% minimize some of the tangling problems. Remember, the spinnaker must be
% rigged outside the jib.
% There are two different techniques needed to raise the spinnaker under all
% conditions. Use of either is dependent on the position of the packed chute before
% the set. If the sail is in the port bag while the sail is on starboard tack, the
% spinnaker will be in the lee of the working sails and blow away from the boat
% when hoisted. This requires the normal or leeward set. If the chute is in the bag,
% disaster is possible. The result, a strong possibility of getting blown into the jib.
% The technique needed here is called a windward set as the spinnaker will be
% upwind of the working sails before being hoisted. The leeward set, being easier
% and less prone to failure, is the first type of spinnaker set you should try. If you
% are out day sailing and wind up needing a weather set, just jibe to get the chute in
% proper position.
% To hoist the spinnaker, bear off to a reach, ease the vang, and let the main out.
% The jib should be eased but may be left slightly over trimmed to keep it out of the
% way. The crew will have trouble in the middle of the boat during the set so it is up
% to the skipper to balance the boat in a race and keep the boat moving.
% Before hoisting, the crew should make sure that the halyard is free to run up.
% Some sailors like to run the halyard around the chain plate to keep it untangled
% when not in use. This requires the crew to go down to leeward, unhook the
% halyard, before the reset. In any case, it's often a good idea to pull the head of the
% sail a foot or two out of the pouch to help it up.
% If your boat has twing lines, make sure that the windward one is cleated and the
% leeward one free. The crew should make sure that the spinnaker sheet is cleated
% and onto the guy. Slide the pole from the bilge and hook to one end to the guy. Be
% sure that the pole is forward of the side stay. Push it out until the topping lift can
% be attached. Once this is done, the pole can be pushed all the way out and
% hooked to the spinnaker ring.
% Once the pole is set, the crew should yell "hoist" to the skipper, but be ready to
% balance the boat as the skipper comes in to get the halyard. As the skipper hoists
% the sail, the crew should pull the guy back and push the pole forward until it
% meets the corner of the sail. The pole is set by aligning it with the boom, Cleat the
% guy. This might require turning around to cleat near the traveler. Some boats
% have guy cleats near the shrouds. Finally ,reach across the boat and pick up the
% sheet. A good skipper can uncleat and trim the sheet as the crew is fooling with
% the guy.
% Note: Throughout the set, all three corners of the sail are under tension. The head
% is being pulled up by the halyard, the tack by the guy, and the sheet cleated to
% keep the clew in. This cuts down the risk of twisting the sail The sheet is
% nucleated as the sail goes up full and over trimmed. Be prepared to have your sail
% area doubled in about two seconds. When this happens, the skipper must be able
% to bear off and the crew needs to hike hard.
% A windward set is trickier because the chute can be blown into the jib. During a
% leeward hoist, the sail will harmlessly be blown away from the boat. Before any
% spinnaker set, the crew first checks the sail and informs the skipper of the type of
% set When preparing for a windward set, the crew should uncleat both the guy and
% the sheet, and give the guy a yank to get some slack in the line. Without this
% slack, the chute will not be able to get around the jib fast enough.
% Next, the pole is attached to the guy but is not put out any further. The entire
% spinnaker is pulled out of the bag and after yelling "go", is tossed up towards the
% fore stay. The skipper raises the sail as fast as possible. The throw must be good
% enough to get the spinnaker all the way around to the lee side of the boat or you'll
% have to peel the sail off the jib and toss it again. Once the spinnaker is clear of
% the forestay, the pole can be fully attached to the topping lift and spinnaker ring
% and trimmed in a normal fashion.
% Its best to try windward sets on low reaches and runs first. These are usually
% quite tame and easy to do as the spinnaker will tend to blow out in front of the
% sail and not into the jib. The hardest weather sets are on windy days on high
% reaches. The spinnaker, like all other sails, is trimmed with the sheet. The sheet is
% eased until the luff , the edge on the same side as the pole, just begins to curl.
% Beware: Spinnakers are tricky beasts and will completely collapse if you take
% your eye off them for even a second. Should the chute collapse, rapidly trim in
% until the sail fills, then ease the sheet back out. The pole is set roughly
% perpendicular to the wind. On a broad reach or run, the pole should be about two-
% thirds of the way back, while on a close reach, it can be eased as far forward as
% possible without resting on the forestay. Spinnaker pole position is adjusted with
% the guy, which may be cleated either near the traveler, or near the forestay.
% If cleated near the traveler, the crew will be forced to look away from the sail while
% adjusting the pole. Using the forward cleats allows the guy to be adjusted while
% the sheet is being played. The pole height is adjusted with the topping lift. It also
% controls the shape of the sail and the width of the slot between the spinnaker and
% working sails. The rules of the 470 class require a spinnaker pole that is smaller
% than optimal.
% Because of this, the lost is frequently choked off and the pole must be flown
% higher than is common on many other classes of boats. The pole should never be
% lower than perpendicular to the mast and can be carried as high as perpendicular
% to the stay. Generally, the pole should be lowest in light air and on broad reaches
% and runs, and highest on close reaches in heavy air.
% A rule of thumb for optimum pole height is the height at which the luff of the
% spinnaker breaks or curls exactly midway between the head and tack of the sail.
% The topping lift down haul on a 470 is made of shock cord which is not strong
% enough to prevent the pole from "skying" in a gust. However, by holding the guy
% down, the pole can be kept under control. This is done either with small hooks on
% the chain plate or with a line and cleat system known twings. The guy should be
% pinned down whenever the boat is reaching and can be pinned on runs except in
% very light air. If your boat has twings, be sure the leeward twing is cleated at all
% times. Keeping the sheet down only chokes the spinnaker.
% A 470 spinnaker can be pumped to get a surge of power to pop the boat on a
% plane or surf. Pumping works best on heavy air runs or broad reaches when the
% waves are up. The chute is pumped by simultaneously and rapid tugging on the
% guy and sheet and then releasing them to their normal positions. At the same
% time, the skipper quickly trims the main and hikes to keep the boat flat. This other
% technique requires lots of practice to be effective. One needs a good feel to know
% when the boat is ready for a pump. A good way to get a general notion of what's
% going on is to go for a sail with someone who knows the feel of the boat.
% When sailing off the wind, especially in a race, the crew's job is to keep the
% spinnaker trimmed. The skipper is responsible for tactics and boat balance. This
% is almost a complete reversal of the jobs from the beat, when the skipper
% concentrates on boat speed and the crew calls the shots and keeps the boat
% level. On the reaches the crew should sit on the windward tank in order to get a
% full view of the chute. The skipper will usually sit on the leeward tank and thus
% have a good look upwind for puffs. Some sailors have the skipper sit on the
% weather tank and the crew on the centerboard trunk. This gives both sailors poor
% views and concentrates weight in the center making the boat more unstable.
% The ultimate thrill in sailing a 470 is trepanning with the chute up. Be prepared to
% go swimming the first couple of times you try it. The skipper should be ready to
% jump all over to keep things under control. The crew must be ready to go from flat
% out on the wire to sitting in the boat, and right back out, while still keeping the
% spinnaker in trim. The trickiest step is getting from the boat to out on the wire.
% Unlike going upwind, you'll have to concentrate on keeping a sail trimmed and
% have one hand full of the sheet. This prevents you from using the trepanning
% handle. Keeping the ring pulled up will allow you to swing out much easier. You
% will also find it necessary to keep your legs spread further apart as reaches are
% bumpier than beats. Remember, as you are trying to swing out, the spinnaker
% sheet will be trying to pull you in ... good luck!
% When flying the chute from the wire, lots of trim adjustments will be necessary.
% As a gust hits and the apparent wind moves aft; the chute may become over
% trimmed. Also, the skipper must be able to bear off rapidly to get the boat under
% control. If the spinnaker is not released, the forces loading up on the rig prevent
% the skipper from bearing off.
% Therefore, as soon as you feel a gust, ease the chute off and extend yourself on
% the wire to keep the boat flat. Try not to release the sheet too much because a
% spinnaker collapse will cause you to roll abruptly to weather. When the boat
% flattens out and starts to plane, the chute will have to be retrimmed. This can be a
% lot of work and you may find that fast reaching legs will leave you quite
% breathless. Jibing a 470 is rather easy under most conditions. Flying jibes are the
% rule. The skipper either grabs the mainsheet parts or the crew grabs the vang,
% and throws the boom across. Jibing in heavy air with the chute up becomes a bit
% trickier because it is a very fast maneuver. Be prepared to the throw your weight
% across the boat to keep it stable. Keep control of the tiller and don't allow the boat
% to round up after the jibe.
% There are two ways to jibe with the spinnaker up. First, the normal jibe which is
% used when jibing on runs or in light air. Second, jibing from reach to reach. The
% reach to reach fast jibe is a high speed technique for jibing from trapeze to
% trapeze.
% For the normal jibe, the skipper informs the crew of the upcoming maneuver and
% takes control of the spinnaker sheet. Spinnaker lines on a 470 are run right
% through the cockpit of the boat. The skipper can reach the sheet right nearby. The
% crew does not pass an end of the sheet back over the traveler and cause an
% unnecessary mess. Once the skipper has the sheet, the guy is taken in the same
% way. The skipper can then straddle the tiller and steer. As the crew jibes the main
% with the vang, the skipper pulls the spinnaker around and flies it while the crew
% jibes the pole.
% This is done by unclipping the spinnaker pole from the mast, clipping it onto the
% new guy, unclipping it from the old guy and rehooking it onto the mast ring. The
% guy is then cleated, the spinnaker is flying, and the twings are reset. The two
% trickiest parts of the jibe involve coordinating weight placement between skipper
% and crew while keeping the chute flying. It helps if the crew, while reattaching the
% pole to the mast ring, pushes it out towards the front and as close to the proper
% height as possible. This will assist the skipper in flying the chute continuously
% through the jibe.
% A good drill involves sailing dead downwind and jibing every 10 boat lengths.
% Also, try flying the chute without the pole as the skipper stands, steers and plays
% the guy and sheet while jibing back and forth.
% The reach to reach jibe is much more demanding and is generally used only once
% during a race (as you round the jibe mark). If done correctly however, it can
% results in passing 3 or 4 boats at that mark. Be forewarned that this type of jibe
% requires quite a bit of practice but is well worth the effort.
% The reach to reach jibe begins with the skipper and crew both being fully hiked
% out and the boat planing on a high reach. The skipper takes the sheet from the
% crew, hikes very hard and bears off slightly to allow the crew to swing in. The
% crew unhooks on the way in, pops the jib out of the cleat and if possible, pulls the
% leeward twing in as much as possible in one tug. The skipper then releases the
% sheet, leaving the chute flailing, bears off and jibes the main by tossing the boom
% across the boat with the mainsheet parts. As the boat is jibing., the crew uncleats
% the (old) windward twing and pulls the chute around the jib by yanking on the old
% guy. Once the main is jibed, the skipper can begin to trim the mainsheet and
% reach across the boat for the spinnaker sheet.
% Meanwhile, the crew moves to the windward side, trims in the (new) windward
% twing all the way and cleats the guy (at the shroud cleat) at a predetermined mark
% that will keep the pole just off the forestay. Finally, the pole is moved to the
% proper side. As soon as the skipper sees the pole hooked to the mast ring, he
% trims in the spinnaker sheet. The crew grabs the handle, pops out, hooks onto the
% trapeze and takes the spinnaker sheet from the skipper. Once things are settled
% down, the skipper and crew both heave a brief sigh of relief, hike out, the skipper
% trims the main in properly. The crew retrims the jib and the boat takes off. That's
% all there is to it.
% Dousing is just the reverse of setting, but easier. The skipper takes the sheet
% from the crew and trims as the crew takes the pole down. As soon as the pole is
% off the mast, the sheet can be released. As the pole is removed and being slid
% into the bilges, the crew grabs the guy. The foot of the sail is gathered and this
% causes the spinnaker to collapse. The crew then yells to the skipper to lower the
% halyard. While the crew stuffs the chute into the bag, the skipper should watch as
% the sail comes down so not to drop it too fast.
% Note that the spinnaker is always taken down on the windward side of the boat.
% heavy weather, the spinnaker can be made more manageable by pulling in both
% twings before the douse.
% A handy racing tip is to set up your boat for the upcoming beat before taking the
% spinnaker down. For example, board down, cunningham in, out haul retightened,
% just to name a few things. Then, after the douse, you can devote your full
% attention to doing a good rounding and getting the boat right on the wind while
% your competitors are flopping around with their heads in their bilges.